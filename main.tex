% Version 2.5 (27/8/17)
% This template was downloaded from:
% http://www.LaTeXTemplates.com
% Version 2.x major modifications by:
% Vel (vel@latextemplates.com)
% This template is based on a template by:
% Steve Gunn (http://users.ecs.soton.ac.uk/srg/softwaretools/document/templates/
%PACKAGES AND OTHER DOCUMENT CONFIGURATIONS

\documentclass[
11pt, % The default document font size, options: 10pt, 11pt, 12pt
oneside, % Two side (alternating margins) for binding by default, uncomment to switch to one side
english, % ngerman for German
onehalfspacing, % Single line spacing, alternatives: onehalfspacing or doublespacing
%draft, % Uncomment to enable draft mode (no pictures, no links, overfull hboxes indicated)
nolistspacing, % If the document is onehalfspacing or doublespacing, uncomment this to set spacing in lists to single
liststotoc, % Uncomment to add the list of figures/tables/etc to the table of contents
%toctotoc, % Uncomment to add the main table of contents to the table of contents
%parskip, % Uncomment to add space between paragraphs
%nohyperref, % Uncomment to not load the hyperref package
headsepline, % Uncomment to get a line under the header
%chapterinoneline, % Uncomment to place the chapter title next to the number on one line
consistentlayout, % Uncomment to change the layout of the declaration, abstract and acknowledgements pages to match the default layout
]{name} % The class file specifying the document structure
\usepackage{siunitx}
%\usepackage[colorinlistoftodos]{todonotes}
\usepackage{expl3}
%\usepackage{algorithm2e}
\usepackage[utf8]{inputenc} % Required for inputting international characters
\usepackage[T1]{fontenc} % Output font encoding for international characters
\usepackage[version=4]{mhchem}
\usepackage{hyperref}
\usepackage{mathpazo}
\usepackage{amsmath}
\usepackage{rotating}
\usepackage{multirow}
\usepackage{adjustbox}
\usepackage[T1]{fontenc}
\usepackage{fourier, heuristica}
\usepackage{array, booktabs}
\usepackage{graphicx}
\usepackage{lmodern}
\setlength{\arrayrulewidth}{0.5mm}
\setlength{\tabcolsep}{8pt}
\renewcommand{\arraystretch}{1.5}
  % hyperlinks
    % simple URL typesetting
% Use the Palatino font by default
%\usepackage{babel}
\usepackage[backend=bibtex,style=numeric-comp,natbib=true,sorting=none]{biblatex} % Use the bibtex backend with the authoryear citation style (which resembles APA)
\addbibresource{Thesis2.bib} % The filename of the bibliography
\usepackage[autostyle=true]{csquotes} % Required to generate language-dependent quotes in the bibliography

%	MARGIN SETTINGS
\geometry{paper=a4paper, % Change to letterpaper for US letter
	inner=2.5cm, % Inner margin
	outer=3.8cm, % Outer margin
	bindingoffset=0.5cm, % Binding offset
	top=1.5cm, % Top margin
	bottom=1.5cm} % Bottom margin
	%showframe, % Uncomment to show how the type block is set on the page
	
\begin{document}

%	THESIS INFORMATION
\thesistitle{New Cathodes for Rechargeable Non-Aqueous Aluminium-Ion Batteries} % Your thesis title, this is used in the title and abstract, print it elsewhere with \ttitle
\supervisor{Thomas \textsc{Nann}\\ Jim \textsc{Johnston}} 
% Your supervisor's name, this is used in the title page, print it elsewhere with \supname
%\examiner{} % Your examiner's name, this is not currently used anywhere in the template, print it elsewhere with \examname
\degree{Doctor of Philosophy} % Your degree name, this is used in the title page and abstract, print it elsewhere with \degreename
\author{Shalini \textsc{Divya}} % Your name, this is used in the title page and abstract, print it elsewhere with \authorname
%\addresses{} % Your address, this is not currently used anywhere in the template, print it elsewhere with \addressname

\subject{Chemistry} % Your subject area, this is not currently used anywhere in the template, print it elsewhere with \subjectname
%\keywords{} % Keywords for your thesis, this is not currently used anywhere in the template, print it elsewhere with \keywordnames
\university{\href{http://www.vuw.ac.nz}{Victoria University of Wellington}} % Your university's name and URL, this is used in the title page and abstract, print it elsewhere with \univname
\department{School of Chemical and Physical Sciences} % Your department's name and URL, this is used in the title page and abstract, print it elsewhere with \deptname
%\group{\href{http://researchgroup.university.com}{Research Group Name}} % Your research group's name and URL, this is used in the title page, print it elsewhere with \groupname
 % Your faculty's name and URL, this is used in the title page and abstract, print it elsewhere with \facname


\frontmatter % Use roman page numbering style (i, ii, iii, iv...) for the pre-content pages
\pagestyle{plain} % Default to the plain heading style until the thesis style is called for the body content

%	TITLE PAGE
\begin{titlepage}
\begin{center}
\vspace*{.06\textheight}
{\scshape\LARGE \univname\par}\vspace{1.5cm} % University name
\textsc{\Large Doctoral Thesis}\\[0.5cm] % Thesis type
\HRule \\[0.4cm] % Horizontal line
{\huge \bfseries \ttitle\par}\vspace{0.4cm} % Thesis title
\HRule \\[1.5cm] % Horizontal line
 \begin{minipage}[t]{0.4\textwidth}
\begin{flushleft} \large
\emph{Author:}\\
\href{http://www.johnsmith.com}{\authorname} % Author name - remove the \href bracket to remove the link
\end{flushleft}
\end{minipage}
\begin{minipage}[t]{0.4\textwidth}
\begin{flushright} \large
\emph{Supervisors:} \\
\href{http://www.jamessmith.com}{\supname}% Supervisor name - remove the \href bracket to remove the link  
\end{flushright}
\end{minipage}\\[3cm]
 \vfill
\large \textit{A thesis submitted in fulfillment of the requirements\\ for the degree of \degreename}\\[0.3cm] % University requirement text
\textit{in the}\\[0.4cm]
\deptname \\[2cm] % Research group name and department name
 
\vfill

{\large \today}\\[4cm] % Date
%\includegraphics{Logo} % University/department logo - uncomment to place it
 \vfill
\end{center}
\end{titlepage}

%	DECLARATION PAGE
\begin{declaration}
\addchaptertocentry{\authorshipname} % Add the declaration to the table of contents
\noindent I, \authorname, declare that this thesis titled, \enquote{\ttitle} and the work presented in it are my own. I confirm that:

\begin{itemize} 
\item This work was done wholly or mainly while in candidature for a research degree at this University.
\item Where any part of this thesis has previously been submitted for a degree or any other qualification at this University or any other institution, this has been clearly stated.
\item Where I have consulted the published work of others, this is always clearly attributed.
\item Where I have quoted from the work of others, the source is always given. With the exception of such quotations, this thesis is entirely my own work.
\item I have acknowledged all main sources of help.
\item Where the thesis is based on work done by myself jointly with others, I have made clear exactly what was done by others and what I have contributed myself.\\
\end{itemize}
 
\noindent Signed:\\
\rule[0.5em]{25em}{0.5pt} % This prints a line for the signature
 \noindent Date:\\
\rule[0.5em]{25em}{0.5pt} % This prints a line to write the date
\end{declaration}
\cleardoublepage

%	QUOTATION PAGE
\vspace*{0.2\textheight}
\noindent\enquote{\itshape Where the mind is without fear and the head is held high. Where knowledge is free.}\bigbreak
\hfill Rabindranath Tagore

\newpage
%	ABSTRACT PAGE
\begin{abstract}
\addchaptertocentry{\abstractname} % Add the abstract to the table of contents
Lithium-ion batteries (LIBs) are a popular battery-choice for most applications. However, future battery demand will place increasing pressure on lithium and cobalt reserves and supply lines in the medium and long term. Moreover, the electrolyte used in LIBs is flammable, which is a safety issue that has to be managed. Any mechanical damage to the cell might result in short circuits or thermal runaway reactions, sometimes leading to an explosion.\\
High abundance and easy accessibility of aluminium resources enable aluminium-ion batteries (AIBs), offer the opportunity to become an ideal alternative. Since a multivalent ion insertion is feasible, higher energy densities can be achieved. Non-aqueous AIBs use a non-flammable ionic liquid as their electrolyte, making them safer than LIBs in this regard. The most common electrolyte for AIBs is currently the ionic liquid 1-ethyl-3-methyl imidazolium chloride ([EMIm]Cl) mixed with aluminium trichloride (\ce{AlCl3}) where a chloroaluminate ion is the active charge-carrying species.\\
This dissertation focuses on discovering new cathode materials. The author aimed to achieve an AIB with a superior performance over those previously reported. Two-dimensional (2D) layered materials showed potential as active cathode materials since they display similar properties as graphite, which is a popular choice for AIBs. Transition metal dichalcogenides, carbon-based materials, and a few others (oxides, carbides, nitrides) were tested. Results showed that some of the materials displayed high stability and long cycle life and outperformed cathodes existing in current AIB literature. A few of them were thoroughly investigated to establish the underlying mechanism. A novel cathode material was patented and is currently in the process of commercialisation!
\end{abstract}
%	ACKNOWLEDGEMENTS
\begin{acknowledgements}
\addchaptertocentry{\acknowledgementname} 
There are no proper words to express my sincere gratitude and respect to my research advisor and supervisor Prof. Thomas Nann for his continuous support during the course of my Ph.D. and related research. For his patience, immense knowledge and endless motivation that helped me strive forward. I could not have imagined having a better advisor and mentor for my Ph.D. study.
I am forever indebted to Prof. James H. Johnston who agreed to be my supervisor after Thomas shifted to Australia. He continuously guided and encouraged me to be professional and do the right thing even when the roads got tough. It is wholeheartedly appreciated that your great advice for my study proved monumental towards the success of this study. 

Besides my advisors, I would like to thank the rest of my thesis committee: Prof. X, Prof. Y, and Dr. Z, for their insightful comments and encouragement, but also for the hard questions, which incented me to widen my research from various perspectives.

I am grateful to all my collaborators with whom I have had the pleasure to work during this and other related projects: Yuta Nakayasu (Assistant professor at Tohoku University, Japan), Nonglak Meethong (Assistant Professor at Khong Kaen University, Thailand), Geoffrey Waterhouse (Associate Professor at University of Auckland, New Zealand), Sara Cavaliere (Lecturer at the University of Montpellier, France) and Ossie Amir (CEO, Carbon Valley, New Zealand).

Mr. Colin Doyle gave me invaluable help with data and statistics for X-ray photoelectron analysis (XPS) which I used in my project. I wish to show my appreciation to Mr. David Flynn, who helped me with various microscopic imaging (SEM and TEM).

I would like to express my gratitude to Mrs. Jayoti Chakraborty, who was a wonderful teacher and inspired me to study chemistry.

Heartfelt thanks to my fellow lab mates and friends, from past and present, Erin, Garima, Geoffry, Jake, Jacob, Moritz, Rohan and Vaibhav for the thought-provoking discussions, and for all the fun we have had in the last three years. In particular, I am grateful to Dr. Nicolo Canever, whose assistance was a milestone in the completion of this dissertation. Their presence was very important as they are the ones with whom I have shared moments of deep anxiety but also of big excitement. Special thanks to my friends from New Zealand: Abhi, Chriselle, Isabella, Kai, Leini, Parth, Ronnie, Sreelakshmi, and Tehreema for offering me advice, and supporting me throughout this entire process. \\
The final words in acknowledgment are usually reserved for those dearest to the author. I do not wish to break this tradition. A warm word for my colleague and great friend Fraser, whose words never failed to lift my spirits amid my discouragement. I would like to pay my special regards to Dr. Poulomi Roy for guiding me towards the right path, the same way she did during my Master's thesis. She taught me so much, but also went beyond that to show her love and care. 

Some special words of gratitude go to my friends who have always had my back when things would get a bit disheartening: Arti, Baibhaw, Harsh, and Saket. Thank you for always being there for me. I deeply thank Megha and Payal, my best friends for the past 20 years, who have loved, entertained, and encouraged me to get through this period in the most positive way.
I wish to acknowledge the support and great love of my family, my father, Roy Upendra, my sister, Nishika and her husband, Arindam. Thank you for teaching me that my job in life was to learn and most importantly to be happy. This thesis would not have been possible without their endless love and care.\\
The last word goes for Sangeeta, my mother, for always showing how proud she is of me for the last three years and who has given me the extra strength and motivation to get things done. This thesis is dedicated to her.

\end{acknowledgements}

%	LIST OF CONTENTS/FIGURES/TABLES PAGES
\tableofcontents % Prints the main table of contents
\listoffigures % Prints the list of figures
\listoftables % Prints the list of tables

%	ABBREVIATIONS
%\begin{abbreviations}{ll} % Include a list of abbreviations (a table of two columns)
%\textbf{LAH} & \textbf{L}ist \textbf{A}bbreviations \textbf{H}ere\\
%\textbf{WSF} & \textbf{W}hat (it) \textbf{S}tands \textbf{F}or\\
%\end{abbreviations}

%	PHYSICAL CONSTANTS/OTHER DEFINITIONS
%\begin{constants}{lr@{${}={}$}l} % The list of physical constants is a three column table

% The \SI{}{} command is provided by the siunitx package, see its documentation for instructions on how to use it
%Speed of Light & $c_{0}$ & \SI{2.99792458e8}{\meter\per\second} (exact)\\
%Constant Name & $Symbol$ & $Constant Value$ with units\\
%\end{constants}

%	SYMBOLS
%\begin{symbols}{lll} % Include a list of Symbols (a three column table)
%a$ & distance & \si{\meter} \\
%$P$ & power & \si{\watt} (\si{\joule\per\second}) \\
%Symbol & Name & Unit \\

%\addlinespace % Gap to separate the Roman symbols from the Greek
%$\omega$ & angular frequency & \si{\radian} \\
%\end{symbols}

%	DEDICATION
\dedicatory{
{\Huge To my mother... because I owe it all to you!}
} 

%	THESIS CONTENT - CHAPTERS
\mainmatter % Begin numeric (1,2,3...) page numbering
\pagestyle{thesis} % Return the page headers back to the "thesis" style

% Include the chapters of the thesis as separate files from the Chapters folder
% Uncomment the lines as you write the chapters
%\section*{\centering Aims and objectives}
This PhD dissertation aims to discover new cathode materials for rechargeable non-aqueous aluminium-ion batteries with improved specific capacity and battery voltage than state-of-the-art. The goal raises the following research objectives:
\begin{itemize}
    \item to test layered-type structures (transition-metal dichalcogenides, main group oxides, carbides and nitrides) as cathodes 
    \item to prepare carbonised natural products and other forms of carbon with high surface area (other than graphite) and test them as cathodes 
    \item to investigate the mechanism behind all successful cathodes using analytical tools such as X-ray diffraction, Raman spectroscopy, and X-ray photoelectron spectroscopy
    \item to reduce the cost of existing prototype of an aluminum-ion battery by testing different solvents and current collectors during cell assembly
    
\end{itemize}
\newpage
\newpage
\section*{\centering Thesis structure}
\begin{itemize}
    \item \textbf{Chapter 1}: This chapter gives a brief introduction on batteries and the terms associated with understanding a battery technology. Lithium-ion batteries and its shortcomings are discussed, and a comparison between batteries that currently exist in the market is made. Aluminium-ion batteries, both aqueous and non-aqueous, are introduced and ways of finding new cathode materials that can be used in aluminium-ion batteries is explored.
    \item \textbf{Chapter 2}: This chapter explains the experimental methods carried out to assemble a battery on a lab-scale. Procedures for preparing cathode slurries and electrolytes for an aluminum-ion cell have been briefly described.  
    \item \textbf{Chapter 3}: This chapter discusses the characterisation techniques that were implemented post-mortem, to fully analyse how a battery works. Electrochemical processes such as cyclic voltammetry and galvanostatic charge/ discharge curves have been discussed in detail.   
    \item \textbf{Chapter 4, 5, 6 and 7}: These chapters discuss the new materials that were tested as cathodes in aluminium-ion batteries. With a brief review of the literature, new batteries were made using molybdenum dichalcogenides (Chapter 4), carbon-based materials (Chapter 5) and boron nitride/oxide (Chapter 6) as cathodes. Results of several other two-dimensional materials have been reported in Chapter 7. 
    \item \textbf{Chapter 8}: To find cheaper alternatives to the state-of-the-art, new solvents and current collectors were use while preparing cathodes and their performance was recorded.   
    \item \textbf{Chapter 9}: This chapter summarises the research findings of chapters 4-8 and provides an outlook for future researches. Many new scientific findings have been made, which need to be studied and analysed in greater detail, so that aluminium-ion batteries can find commercial use.
    \end{itemize}

%\section*{\centering Conferences}
\section*{International}
\begin{enumerate}
    \item \textbf{S. Divya} and T. Nann \enquote{Aluminum-Ion Batteries}, Pacific Climate Change Conference (\textbf{PCCC}), 2018, New Zealand.
    \item \textbf{S. Divya} and T. Nann- \enquote{New Cathodes for Aluminum-Ion Batteries.} 9th International Conference on Advanced Materials and Nanotechnology (\textbf{AMN9}), 2019, New Zealand.
    \item \textbf{S. Divya} and T. Nann- \enquote{Cathodes for Aluminum-Ion Batteries.} 10th International Conference on Materials for Advanced Technologies (\textbf{ICMAT}) by MRS Singapore, 2019, Singapore.
     \item \textbf{S. Divya} and T. Nann- \enquote{New Carbon Cathodes for Aluminum-Ion Batteries.} New Zealand Institute of Chemistry Conference (\textbf{NZIC}), 2019, New Zealand.
     \item \textbf{S. Divya}, J. Johnston and T. Nann- \enquote{New Cathodes for Rechargeable Aluminium-Ion Batteries.} TechConnect 2020, Washington D.C., USA, 2020. Abstract submitted.
\end{enumerate}
\section*{Domestic}   
\begin{enumerate}
     \item \textbf{S. Divya} and T. Nann- \enquote{New Cathodes for Aluminum-Ion Batteries.} Victoria University of Wellington-Massey University Symposium, 2018, New Zealand.
      \item \textbf{S. Divya} and T. Nann- \enquote{New Cathodes for Aluminum-Ion Batteries.} Nobel Prize Public Lecture at Victoria University of Wellington, 2019, New Zealand.
      \item \textbf{S. Divya}, J. Johnston and T. Nann- \enquote{Aluminum-Ion Batteries.}, Science Wairarapa, \textit{Invited talk}, May 2020, New Zealand.
\end{enumerate}

\section*{\centering Publications}
\begin{enumerate}
    \item \textbf{Shalini Divya}, James H.Johnston,and Thomas Nann.“Molybdenum Dichalcogenide Cathodes for Aluminium-Ion Batteries”. In: ArXiv191210607 Cond- MatPhysicsphysics (Dec. 2019) %\cite{divya_molybdenum_2019-1}
    \item \textbf{Shalini Divya} and Thomas Nann.“Carbon Cathodes for Rechargeable Aluminium-Ion Batteries”, to be submitted.
    \item \textbf{Shalini Divya}, Yuta Nakayasu,and Thomas Nann.“Molybdenum Dichalcogenide Nanoflowers as Cathodes for Non-Aqueous Aluminium-Ion Batteries”, to be submitted.
\end{enumerate}
\section*{\centering Patent}

%% Chapter 1
\chapter{Batteries --- an introduction} % Main chapter title
 \label{chap1} % For referencing the chapter elsewhere, use \ref{Chapter1} 
%----------------------------------------------------------------------------------------
% Define some commands to keep the formatting separated from the content 
\newcommand{\keyword}[1]{\textbf{#1}}
\newcommand{\tabhead}[1]{\textbf{#1}}
\newcommand{\code}[1]{\texttt{#1}}
\newcommand{\file}[1]{\texttt{\bfseries#1}}
\newcommand{\option}[1]{\texttt{\itshape#1}}

%----------------------------------------------------------------------------------------
To combat climate change, significant number of steps have been taken to switch over to renewable sources of energy from non-renewables. By installing solar panels on their rooftops, consumers are now powering their houses by trapping solar energy. Figure \ref{Figures/chap1fig:iec} a and b displays the data from 2018 generated by International Energy Association (IEA) showing the energy consumption rate in terms of terawatt hour (TWh) per year. The growing population is responsible for the increase in energy demands. Figure \ref{Figures/chap1fig:iec} b shows that the share of modern renewable sources of energy has also increased and there's a sharp increase in their share from the year 2005 to 2016. However, solar and wind energies have variable output. The sun doesn't always shine and the wind doesn't always blow. Energy storage is necessary for a continuous power supply. Batteries maximize the ability to use the electricity generated by renewable sources of energy, on a day-to-day basis. They store energy in a chemical form, which can be used at a later time. If used in houses, a battery can store the power generated by the sun for several hours. In a grid system, when the supply is higher than demand, electricity can be used to charge batteries. When the demand is higher than supply, stored energy can be used by the grid and distributed. To understand how a battery works, it is important to understand a few terms that are helpful in evaluating its performance.  

\begin{figure}[tbh!]
\centering
\includegraphics[width=\textwidth]{Figures/chap1fig/iec.pdf}
\caption{Based on International Energy Agency (IEA) data from 2018 monthly oil data service. Significant increase can be seen in recent years in energy consumption and share of modern renewables. This means in spite  of increase in energy consumption, there is also an awareness about using non-renewable sources of energy, www.iea.org/statistics. All rights reserved.}
\label{Figures/chap1fig:iec}
\end{figure}

\begin{itemize}
\item \textbf{Battery capacity}: Capacity is the amount of charge or energy stored in a battery. Mathematically, it is evaluated by integrating current over time. The fundamental unit of battery capacity is coulomb (C), though Ampere-hours (Ah) is more commonly used. Theoretical capacity (the ideal capacity a battery can store) is calculated with the help of chemical reactions that take place inside the cell. Using Faraday's constant (F = 96,484.56 C mol$^{-1}$), theoretical capacity of a battery can be determined using Equation \ref{eq1}:

\begin{equation} \label{eq1}
  \text{Capacity(Ah) = } \frac{n \times F \times 1 \text{ hour}}{3600 \text{sec}}\\
\end{equation}
where n = number of electrons participating in the chemical reaction \\
F = Faraday's constant. \\ Specific capacity is the capacity stored in a material per unit mass. The most commonly used unit for specific capacity is mAh g$^{-1}$. 
\item \textbf{Battery potential or voltage}: This is the point (usually in the middle of a discharge curve), where voltage stays constant for the longest period forming a plateau. Various factors such as electrolyte stability, polarization of the battery (displacement of electrode potential from the equilibrium value) and concentrations of the active species help in determining a cell's voltage. Figure \ref{Figures/chap1fig:CDCforcellvoltage} represents a typical charge/ discharge curve. To avoid any permanent damage, a battery should not be discharged below or charged above a certain level. These voltages are called the "cut-off voltages". Going beyond the upper or lower cut-off voltages might lead to certain reactions that decompose the electrolyte (also known as side reactions) resulting in an irreversible capacity loss. Sometimes however, a distinct discharge plateau is not observed. In that case, the average potential of the cell is considered to be the intersection of the charging and discharging curves, displayed in Figure \ref{Figures/chap1fig:batpot}. 

\begin{figure}[h!]
\centering
\includegraphics[width=\textwidth]{Figures/chap1fig/CDCforcellvoltage}
\caption{A charge and discharge curve of an aluminium-ion cell using graphite as the cathode and pure aluminium as the anode. The cell was charged and discharged to 2.35 V and 0.2 V respectively. The plateau is observed at 1.75 V, which is also referred to as the nominal voltage.}
\label{Figures/chap1fig:CDCforcellvoltage}
\end{figure}

\begin{figure}[tbh!]
\centering
\includegraphics[width=\textwidth]{Figures/chap1fig/batpot.pdf}
\caption{A charge and discharge curve of an aluminium-ion cell charged and discharged to 2.35 V and 0.2 V respectively. The point of intersection of the charge and discharge curves at 0.9 V will be considered as the battery voltage in case a voltage plateau is absent.}
\label{Figures/chap1fig:batpot}
\end{figure}

\item \textbf{Energy density}: The amount of energy stored in a battery per unit mass or volume is called its energy density. Sometimes, heavy batteries are required to move something as large as a car over long distances, therefore they use batteries with high energy density. A simple way to determine the specific energy or energy density of a battery is using Equation \ref{eq2}:
\begin{equation} \label{eq2}
    \text{Energy density (Wh) = } \text{Battery capacity (Ah)} \times \text{Battery voltage (V)}
\end{equation}

A higher battery capacity or voltage or both are the prerequisites of a good battery system. 

\item \textbf{Power density}: Power density measures how quickly a battery can deliver energy. Also known as specific power, it is equivalent to the maximum current one can draw from a battery. Units used to describe power density are W kg$^{-1}$ or W m$^{-3}$. The best way to differentiate between energy and power density of a battery is to use an example of a moving car. Energy density determines how 'far' the car will go, whereas power density determines how 'fast' the car will go.

\item \textbf{Coulombic efficiency (CE)}: Coulombic efficiency of a battery is the ratio of number of charges that enter during charge to the number that can be extracted from the battery during discharge. A high CE in excess of 95\% is considered a standard value for commercial battery systems. 
\end{itemize}

Batteries can be categorised as primary (non-rechargeable) or secondary (rechargeable). \textbf{Primary} batteries produce current immediately when assembled in a charged state, and connected to an external circuit. Since these battery systems cannot be recharged again, they have high energy densities for one-time use. Common types of primary batteries include zinc-carbon and alkaline batteries. 
\textbf{Secondary} batteries need to be charged before their first use. They are assembled in the discharged state and applying a charging current reverses the cell's active materials chemical state. They find extensive use in portable devices as they store energy that can be drawn after every charge/ discharge cycle. The most commonly used examples of rechargeable batteries are nickel-cadmium (Ni-Cad), nickel metal hydride (NiMH) and lithium-ion batteries (LIBs). \\  
In addition, charging and discharging rates affect the battery capacity. If a high discharge current is applied (i.e. the battery is being discharged very quickly) the amount of energy that can be extracted from the battery is reduced and its capacity decreases. This is because the reactions that take place inside a battery are not completed and only a fraction of the total reactants are converted to the final product and therefore the battery becomes less efficient. Alternately, if a battery is discharged using low current, more energy can be extracted from the battery and its capacity is higher. The battery temperature also affects the energy that can be extracted. At a higher temperature, the capacity is typically higher than at a lower temperature. However, intentionally elevating battery temperature is not an effective method as this might decrease battery lifetime \cite{leng_effect_2015, ma_temperature_2018}. 
An ideal battery should be low-cost, charge and discharge indefinitely under high or low current rate, have a long lifetime with high CE (>99\%), and experience low-self discharge. However, it is difficult to fulfil all of the above set of requirements. Researchers are building new batteries that might achieve these objectives in their own ways \cite{slater_sodium-ion_2013, jian_carbon_2015, aurbach_prototype_2000, lin_ultrafast_2015}. Every appliance that uses rechargeable batteries has its own specifications. Portable electronic items need faster charging rates, therefore LIBs are used extensively. On the other hand, NiMH and Ni-Cad batteries are more cost-effective than LIBs and have been used widely in making electric vehicles. Although, the cost of a hybrid car that uses Ni-Cad or NiMH battery starts from \$23,000 and a Tesla that uses LIBs starts from \$35,000. \\

Table\ref{table1} compares a few characteristics of rechargeable batteries that currently exist in the market. 

\begin{sidewaystable}
\centering
\caption{Characteristics of commonly used rechargeable batteries.} \label{table1}
\begin{tabular}{ |p{3.5cm}|p{2cm}|p{2cm}|p{2cm}|p{4.5cm}|p{4.5cm}|}
 \hline 
\textbf{Battery type} & \textbf{In market since} & \textbf{Energy density} & \textbf{Nominal voltage} & \textbf{Applications} & \textbf{Limitations}\\ 
\textbf{} & \textbf{} & \textbf{(Wh kg$^{-1}$)} & \textbf{V} & \textbf{} & \textbf{}\\ 
\hline
Lead-acid & 1881 & 30-50 & 2.0 & backup power supplies for telephone and computer centres, grid energy storage, uninterrupted power supply (UPS), marine applications- submarines & Environmental hazard, low energy density, risk of thermal runaway, transportation restrictions\\
Nickel-cadmium (Ni-Cad) & 1960 & 40-80 & 1.2 & portable electronics, toys, cordless telephones & Environmental hazard, low energy density, high self-discharge, explosive\\
Nickel metal hydride (NiMH) & 1990 & 60-120 & 1.2 & Consumer electronics, electric vehicles, hybrid cars & Expensive, high self-discharge, high maintenance\\
Lithium-ion (\ce{LiCoO2}) & 1991 & 150-190 & 3.6 & Smartphones, laptops, tablets, digital cameras, hybrid vehicles, electric motorcycles, scooters, bicycles, personal transporters & Safety hazard, risk of thermal runaway, transport restrictions, environmental hazard\\
Lithium-ion (\ce{LiMn2O4}) & 1996 & 100-135 & 3.8 & Same as above & Same as above\\
Lithium-ion (\ce{LiFePO4}) & 1999 & 90-120 & 3.3 & Same as above & Same as above\\
Lithium-ion (LiNi$_{x}$Co{$_{1-x-y}$}O$_{2}$, LiNMC) & 2008 & 190-210 & 3.6 & Same as above & Same as above\\
\hline
\end{tabular}
\end{sidewaystable}

\newpage

\section{Lithium-ion battery (LIB)}
\begin{figure}[tbh!]
\centering
\includegraphics[width=\textwidth]{Figures/chap1fig/LIB}
\caption{A typical lithium-ion cell. Ion insertion or intercalation of \ce{Li+} ions is shown during cell discharge and de-intercalation during charge. Graphite is used as the anode (coated on a copper current collector) and \ce{LiCoO2} is used as the cathode (coated on an aluminium current collector).}
\label{Figures/chap1fig:LIB}
\end{figure}

The standard potentials of a pair of half reactions determines whether a voltage is generated in a redox reaction or not. If the difference between standard potentials is positive, then the reaction will proceed spontaneously, which is precisely what is needed in a battery. Therefore, standard potential is an important parameter while finding a suitable battery anode \cite{liu_understanding_2016}. Out of all the metals that can be used in a battery system listed in Table \ref{table2}, Lithium has the most negative standard reduction potential (-3.04 V), which enables it to achieve very high energy and power densities when paired with many different materials (refer to equation \ref{eq2}). This is the reason why LIBs are a popular battery-choice for most applications.

\begin{figure}[tbh!]
\centering
\includegraphics[width=\textwidth]{Figures/chap1fig/energy}
\caption{An energy level diagram involved in an electrochemical cell. The dashed red lines correspond to the chemical potentials of the anode and the cathode. The difference is termed as the working voltage, also known as the open circuit voltage (V$_{oc}$). The energy gap between the highest occupied molecular orbital (HOMO) and the lowest unoccupied molecular orbital (LUMO) determines the electrochemical window of the electrolyte \cite{liu_understanding_2016}.}
\label{Figures/chap1fig:energy}
\end{figure}

\subsection*{Mechanism of a lithium-ion battery}
The cathode typically consists of lithium cobalt oxide \ce{LiCoO2} or lithium iron phosphate \ce{LiFePO4}. The anode is generally made from graphite and the electrolyte varies from one type of lithium battery to another. All LIBs have a similar working principle. During charging process, the two electrodes are connected externally to an external electrical supply. The electrons are forced to be released at the cathode and move externally to the anode. Simultaneously the lithium ions move in the same direction, but internally, from cathode to anode via the electrolyte. In this way the external energy is electrochemically stored in the battery in the form of chemical energy in the anode and cathode materials that have different chemical potentials. The anode and cathode must be selected such that the $\mu$A of the anode lies below the LUMO and the $\mu$C of the cathode is located above the HOMO; otherwise, the electrolyte will be reduced on the anode or oxidized on the cathode to form a passivating solid electrolyte interphase (SEI) film \cite{goodenough_challenges_2010}. The opposite occurs during discharge-- electrons move from the anode to the cathode through the external load to do work and \ce{Li+} ions move from anode to the cathode through the electrolyte. This is also known as \enquote{rocking chair} mechanism, where the \ce{Li+} ions shuttle between the anode and cathodes during charge and discharge cycles. Electrochemical reactions at the two electrodes release the stored chemical energy \cite{deng_li-ion_2015}. The overall reaction during discharge is given in Equation \ref{eql} and it is reversed during charge. 
\begin{equation}\label{eql}
    \ce{CoO2 + LiC6 -> LiCoO2 + C6}
\end{equation}

\vspace{3mm}
%A lithium-based battery is a power pack of choice not only on the Earth, but also in space. In October 2019, a few astronauts on board the International Space Station (ISS) stepped outside their quarters for a spacewalk. Flight engineers Christina Koch and Jessica Meir were assigned the task of manually swapping out two nickel hydrogen (NiH) batteries for one brand new LIB. It was also the first ever all-female spacewalk in human history! The battery replacement would not only upgrade the station's electrical system but also extend it's life, at least through 2020's. ISS was launched into orbit in 1998 with 48 NiH batteries. The National Aeronautics and Space Administration (NASA) has started swapping these old batteries, since 2017, with 24 new LIBs  that provide higher energy density and a better power efficiency. Naturally, they had to be careful while handling these heavy batteries (195 kg). LIBs come with a potential risk of thermal runaway, which means an increase in temperature leads to conditions that cause a further increase in temperature inside the batteries, often leading to destructive results. Inside a pressurised oxygen-rich capsule, that would have been catastrophic.\\
The electrolyte is a chemical medium that allows the flow of charge between the cathode and anode. It promotes movement of ions from the cathode to the anode. It is usually made of soluble salts, acids or other bases in liquid, gelled or dry media, a polymer, or ionic liquids (ILs) \cite{xu_nonaqueous_2004, armand_ionic-liquid_2009, croce_nanocomposite_1998}. The ions transport charge through the electrolyte, while the electrons flow in the external circuit generating an electric current. LIBs use lithium hexafluorophosphate (\ce{LiPF6}) and mixture of carbonate solvents such as dimethyl carbonate (DMC), ethyl methyl carbonate (EMC), or diethyl carbonate (DEC) as electrolytes. They are a less viscous electrolyte and enhance its conductivity as a result. However, they are flammable and show flash points around room temperature (between 16 and 33$^{\circ}$C). In combination with an oxidant and an ignition source, they may catch fire and cause explosions. In addition, future battery demand will place increasing pressure on lithium and cobalt reserves \cite{turcheniuk_ten_2018}. To find a suitable alternative, one needs to examine theoretical specific capacities of different ions that can replace lithium. Metals in the upper left corner of the periodic table, such as sodium (Na), magnesium (Mg), potassium (K) and calcium (Ca) report higher theoretical capacities than other metals and can be alternatively used as battery anodes. Table  \ref{table2} compares the metrics of a few potential metal anodes including aluminium. With three-electron redox properties (\ce{Al^3+}/Al), aluminium has high specific capacity per mass unit (2980 Ah kg$^{-1}$) and the highest capacity per unit of volume (8046 Ah L$^{-1}$) \cite{ambroz_trends_2017}.

\begin{table}[tbh!]
\centering
\caption{Comparing important parameters of various metal anodes.} \label{table2}
\begin{tabular}{|p{2cm}cccccc|}
\hline
 & \textbf{Li} & \textbf{Na} & \textbf{Mg} & \textbf{Al} & \textbf{K} & \textbf{Ca}\\
\hline
Valence electrons & 1 & 1 & 2 & 3 & 1 & 2\\
Specific capacity (mAh g$^{-1}$) & 3862 & 1166 & 2205 & 2980 & 685 & 1340\\
Standard reduction potential (V) & -3.04 & -2.71 & -2.36  & -1.68 & -2.93 & -2.87\\
Crustal abundance (ppm) & 1.8x 10$^{4}$ & 2.2 x 10$^{4}$ & 2.3 x 10$^{4}$ & 8.2 x 10$^{4}$ & 1.8 x 10$^{4}$ & 4.1 x 10$^{4}$\\
\hline  % Please only put a hline at the end of the table
\end{tabular}
\end{table}

High abundance and easy accessibility of aluminium resources enable aluminium-ion batteries (AIBs), together with their electrochemical characteristics, offer an opportunity to become the ideal alternative.

\section{Aluminium-ion batteries (AIBs)}
The second most promising metal despite having a more positive standard reduction potential than lithium, is aluminium (Al). AIBs use aluminium metal as an anode, which makes it cost-effective, recyclable, and environmentally friendly. Due to the presence of three electrons in its valence shell that can easily participate in an electron transfer process, a multi-electron reaction is feasible, which increases its theoretical energy density (Table \ref{table2}). Moreover, the high crustal abundance and easy accessibility of aluminium resources enable AIBs to become an ideal candidate for large-scale energy storage systems. \\
The idea to use aluminium in batteries was born in the late 1800's. Inventors like Joseph Richards, in 1890, and James Sully, in 1897, worked on primary batteries using a carbon-aluminum electrode \cite{noauthor_james_1897, richards_aluminium_1890}. These cells maintained a nearly constant electromotive force (EMF) on a closed circuit for several weeks at a time. The negative electrode was externally exposed to air and a mixture of potassium carbonate (\ce{K2CO3}) and kerosene oil was used as the electrolyte. Their motive was to provide an aluminum dry cell with a longer shelf life. Another attempt was made to make commercially viable aluminium batteries in the 1950s. In 1951, Donald Sargent reported a voltaic cell especially adapted for use as a dry cell. The negative electrode consisted of aluminium and the electrolyte was made of a mixture of zinc oxide (ZnO) and sodium hydroxide (NaOH) with carbon acting as the positive electrode \cite{sargent_voltaic_1951-2}. The ZnO/NaOH system suffered similar problems of formation of a layer of aluminium oxide \ce{Al2O3}, also known as \enquote{passivation}, leading to very low energy densities and hence could not be commercialised. Aluminium–air batteries produce electricity from the reaction of oxygen in the air with aluminium. The cathode is immersed in a water-based electrolyte (alkali metal salts) and forms hydrated \ce{Al2O3}. Once the Al anode is consumed by its reaction with atmospheric oxygen at the cathode, the battery can no longer operate. One of the first aluminium-air batteries were reported by Zaromb in the 1960s \cite{zaromb_use_1962}. He found that addition of zinc oxide or ammonium salts to the electrolyte reduced the rate of aluminum corrosion in the strongly basic electrolytes. However, the nature of the stabilization was irregular and sometimes the corrosion rate would experience a sudden increase making the battery unstable \cite{bockstie_control_1963}. Using molten salt media generally makes the charge transfer process fast and the cathodic deposition is diffusion controlled, which shows that aluminum can be reversible electrodeposited from both aqueous and non-aqueous electrolytes \cite{li_aluminum_2002}. This showed that aluminum might be used as an anode in rechargeable batteries. However, the passivating oxide layer on the anode surface was still a matter of concern and was continuously deteriorating the battery performance. The passivation caused a decrease in the electrode potential resulting in lower cell voltages than the expected theoretical value. The oxide layer further inhibited the cell performance by causing a \enquote{delayed action}, which is a time delay during which the cell reaches its operating voltage during discharge. This delay was due to the gradual breakdown and removal of the oxide layer from the aluminum anode surface, which also affected the anode activation. Since the standard electrode potential of \ce{Al^3+}/Al at -1.68 V is lower than \ce{H+}/\ce{H2}, evolution of \ce{H2} gas occurs when Al foil reacts with aqueous acid or alkali solution. Thus, Al is unable to undergo electrochemical stripping or deposition in a common aqueous solution \cite{wu_electrochemically_2019}. Finally in 2010, Paranthaman \textit{et al.} made the first rechargeable aluminum-ion battery using a cathode composed of manganese (IV) oxide (\ce{MnO2}, \ce{Mn2O4}) and an ionic liquid (IL) as the electrolyte, based on the works by Jiang \textit{et al.} and Peng \textit{et al.} \cite{paranthaman_transformational_2010, jiang_electrodeposition_2006, peng_investigation_2008}. 
The electrolyte was made of aluminium trichloride (\ce{AlCl3}) and 1-Ethyl-3-methylimidazolium chloride (EMImCl) (discovered by Gillford \textit{et al.} \cite{gifford_aluminum/chlorine_1988}) in a ratio of 2:1. The higher concentration of \ce{AlCl3} makes the ionic liquid (IL) a Lewis acid, which prevented Al passivation. 
Reversible deposition and dissolution of Al occurs in non-aqueous electrolytes such as molten salts NaCl-\ce{AlCl3} or ILs (quaternary ammonium species) at room temperature with no passive oxide layer being formed on the aluminum anode \cite{vestergaard_molten_1993, galinski_ionic_2006, elia_insights_2017}. The stability of the ILs within the electrochemical window prevented side reactions and enabled deposition of Al, which increased the cell voltages \cite{li_aluminum_2002}.\\ ILs consist of weakly coordinated complex ions, which are liquid below 100 $^{\circ}$C, or at room temperature \cite{hayes_structure_2015}. Another advantage of the ILs is their wide electrochemical window, ranging from 4.5 to 6.0 V. This makes them suitable for high-performance energy storage devices \cite{wang_binder-free_2015}. In addition, most ILs show a high thermal stability, non-flammability, non-volatility and a zero vapor pressure, in comparison to organic solvents \cite{dieter_ionic_1988}. Cations such as 1-ethyl-3-methylimidazolium (EMIm), 1-butylpyridinium (1-BuPy) or 1-butyl-3-methylimidazolium (BMIm, displayed in Figure \ref{Figures/chap1fig:cations}), in combination with chloroaluminate (Al$_x$Cl$_y$) anions can achieve a potential window of up to 6 V. Furthermore, they reach an electric conductivity of 10 mS cm$^{-1}$, which is similar to the magnitude of most aqueous electrolytes \cite{ngo_thermal_2000}. ILs based on \ce{AlCl3} are mostly hygroscopic \cite{ueda_electroplating_2012}. A tiny amount of moisture might decompose them or decrease its voltage window. The Lewis acidity of chloroaluminate-based ILs depends on the molar composition of the anion and cation components, which also affects their conductivity and viscosity \cite{buzzeo_non-haloaluminate_2004}. Aluminium deposition (during charge) and dissolution (during discharge) at the anode takes place in a Lewis acidic IL, containing \ce{Al2Cl7}$^-$ displayed in Equation \ref{eq3} \cite{galinski_ionic_2006}. The equations are reversed during discharge.\\

\begin{figure}[tbh!]
\centering
\includegraphics[width=\textwidth]{Figures/chap1fig/cations}
\caption{Schematic illustration of the molecular structure of cations from the room temperature ionic liquids (ILs) used for rechargeable batteries. a) 1-ethyl-3-methylimidazolium cation (\ce{EMIm+}), b) 1-butyl-3- methylimidazolium cation (\ce{BMIm+}) and c) 1-butylpyridinium cation (\ce{BuPy+}).}
\label{Figures/chap1fig:cations}
\end{figure}
\textit{During charge:}
\begin{equation} \label{eq3}
4\ce{Al2Cl7^{-}} + 3\ce{e- -> Al + 7AlCl4^{-}} 
\end{equation}
\ce{Al2Cl7}$^-$ anions are formed when the molar ratio of \ce{AlCl3} is higher than 0.5 and the IL becomes a Lewis acid. An excess of the cation results in an IL which is a Lewis base due to the presence of free halide ions shown in Equation \ref{eq4} \cite{holbrey_ionic_1999}. A neutral composition consists of an equimolar molar ratio of \ce{AlCl3} and EMImCl. \\
\begin{equation} \label{eq4}
2\ce{AlCl4^{-} -> Al2Cl7^{-} + Cl-}  
\end{equation}
Using an \ce{AlCl3}/EMImCl electrolyte in AIBs was a major breakthrough since most of the earlier inventions displayed low capacity and were not good enough for long term practical use. EMImCl/ \ce{AlCl3} has since been the most commonly used electrolyte for non-aqueous aluminum-ion batteries.
% This was the reason why aluminium batteries were never good enough for long-term practical use. Holleck and Giner studied the electrochemistry of aluminum metal in the \ce{AlCl3}-KCl-NaCl eutectic melt as a possible electrolyte for aluminum batteries with high energy densities. They also found that the aluminum anode was passivated in these melts by formation of a solid salt layer due to local concentration changes at the metal surface during current flow. In addition, the cathodic deposition of aluminum resulted in formation of dendrites and production of chlorine gas\cite{holleck}. The oxide layer of an aluminum anode caused a number of problems preventing these cells from commercialisation. 
Figure \ref{Figures/chap1fig:AIBmech} is a schematic of an AIB using an IL  (made EMImCl/\ce{AlCl3}) electrolyte and 99\% pure Al foil as anode and a graphite cathode. 

\begin{figure}[tbh!]
\centering
\includegraphics[width=\textwidth]{Figures/chap1fig/AIBmech}
\caption{A schematic representation of a non-aqueous aluminium-ion battery (AIB).}
\label{Figures/chap1fig:AIBmech}
\end{figure}

Based on the type of electrolyte used, AIBs can be categorised as aqueous AIBs (AAIBs) and non-aqueous AIBS.  

\subsection{Aqueous aluminium-ion batteries (AAIBs)}
Using water as an electrolyte would reduce the battery costs significantly and increase battery safety. However, these batteries have their own set of problems. 

\begin{itemize}
    \item the electrochemical plating/stripping of aluminium occurs during charge and discharge cycles at a voltage far from the stable voltage window of water, which means unnecessary side reactions such as water splitting are bound to occur reducing efficiency as well as producing flammable \ce{H2} and \ce{O2} gases
    \item aluminium trichloride \ce{AlCl3} when used without ILs is highly acidic and might result in dissolution of the cathode material and corrosion of other battery parts
    \item a cathode material with a good cycling stability is yet to be discovered for aqueous AIBs
\end{itemize}  

Improving characteristics such as cycle life, efficiency and rate capability would enable these batteries to be used for high power applications. However, limited focus has been given to cycle life, efficiency and rate capability in the recent aqueous Al-ion literature. Anatase titanium oxide (\ce{TiO2}) has been investigated as an intercalation electrode for AAIBs and \ce{TiO2} nanostructures have been commonly used. Cycling data for \ce{TiO2} nanotubes were presented up to 13 cycles by Liu et al \cite{liu_aluminum_2012}.  %Recently, a graphene enhanced \ce{TiO2} electrode was shown to produce a capacity between 10 and 25 mAh g$^{-1}$ at a current density of 6.25 A g$^{-1}$. However, only 125 cycles were possible and coulombic efficiency was observed to be very low at approximately 50\%. Similarly, the charge discharge profiles observed for the nanospheres and nanotubes show low coulombic efficiencies of around 80-85\%. The reason for low efficiency of \ce{TiO2} in aqueous AIB electrolyte is yet to be explored, but the possibilities could be oxidation of \ce{Ti3+} due to dissolved \ce{O2} in electrolyte, \ce{H2} gas evolution or an irreversible reduction of \ce{Ti4+} to \ce{Ti2+} while charge.  
Aluminium in aqueous systems instantaneously forms a very thin (thickness $\sim$2-10 nm) surface layer of \ce{Al2O3} \cite{vargel_corrosion_2008}. This layer is quite stable over a pH range of about 4.0-8.6. Electrode passivation might prove advantageous in LIBs, but it creates a barrier for dissolution of Al anode and transportation of \ce{Al^{3+}} ions \cite{myung_electrochemical_2011}. For this reason, aqueous AIBs fail to achieve high energy densities and long cycle lives for long-term practical uses.

\subsection{Non-aqueous AIBs}
As discussed in Section 1.2, chloroaluminates (Al$_x$Cl$_y$) have been known for aluminium electrodeposition since the 1970s and are now being used as electrolytes for non-aqueous rechargeable aluminium batteries \cite{weppner_ionic_1976, fung_reaction_1972}. Systems with a general formula MCl-\ce{AlCl3} where \ce{M+} can be a cation like \ce{Na+}, \ce{Li+} or an organic species like pyrrolidinium or imidazolium have been tested as electrolytes for rechargeable AIBs \cite{das_aluminium-ion_2017}. These melts can be acidic, basic or neutral. In the acidic systems, the dominant species is \ce{Al2Cl7^{-}}, while in a basic system \ce{Cl-} and \ce{AlCl4^{-}} anions coexist whereas in neutral melts, only \ce{AlCl4-} anions exist \cite{galinski_ionic_2006, holbrey_ionic_1999}. To be compatible with Al anode, the IL \ce{AlCl3}/EMImCl emerges as the typical electrolyte. It provides a mild corrosive effect on the anode surface to activate the Al stripping and plating reaction. The equation below describes the reaction that takes place between the anode and the IL. During the discharge, Al is oxidized to form \ce{Al^3+} ions. These ions move to the cathode together with the coordination anions from the electrolyte (\ce{AlCl4-} and \ce{Al2Cl7-}) and intercalate into the material. When the battery is recharged, the redox reactions are reversed. The electrochemical reaction is given below.
\begin{equation}
        \ce{Al + 7AlCl4^{-} <=> 4Al2Cl7^{-} + 3e-}
\end{equation}
In 2017, Kravchyk and his group emphasized that an \ce{AlCl3}-graphite battery does not have a \enquote{rocking-chair} mechanism. The group claimed that these systems should not be called \enquote{Al-ion batteries} because the species that reversibly intercalate into the graphitic cathode i.e. \ce{AlCl4-}, is different from the ions involved in the deposition and stripping of Al atoms and/or \ce{Al^3+} ions from the anode. They showed that \ce{AlCl3} acts as the anode during the charging process, and gets reduced to Al atoms for electrodeposition, and simultaneously generates \ce{AlCl4^-} ions for intercalation. They proposed that the IL, in fact, acts as the capacity-limiting liquid anode and not just as an electrolyte. AIBs in the current literature are not utilising the anode completely. The author of this thesis agrees with Kravchyk and suggests that new electrolytes need to be prepared for non-aqueous AIBs that consume all of the \ce{AlCl4-} ions of the electrolyte. \\
The ILs not only overcome the above-mentioned problems with AAIBs, \ce{AlCl3}/ imidazolium is the quintessential electrolyte for non-aqueous AIBs making them much safer than LIBs \cite{jayaprakash_rechargeable_2011, lin_ultrafast_2015, wang_new_2013-1, rani_fluorinated_2013}. 

\section{Cathodes for rechargeable batteries}
A usable cathode material should have certain properties: good conductivity (both ionic and electrical), stability in the presence of the electrolyte, good working voltage and a large reversible storage capacity and high natural abundance. In addition, it should be easy to fabricate. The theoretical capacity of a cathode material (C$_{t}$) can be found using equation:

\begin{equation} \label{eq5}
   C_{t} \text{ = } \frac{n \times F}{3.6 \times M}
\end{equation}\\
where n = number of reactive electrons per formula unit,\\
M = molar mass of the cathode material and\\
F = Faraday's constant\\
Equation \ref{eq5} implies that a material with a low molecular weight and high number of reactive electrons would deliver a high theoretical capacity.

\begin{figure}[h!]
\centering
\includegraphics[width=\textwidth]{Figures/chap1fig/pt}
\caption{The periodic table suggesting elements that can be used as battery materials. However, a few elements from this table have shown good electrochemical performance such as molybdenum (Mo), tin (Sn), niobium (Nb) and tungsten (W). Potassium and calcium-ion batteries have also been studied. A few elements have been marked unsuitable due to various reasons such as low capacity, high cost, toxicity or radioactivity \cite{liu_understanding_2016}.}
\label{Figures/chap1fig:pt}
\end{figure}

A perfect battery that works for every application does not exist. It is important to identify the correct battery metrics while selecting a battery for a particular application. For example, a lead-acid battery works great in an automotive starter battery where it provides the required high rate capability. However, with its toxicity and low energy density it would not be suitable for portable electronics. Similarly, not all elements in the periodic table that may provide high energy density, be commercialised for everyday use. Figure \ref{Figures/chap1fig:pt} displays the elements that can be used for designing new cathodes. Radioactive elements, heavy metals, and inert gases are some of the elements from the periodic table that should never find any long-term practical use in battery applications. Transition metals have variable valence states, which increases the number of electrons that can be stored (higher n), increasing the battery capacity (higher C$_t$), refer Equation \ref{eq5}. Some transition metals such as vanadium (V) and cobalt (Co), are being used as cathode materials in LIBs despite their toxic nature \cite{cui_carbon/titanium_2010, qu_vanadium_2019, salunkhe_direct_2014, spreafico_pvdf_2014}. Carbon is one of the cheapest materials that has the ability to store large amount of energy \cite{candelaria_nanostructured_2012}. This is one of the reasons why carbon-based materials are the premium choice in any energy storage device.
%The type of bond formed between a metal ion and the ligand plays an important role in determining the electrochemical potential of the material. When the electronegativity difference between the two is high, an ionic bond is formed. A smaller difference results in a covalent bond. Materials with a covalent bonds form poorly packed structures, while ionic bonds form dense structures. A stable structure would enhances the phase stability and electrochemical potential of the material \cite{melot_design_2013}. Furthermore, electrode potential depends on the ionic radius of the material. When an atomic nuclei is loosely bound to its valence electrons, the system needs lower energy for electron transfer , which leads to a lower potential. If more energy is consumed during electron transfer  for materials having a lower ionic radii, the electrochemical potential of the material increases. 
%The cell potential \textbf{E} associated with an electrochemical reaction is defined as the decrease in the Gibbs free energy, \textbf{G}, per coulomb of charge transferred and leads to the fundamental thermodynamic relationship given in Equation \ref{eq6}:
%\begin{equation} \label{eq6}
 %   \Delta G \text{ = } -nFE
%\end{equation}\\
%where $\Delta$G = change in Gibbs free energy,\\
%n = number of electrons taking part in the redox process,\\
%F = Faraday constant and\\
%E = electrochemical potential of the material. \\
%Liu \textit{et al.} showed that the electrochemical potential of a material is affected by interaction between the atoms or electrons and those interactions also change the internal energy \cite{liu_understanding_2016}. Using polyanionic ligands such as phosphates, sulfates and silicates, which are more electronegative, results in formation of bonds with increased ionic character \cite{liu_understanding_2016}. This further enhances the electrochemical potential of a material \cite{melot_design_2013}. For example, \ce{LiCoPO4} (polyanionic) displays a higher potential at 4.8 V than \ce{LiCoO2} (oxide), which has a potential at 4.0 V\cite{masquelier_polyanionic_2013}.

\subsection{Cathodes for non-aqueous AIBs}
Based on the material type and structure, cathodes tested during the PhD were mainly categorised into two families.\\
\textbf{Carbon-based materials}: Graphite, a form of carbon with layered hexagonal structure, has been commonly used as an electrode in various battery systems \cite{xu_charge-transfer_2007, zhang_novel_2016, wu_carbon_2003, jian_carbon_2015}. Its layered structure allows insertion of ions and it also has good thermal and electrical conductivity, and a high electrical potential \textit{vs.} \ce{Al}/\ce{Al^{3+}} of 2.1 V \cite{lin_ultrafast_2015}. In graphite-based batteries, Al$_x$Cl$_y$ anions intercalate into the graphitic layers when the cells are being charged and deintercalate during discharge. Electroplating and dissolution of Al takes place at the anode. Figure \ref{Figures/chap1fig:AIBmech} displays the intercalation mechanism, which is very similar to that of LIBs. Different forms of graphite have been used in AIBs. Yu \textit{et al.} made an AIB using graphene nanoribbons on highly porous three-dimensional (3D) graphene foam as cathode. The cell exhibited low charging voltage plateaus, high discharge voltage plateaus near 2 V, high capacity about 123 mAh g$^{-1}$ at a current density of 5000 mA g$^{-1}$ with CE higher than 98\% for 10000 cycles \cite{yu_graphene_2017}. Fluorinated graphite, kish graphite flakes, 3D graphitic-foam, graphene aerogels and several other forms have been tested, which showed discharge capacities ranging from 60-250 mAh g$^{-1}$ \cite{rani_fluorinated_2013, wang_kish_2017, wu_3d_2016, huang_graphene_2019} . \\
Beyond graphite, various other \textbf{carbon-based materials} have been used. Several reports show that the amorphous activated carbon with a large surface area exhibit an excellent rate performance and a relatively high capacity, since high porosity provides plenty of active sites \cite{tang_hollow_2012, li_commercial_2018}. They are a popular choice in sodium-ion batteries\cite{zhu_high_2016,hou_carbon_2015,cao_sodium_2012}. Elazari \textit{et al.} impregnated microporous activated carbon with elemental metallic sulfur and used it as a cathode a in Li-S battery. The cell exhibited very good electrochemical performance and the porous structure enabled electrolyte penetration \cite{elazari_sulfur-impregnated_2011}.  

%Eliad and Zhu \textit{et al.} showed that activated carbon (AC), owing to its highly porous structure, provides a high surface area and renders an additional capacitor-like charge storage when used in a battery. They demonstrated the adsorption/ desorption of charge-carrying species takes place on the cathode's surface during charge/ discharge\cite{eliad_ion_2001, zhu_carbon-based_2011}.\\

The second type of cathode materials are \textbf{two-dimensional (2D) materials} such as transition metal dichalcogenides (TMDs), transition metal oxides (TMOs), MXenes, and a few other metallic oxides. These materials offer tunable chemical and physical properties due to their various elemental compositions and different crystallographic structures. In addition, they possess excellent electrochemical properties \cite{chia_electrochemistry_2015}. A 2D plane imparts a high surface area, which allows complete utilization of all available sites in a cathode material similar to graphite \cite{jia_interfacial_2016, naguib_mxene_2012}.\\ 
Each metal in a TMD is coordinated with two chalcogens; the metal is in +4 oxidation state while the chalcogen atom is in -2 state. Interaction or insertion of ions from the electrolyte alters the oxidation states of the atoms as a result of the redox process. In addition, a few TMDs, especially molybdenum (Mo) and tungsten (W) dichalcogenides, are capable of phase transition after intercalation (discussed in Chapter \ref{chap4}). Due to introduction of extra electrons and rearrangement of d orbitals, a phase transition from the 2H-phase to the 1T-phase takes place \cite{acerce_metallic_2015-1}. This changes the electronic properties of the material since 2H-\ce{MoS2} is semiconducting, while 1T-\ce{MoS2} is metallic. For the above-mentioned reasons, 2D materials have gained enormous attention in recent years. Prussian blue, an inorganic dye was also tested as a cathode due to its three-dimensional framework and that it should allow insertion of the chloroaluminates. 
%These materials have promoted a relatively reversible trivalent reaction. Strong electrostatic nature of trivalent \ce{Al^{3+}} sometimes leads to sluggish kinetics, high over-potentials, and degradation of the host structure. Therefore, to accommodate the highly charged ions, it is essential for the cathode materials to possess weak bond strengths between the host frameworks. Our unpublished, preliminary density functional theory (DFT) calculations indicated a significant decrease in inter-layer spacing of these materials when \ce{Al^{3+}} cations were assumed to intercalate (owing to the very high charge density of \ce{Al^{3+}}). Therefore, we propose intercalation of \ce{AlCl4-} anions into the layered cathode. XRD, Raman spectroscopy and XPS were used to verify our hypothesis.
During this study, various carbon-based and 2D materials were tested as potential cathodes for rechargeable non-aqueous AIBs. The results have been reported in the subsequent chapters.

\section{Research objectives}
The goal of this PhD project is to find new cathodes for rechargeable AIBs that perform better than state-of-the-art. 
\begin{itemize}

    \item to prepare, test and investigate layered-type structures such as molybdenum dichalcogenides, or boron nitride as cathodes and explore their mechanism
    \item to prepare carbonised natural products and other forms of carbon with high surface area (other than graphite) and test them as cathodes 
    \item to establish the mechanism behind all successful cathodes (two-dimensional, carbon-based, etc.) using analytical tools such as X-ray diffraction, Raman spectroscopy, and X-ray photoelectron spectroscopy
    \item to explore the impact of cost-effective solvents and current collectors used in an aluminium-ion battery. 
\end{itemize}
\newpage
\section*{\centering Preface}
The next chapter explains the experimental methods carried out to assemble a battery on a laboratory-scale. Procedures for preparing cathode slurries and electrolytes for an aluminum-ion cell have been briefly described.  





 
% Chapter 3

\chapter{Experimental methods} % Main chapter title
In this chapter we discuss the methods used when assembling a lab-scale battery for preliminary electrochemical tests. 
\label{chap3} % For referencing the chapter elsewhere, use \ref{Chapter1} 
%----------------------------------------------------------------------------------------

% Define some commands to keep the formatting separated from the content 
%----------------------------------------------------------------------------------------
\section{Components of a cathode}
A cathode consists of an active material, a binder and conductive carbon. They are mixed together to form a slurry, which is then coated on a current collector.  
\begin{itemize}
    \item Active material: The main material that gives a battery its capacity
    \item Binder: A binder is usually a polymer that uniformly binds the cathode materials and allows it to stick firmly on the current collector. Polytetrafluoroethylene (PTFE or Teflon) and Polyvinylidene fluoride (PVDF) are most commonly used as binders while making battery slurries.
    \item Conductive carbon: A conductive form of carbon which is amorphous in nature, is added to improve the conductivity of the slurry. Super-P\textsuperscript{\textregistered} or carbon black is normally used as a conductive agent. 
    \item Current collector: Typically a metallic foil (copper, aluminium, steel, etc.) connected to the electrode with external loading. The main role of the current collector is to support the electrode and to collect the accumulated electrical energy from the electrode\cite{sun_effect_2017}.
\end{itemize}
\section{Formation of slurry}
A slurry is a paste-like substance used in making battery electrodes. It is important for slurries to be homogeneous. This helps in getting a uniform coating at later stages.  PVDF was mixed separately in an organic solvent, N-methyl pyrrolidinone (NMP). This viscous solution was added to the mixture of active material and Super-P\textsuperscript{\textregistered} . NMP can be added later to adjust the consistency of the slurry. It was then mixed on a magnetic stirrer for 8-10 hours. 
\section{Preparing a cathode}
Once the slurry was prepared, it was 'doctor-bladed' on a current collector. Doctor blading is a coating process where the slurry spreads on a substrate (molybdenum foil)  using a blade, to form a this sheet which dries off to give a layer of coating.  Initially, we used nickel foils as current collectors, however we found that Ni oxidised at $\sim$ 1.0 V. This did not allow the cell to reach it's cut-off potential at 2.4 V. It impeded the cell's redox processes, which reduced the cell's capacity. After trying a few current collectors, we found that molybdenum foil was inert in this cell chemistry (discussed in Appendix A) and was used in all our battery systems. After coating the foils with the slurry, the electrodes were dried at 80$^{\circ}$ C for two hours to improve the adhesion of the slurry onto molybdenum foil. 
\begin{figure}[tbh!]
\centering
\includegraphics[width=\textwidth]{Figures/chap3fig/coating}
\caption{a) Doctor-blading on a current foil using a steel blade. A dried cathode after b) uniform and c) non-uniform coating.}
\label{Figures/chap3fig:coating}
\end{figure}
\section{Vacuum drying}
Vacuum drying is a moisture-removal technique by means of creating a vacuum. Vacuum drying is used for drying things which are hygroscopic (water sensitive). IA vacuum is created to decrease the chamber pressure below vapor pressure of the solvent (NMP), causing it to boil. This increases its rate of evaporation and increases the drying rate of the product. The pressure maintained in vacuum drying is generally 0.03–0.06 atm. The cathodes are ready for use after drying. 
\section{Assembling a cell}
Using pouch cells or coin cells (CR-2032\textregistered) is a common practice amongst researchers working on batteries. CR-2032 is made of steel and could not be used in our experiments. Since our electrolyte reacts with steel, we switched to custom-made cells that used PEEK (polyether ether ketone) for the main body and steel rods as plungers that would push the electrodes towards each other. PEEK  is a colourless organic thermoplastic polymer. It has a melting point of 343$^{\circ}$C, with excellent mechanical and chemical resistance properties. Steel rods were later replaced with molybdenum rods because of above mentioned reasons. 
\begin{figure}[tbh!]
\centering
\includegraphics[width=\textwidth]{Figures/chap3fig/swagelok}
\caption{a) Assembling a two-electrode PEEK cell using a cathode, separators wetted with electrolyte and an anode. Mo rods were used as plungers and Mo sheets was used as current collector. b) A custom-made lab cell ready for preliminary electrochemical measurements.}
\label{Figures/chap3fig:swagelok}
\end{figure}
A cathode was placed at the bottom of the cell, two separators made of glass microfibers were put above the cathode. Electrolyte was then added ($\sim$80 $\mu$l) until the separators were completely wet. An anode which was 99\% pure aluminium foil was cut into a disc and placed on top and the cell was then screwed tight, shown in Figure \ref{Figures/chap3fig:swagelok}a. Since the electrolyte is hygroscopic, the cell was assembled inside a glove box with <0.1ppm \ce{O2},\ce{H2O}. The cell was taken out of the glove box and wrapped tightly with a paraffin film to further inhibit contact with moisture or air, as shown in Figure \ref{Figures/chap3fig:swagelok}b. The cells were ready for electrochemical measurements . 





















 
% Chapter 1
\section*{Preface}
This chapter discusses the characterisation techniques that were implemented post-mortem, to fully analyse how a battery works. Electrochemical processes such as cyclic voltammetry and galvanostatic charge/ discharge curves have been discussed in detail.   

\pagebreak

\chapter{Techniques for characterisation} % Main chapter title

\label{chap2} % For referencing the chapter elsewhere, use \ref{Chapter1} 

Performance of a battery cannot be assessed without galvanostatic charge and discharge cycles and cyclic voltammetry (CV). Battery potential is determined by the stability of its electrolyte. CV scans help in determining the voltage range of a cell. A charge/discharge cycle (CDC) on the other hand helps in evaluating a cell's specific capacity and nominal (discharge) voltage. Long term CDCs are necessary to verify a cell's stability and observe whether it can be put to commercial use or not. 
During continuous charge and discharge, a cathode material undergoes many changes. For example, distance between two layers should increase when intercalation takes place. Or, the oxidation states of a metal should change when it oxidises or reduces itself during cycles. Analytical tools are needed to study the following changes. 

\begin{itemize}
    \item Changes in the crystal lattice of a material can be studied in its XRD patterns 
    \item Changes in the oxidation state of an element can be noticed in its XPS spectra
    \item Changes in the vibrational mode of a molecule can be observed in its Raman spectra 
\end{itemize}

\section{Galvanostatic charge/discharge cycles}
Galvanostatic charge and discharge is a method to evaluate the amount of charge stored in a cell, typically under constant current. The technique measures voltage at a controlled or fixed current rate. Since the current is repeatedly reversed, it is also known as 'cyclic chronopotentiometry'. It is used to estimate the specific capacity and cycling stability of a cell. The total quantity of electricity per mass available from a fully charged cell can be calculated, from the charge transferred during discharge in terms of mAh g$^{-1}$. Specific discharge capacity is frequently measured at different discharging rates to establish rate capability of a cell \cite{pyun_electrochemistry_2012-1}. The voltage profile obtained can be used to identify multi-step redox reactions, in Figure \ref{Figures/chap2fig:ChrononCDC}. 

\begin{figure}[th!]
\centering
\includegraphics[width=\textwidth]{Figures/chap2fig/ChrononCDC}
\caption{a) Chronopotentiogram- a graph of electric potential versus time, at constant current. b) A galvanostatic charge/discharge curve showing the voltage plateaus at which reactions occur and the charging/discharging capacity.}
\label{Figures/chap2fig:ChrononCDC}
\end{figure}

\section{Cyclic voltammetry}
Cyclic voltammetry (CV) is a technique which measures the current that develops in an electrochemical cell during oxidation and reduction of an analyte (say M). It is performed by cycling the potential of a working electrode, and measuring the resulting current. In Figure \ref{Figures/chap2fig:CV}, we started a forward sweep with a positive scan (lower potential to higher potential). S1 is called a switch potential where the voltage is sufficient enough to cause an oxidation or reduction, and the scan is reversed. Potential is then swept negatively (higher potential to lower potential) until it reaches S2 (another switch potential). In an ideal situation, during forward sweep, M is depleted from the solution as it gets oxidised to \ce{M+}. Further oxidation after scanning higher potentials, leads to growth of a diffusion layer (solution containing M/\ce{M+} ions) at the electrode surface throughout the scan. The layer continues to expand until a certain point, recording maximum current density. However, since diffusion layer continues to grow at this stage, flux of M from the bulk solution to electrode surface decreases. Therefore, current starts to decrease and we get an oxidation peak. A reverse scan converts \ce{M+} back to M (reduction) via similar pathway- formation of a diffusion layer containing M and eventually we record a reduction peak. The two peaks are separated due to the diffusion of the analyte to and from the electrode. If the reduction process is chemically and electrochemically reversible, a peak-to-peak separation of 57 mV is observed \cite{bard_electrochemical_1980}. When there is a high barrier to electrochemical irreversibility, electron transfer reactions are sluggish and more positive/negative potentials are required to observe oxidation/reduction reactions respectively. 

\begin{figure}[tbh!]
\centering
\includegraphics[width=\textwidth]{Figures/chap2fig/CV}
\caption{Cyclic voltammogram of an AIB at a scan rate of 10 mV s$^{-1}$ using a two-electrode cell with aluminium foil acting as a counter and reference electrode.}
\label{Figures/chap2fig:CV}
\end{figure}

Scan rates play a very important role too. If a CV is run on a slower scan rate (0.05 mV s$^{-1}$), diffusion layer grows farther from the electrode, which reduces the flux, consequently decreasing the current value. At a faster scan rate lead (1 V s$^{-1}$), the size of the diffusion layer decreases and higher currents are recorded. Cyclic voltammetry is a helpful tool in understanding the presence of a surface reaction and its reversibility during cell cycles. It can be used for both single-electron and multi-electron processes.  

\section*{Sample preparation}
The cell was disassembled inside a glove box to prevent the cathode from any contact with air or moisture. The cathode was taken out and washed with dry ethanol to get rid of any remaining electrolyte on its surface. It was found that scrapping off the active material from the current collector did not yield enough sample for analysis, since some of it got wasted in the process. Thus, cathode, including the current collector, was used for X-ray diffraction, Raman and X-ray photoelectron spectroscopic analyses.

\section{X-ray diffraction Studies}
Diffraction of x-rays by crystal planes allows us to derive lattice spacings by using Bragg's law. 

 \begin{equation} \label{eq1}
     2d\sin\theta \text= n{\lambda}
 \end{equation}
 where d = spacing between diffracting planes,\\
$\theta$ = incident angle,\\ 
n = any integer, and \\
$\lambda$ = wavelength of the incident beam. X-rays produce the diffraction pattern because their wavelength $\lambda$ is typically the same order of magnitude (1-100 $\AA$ ) as the d-spacing between the crystal planes. According to Eq.\ref{eq1} any decrease in 2$\theta$ suggests an increase in the d-spacing. 
A pure crystalline sample such as \ce{MoS2} (Figure \ref{Figures/chap2fig:XRD}a) yields sharp peaks in a XRD pattern since it has a long ordered structure. Random orientation of the powdered material is attained after scanning the sample through a range of 2$\theta$ angles. Conversion of the diffraction peaks to d-spacings allows identification of the sample because each sample has a unique set of d-spacings. Typically, d-spacings of the sample are compared with standard reference patterns (International Centre for Diffraction Data, ICDD). For determination of unit cell parameters, each reflection implies a specific lattice plane indicated by miller indices \textit{hkl} (labelled in red  for \ce{MoS2} crystal lattice). Figure \ref{Figures/chap2fig:XRD}b displays the diffraction pattern of activated carbon obtained from human hair. Since the structure of activated carbon is much less ordered, its pattern shows line broadening of the major diffraction bands, which exist at $\sim$ 25$^{\circ}$  and $\sim$ 44$^{\circ}$.
X-ray diffraction studies is a useful tool and can easily help in proving intercalation of ions. Rani \textit{et al.} used fluorinated graphite as a cathode for AIBs. The d-spacing values of the discharged graphite cathode were higher than natural graphite. The results indicate that intercalation of aluminum ions in the graphene sheets increase the d-spacing of the graphite crystal\cite{rani_fluorinated_2013-1}.
%X-ray diffraction shows line broadening of only the principal graphite diffraction bands. This broadening is usually interpreted in terms of dimensions of a hypothetical crystallite. Although the crystallite concept has been used when comparing structures in carbons, it has to be stressed that the crystallite does not exist as such within these structures. The disorganized carbon are present in cross-linkage structures [15] forming non-crystallite structures to form microstructures! 
Panalytical X-Ray diffractometer was used to record the XRD patterns using Cu-K$\alpha$ radiation at an operating voltage of 45 kV and a 40 mA current. The patterns were run with copper radiation ($\lambda$ =1.5405\AA) at a scanning speed of 2$^{\circ}$ in 20 minutes. 

\begin{figure}[tbh!]
\centering
\includegraphics[width=\textwidth]{Figures/chap2fig/XRD}
\caption{X-ray diffraction pattern of a) bulk molybdenum disulfide (ICDD: 04-001-9285), and b) activated carbon.}
\label{Figures/chap2fig:XRD}
\end{figure}

\section{Raman spectroscopy}
Raman spectroscopy is a technique, which is used to determine vibrational modes of a molecule. A source of monochromatic light, usually from a laser, interacts with molecular vibrations in the system, resulting in the energy of the laser photons being shifted up (blue shift) or down (red shift). The shift in energy gives information about any changes taking place in the vibrational modes of a material. 
%This technique uses the inelastic scattering of photons, also known as Raman scattering. 
\begin{figure}[tbh!]
\centering
\includegraphics[width=\textwidth]{Figures/chap2fig/Raman}
\caption{Raman spectra of graphite.}
\label{Figures/chap2fig:Raman}
\end{figure}

Graphite is composed of \ce{sp2} bonded carbon atoms in planar sheets. The bond energy of the \ce{sp2} bonds displays its vibrational frequency at 1582 cm$^{-1}$. The presence of additional bands in graphite spectrum indicate that there are some carbon bonds at different bond energies. The D band indicates presence of some disorder in the structure. The ratio of intensity of D/G peaks is a measure of the defects present. Both D and G peaks are the result of vibrations of \ce{sp2} bonded carbon atoms. G band is an outcome of in-plane vibrations, whereas the D peak is due to out of plane vibrations attributed to the presence of structural defects. If the D band is more intense, it means that the \ce{sp2} bonds are broken which in turn means that there are more \ce{sp3} bonds, there will be a maximum D/G ratio. If I$_D$/I$_G$ ratio is higher than pristine sample, it means that defects are present on the material. 
Raman spectroscopy is a helpful technique in detecting the mechanism of an intercalation-based battery. Wang \textit{et al.} used Raman spectroscopy to show two different intercalation processes involving chloroaluminate anions in a graphite cathode. The Raman data pointed to two different intercalation processes at two different charging plateaus. The first plateau in the charging curve showed G band splitting and the higher voltage plateau showed a single, dominant blue-shifted peak. During discharge, the opposite trends were observed when chloroaluminate anions were deintercalated. The original graphite spectrum was recovered when the cell was fully discharged\cite{wang_advanced_2017}. 

\section{X-ray photoelectron spectroscopy}
 X-ray photoelectron spectroscopy is used to measure elemental composition and oxidation states of various elements. It is a surface-based technique that quantitatively analyses a sample. By irradiating a sample with a beam of X-rays, kinetic energy and number of electrons escaping from the top 10 nm of the sample are measured. 
%The instrument requires high vacuum (10$^{-8}$ millibar) conditions to count these electrons. The electron emission after irradiation is also called a 'photoelectron effect'. These electrons are separated according to their energies and counted. ,
A normal XPS spectrum is a plot of the number of electrons detected versus the binding energy of the electrons detected. XPS helps in studying the redox processes that take place inside a cell. XPS helps in characterising elements species and valence changes of original sample and sample during electrochemical reactions. Understanding curve-fitting of an XPS spectra is important as it suggests the number of chemical states and therefore number of peaks present in a sample. It is essential to apply constraints to restrict the peak widths and relative intensities of the peaks. The peaks were fitted using Shirley background and any observed peak was convoluted using Gaussian and Lorentzian function. \\
Li \textit{et al.} studied the XPS spectra of \ce{MoS2} microspheres to probe the valence changes and the Al$^{3+}$ storage mechanism during the charging/ discharging process at various charge and discharge states of the cathode \cite{li_rechargeable_2018-2}.

\begin{figure}[tbh!]
\centering
\includegraphics[width=\textwidth]{Figures/chap2fig/XPS}
\caption{X-ray photoelectron spectra of molybdenum disulfide. Molybdenum 3d orbitals appear as a doublet at 229 and 232 eV. The area ratio for the two peaks (3d$_{3/2}$ : 3d$_{5/2}$) was 2:3 (corresponding to two electrons in the 3d$_{3/2}$ level and 4 electrons in the 3d$_{5/2}$ level).}
\label{Figures/chap2fig:XPS}
\end{figure}





%\section*{Preface}
In this chapter, we discuss performance of AIBs using two-dimensional molybdenum dichalcogenides such as \ce{MoS2}, \ce{MoSe2} and MoSSe in their bulk and nano form, as cathodes for AIBs attempt to establish their mechanism. We explored different molybdenum dichalcogenide-based cathodes and their mechanism of energy storage. We expected that two-dimensional (2D) layered materials that support intercalation of charged species might be suitable as active cathode materials in AIBs. 
\pagebreak
\chapter{Molybdenum dichalcogenides as cathodes for rechargeable AIBs} % Main chapter title

\label{chap4} % For referencing the chapter elsewhere, use \ref{Chapter1} 

\section{Theory and background}
Transition metal dichalcogenides (TMDs) have a strong in-plane covalent bonding and weaker van der Waals (vdW) bonds exist between any two layers, which is quite similar to graphene layers in graphite. The 2D structure facilitates intercalation of ions within these layers. Molybdenum dichalcogenides not only provide redox variability but also a high theoretical capacity (600-1200 mAh g$^{-1}$). The intercalation voltage observed with \ce{MX2}, where M = Mo, W, Ti, etc. and X = S and Se, is high ($\approx$ 2.0 V). The mechanism that TMDs follow is based on intercalation as well as conversion. 
Molybdenum dichalcogenides (\ce{MoX2} where X=S, Se or Te) display similar properties as graphite. They have a 2D-layered structure, which allows intercalation of ions and are electrically conductive. Lower volumetric expansion on cycling is an advantage these materials have over graphitic cathodes \cite{liang_rechargeable_2011, huang_molybdenum_2019}. Amongst various transition metal chalcogenides, \ce{MoS2} has been extensively studied as a cathode for rechargeable batteries \cite{li_mos2_2004, zhu_fast_2015}, making them attractive candidates for AIB cathodes. In 2015, Geng \textit{et al.} found that \ce{Al^3+} ions fully intercalated into chevrel phase \ce{Mo6S8} with the cations occupying two different sites in the crystal lattice \cite{geng_reversible_2015}. This mechanism was called the 'rocking chair' mechanism where charge carrying species shuttled back and forth between intercalating electrodes during cycles while the overall electrolyte concentration remains constant. The discharging and charging reactions at the anode (equation 1) and cathode (equation 2) were proposed as follows:

\begin{align}
          \ce{Al + 7AlCl4^-} &\rightleftharpoons 4\ce{Al2Cl7^- + 3e^-}\\
\ce{8AlCl7^- + 6e^- + Mo6S8 &\rightleftharpoons Al2Mo6S8 + 14AlCl4^-}
\end{align}

Three years later, Li \textit{et al.} prepared \ce{MoS2} microspheres by a simple hydrothermal method \cite{li_rechargeable_2018-2}. They proposed a similar mechanism where \ce{Al^3+} ions inserted into the electrode accompanied by a phase transformation at the electrode interface. Li and his group confirmed this phase-transition by using \textit{ex-situ} XPS and XRD etching techniques. The reaction equations for this battery system at the cathode (equation 3) and anode (equation 4) were proposed as follows:
\begin{align}
    \ce{MoS2 + x\ce{Al^{3+}}  + 3x\ce{e-} &\rightleftharpoons Al_xMoS2}\\
    \ce{Al} + 7x\ce{AlCl4-} &\rightleftharpoons 4x\ce{Al2Cl7-} + 3x\ce{e-}
\end{align}

In general, these cells showed low energy density and had reversibility issues in the redox processes. It has been reported that transition metal dichalcogenide electrodes tend to transition from a 2H phase into a more conducting 1T phase when used in a battery \cite{fan_hybrid_2017}. A hybrid \ce{Mg^{2+}}/\ce{Li+} cell was tested using bulk \ce{MoS2} as a cathode material. During cyclic voltammetry (CV) scans, the authors associated the first cathodic peak, with a phase transition. 2H phase \ce{MoS2} was converted to 1T phase during initial ion intercalation. This seems to be a common phenomenon for molybdenum dichalcogenides, since Li \textit{et al.} observed similar transitions in sodium ion batteries \cite{li_enhancing_2015}. It mostly takes place during the first cycle and since the phase change is irreversible, it can be detected in a cyclic voltammogram.

This chapter documents a range of 2D molybdenum dichalcogenides including \ce{MoS2}, \ce{MoSe2} and \ce{MoSSe}, as cathodes for non-aqueous AIBs. Our unpublished, preliminary density functional theory (DFT) calculations indicated a significant decrease in inter-layer spacing of these materials when \ce{Al^3+} cations were assumed to intercalate (owing to the very high charge density of \ce{Al^3+}). Therefore, we propose intercalation of structurally distorted \ce{AlCl4-} anions into the cathode layers. Surprisingly we found that \ce{MoSe2}-based cathodes performed different and better than all of the other molybdenum dichalcogenides.

\section{Experimental methods}
Same as discussed in Chapter 3.

\section{Results and Discussion}
Figure \ref{Figures/chap4fig:S1} shows the crystal structure of \ce{MoX2} where X is sulfur (S) and/or selenium (Se). The material has two vacant sites for intercalation --- M1 and M2. M1 denotes the spaces in between the X-Mo-X atoms, whereas M2 represents the space created between the \ce{MoX2} layers as shown in Figure \ref{Figures/chap4fig:S1} a). The inter-layer distance in \ce{MoX2} is 6.3 \AA\ with a gallery height of 3 \AA. The layers are held together by weak van der Waals (vdW) forces. M2 presents an open network and provides various interstitial sites for intercalation. Since \ce{AlCl4-} ions are 5.28 \AA\ in diameter, as reported by Takahashi {\it et al.} \cite{takahashi_niv2o5nh2o_2005}, they undergo some distortion during intercalation to fit into these layers. Our preliminary results showed that \ce{Al^{3+}} would \lq contract\rq\ the \ce{MoX2} layers when trying to intercalate, making \ce{AlCl4-} anion intercalation more likely. Also, the triply charged \ce{Al^{3+}} cation has to overcome strong electrostatic forces from the \ce{S^{2-}} or \ce{Se^{2-}} anion network in order to enter, making the intercalation process slow and most likely not reversible. Therefore, we propose intercalation of \ce{AlCl4-} anions from the electrolyte into M2 sites of \ce{MoX2} during charge. Galvanostatic cycles, cyclic voltammetry (CV), X-Ray diffraction (XRD), Raman spectra and X-Ray Photoelectron Spectroscopy (XPS) results discussed later, strongly support our claim of a reversible intercalation process especially in \ce{MoSe2}.

\begin{figure}
  \centering
  \includegraphics[width=\textwidth]{Figures/chap4fig/S1}
  \caption{Schematic representation of a) a \ce{MoX2} crystal structure with possible intercalation sites at M1 and M2 b) intercalation at M1 site and c) intercalation at M2 site.}
  \label{Figures/chap4fig:S1}
\end{figure}

\begin{figure}[htb!]
\centering
\includegraphics[width=\textwidth]{Figures/chap4fig/MoX2CDCCV}
\caption{First charge/discharge curve at 40 mA g$^{-1}$ for a) \ce{MoS2}, b) \ce{MoSe2} and c) MoSSe. d) Specific capacities of \ce{MoS2}, \ce{MoSe2} and MoSSe at current rates of 25, 40 and 100 mA g$^{-1}$. e)First CV scan of \ce{MoS2}, f) \ce{MoSe2} and g) MoSSe at a scan rate of 10 mV s$^{-1}$ in an aluminium-ion battery against Al reference electrode.}
\label{Figures/chap4fig:MoX2CDCCV}
\end{figure} 

Figure \ref{Figures/chap4fig:MoX2CDCCV} a)--c) shows the charge/discharge cycles (CDCs) for \ce{MoS2}, \ce{MoSe2} and MoSSe at a current rate of 40 mA g$^{-1}$. The discharge capacity of Al/\ce{MoS2} in its first cycle was found at $\sim$45 mAh g$^{-1}$, Figure \ref{Figures/chap4fig:MoX2CDCCV} a). Comparing this with its first CV scan (Figure \ref{Figures/chap4fig:MoX2CDCCV} g), a good correlation between the discharge voltage plateau and reduction peaks, and other redox features was found. With discharge plateaus at 1.9 V and 1.7 V, the first CV scan for \ce{MoSe2} displayed two reduction peaks at 1.65 V (point A) and 1.0 V (point B), Figures \ref{Figures/chap4fig:MoX2CDCCV} b) and \ref{Figures/chap4fig:MoX2CDCCV} h). The peak at 1.0 V suggested an irreversible reaction since this peak was absent in the following scans. Based on this, we agree with Li \textit{et al.}'s interpretation and attributed this peak to an irreversible phase transition \cite{li_enhancing_2015}. During this transition, the semi-conducting 2H phase converted into a more metallic 1T phase. This transition seemed to increase the interlayer spacing of \ce{MoSe2} by reducing the vdW forces that exist between the two layers \cite{fan_hybrid_2017}. Al/MoSSe cells showed three distinct plateaus during charging at 1.2 V, 2.0 V and 2.3 V in its first cycle, with a discharge plateau at 0.5 V shown in Figure \ref{Figures/chap4fig:MoX2CDCCV} c). Capacities of all molybdenum dichalcogenides were recorded at different current rates of 25, 40 and 100 mA g$^{-1}$, and displayed in Figure \ref{Figures/chap4fig:MoX2CDCCV} d). Since \ce{MoSe2} displayed stable specific capacities at all current rates, we recorded further 200 cycles at the highest current rate of 100 mA g$^{-1}$. A highly reversible electrochemical reaction was observed since the capacity remained at 30-32 mAh g$^{-1}$ after 200 cycles (Figure \ref{Figures/chap4fig:MoX2CDCCV} e)). The presence of multiple charging plateaus in MoSSe might correspond to various oxidation processes occurring when \ce{AlCl4-} interacts individually with S and Se atoms. The first CV scan in Figure \ref{Figures/chap4fig:MoX2CDCCV} i) showed an irreversible reduction potential at 0.9 V, point B', like \ce{MoSe2}, implying a similar phase transition. It seems MoSSe undergoes a lattice distortion and the material loses its long range order after converting to its 1T phase. This might be the reason why the cells fail to deliver a stable capacity.

CV of a blank cell with an uncoated Mo foil (Figure inset\ \ref{Figures/chap4fig:fig2} a)) showed that the current collector did not contribute to the cell's capacity. Both \ce{MoS2} and \ce{MoSe2} have similar interlayer distance (6.3 \AA) and a gallery height of 3.0 \AA. However, \ce{MoSe2} showed a higher capacity and a more stable cycle life. To account for this behaviour, we compared the cyclic voltammograms of all electrodes at a scan rate of 10 mV s$^{-1}$ in Figure \ref{Figures/chap4fig:fig2}.  Different charge-storage mechanisms lead to distinct features in the CVs. Ideal capacitors result in a rectangular CV shape. Due to the absence of faradaic processes, the charging/discharging currents are directly proportional to the scan speed. Batteries show oxidation and reduction peaks in their voltammograms \cite{jiao_aluminum-ion_2016}. We observed that the CVs of \ce{MoSe2} and MoSSe in Figure \ref{Figures/chap4fig:fig2} b) and \ref{Figures/chap4fig:fig2} c) covered a broader area suggesting an additional capacitor-like charge storage mechanism. This additional non-faradaic process taking place at their surfaces might have resulted in a higher capacity for \ce{MoSe2} and  MoSSe. Also, the peak indicating phase transition from 2H$\rightarrow$1T at $\sim$0.9 V was visible only for \ce{MoSe2} and MoSSe. Charge storage in \ce{MoS2} is primarily based on reversible oxidation and reduction of Mo from \ce{Mo^{4+}} to \ce{Mo^{5+}} with oxidation peaks visible at 1.8 V (O1) and 2.1 V (O2), and a corresponding reduction peak at 2.0 V (R3), Figure \ref{Figures/chap4fig:fig2} a). Two more reduction peaks were found at 1.6 V (R2) and 0.9 V (R1). However, their peak intensities decreased with every scan. CV scans of Al/\ce{MoSe2} cells in Figure \ref{Figures/chap4fig:fig2} b) indicated a reversible electrochemical process, which was in agreement with their CDCs. The scans overlapped with each other displaying two oxidation peaks at 1.7 V (O'1) and 2.1 V (O'2) and corresponding reduction peaks at 1.8 V (R'1) and 1.6 V (R'2). In Figure \ref{Figures/chap4fig:fig2} c) an oxidation and a reduction peak at 1.7 V (O''1) and 1.8 V (R''1) was observed for Al/MoSSe respectively. R''1's peak intensity increased after every scan, which might suggest sluggish kinetics in the system; perhaps due to strong interaction between the host material and the intercalating anion. The voltammogram became more capacitor-like after a few scans, indicating the absence of reversible redox processes.

\begin{figure}
  \centering
  \includegraphics[width=\textwidth]{Figures/chap4fig/fig2}
  \caption{Cyclic voltammograms of a) \ce{MoS2}, b) \ce{MoSe2} and c) MoSSe at a scan rate of 10 mV s$^{-1}$ in a two-electrode aluminium-ion cell against an Al/\ce{Al^{3+}} reference electrode.}
  \label{Figures/chap4fig:fig2}
\end{figure} 

\begin{figure}
  \centering
  \includegraphics[width=\textwidth]{Figures/chap4fig/fig3}
  \caption{X-ray diffraction patterns of pristine (black), charged (green) and discharged (red) a) \ce{MoS2}, b) \ce{MoSe2} and c) MoSSe electrodes charged to 2.35 V and discharged to 0.2 V vs. Al/\ce{Al^{3+}}, with International Centre for Diffraction Data (ICDD) references.}
  \label{Figures/chap4fig:fig3}
\end{figure}

Figure \ref{Figures/chap4fig:fig3} shows the XRD patterns of \ce{MoS2}, \ce{MoSe2} and MoSSe electrodes. Pristine (in black), charged (in green) and discharged (in red) cathodes were compared after 30 cycles. \ce{MoS2} cells displayed a very small shift in their d-spacings. The peak at 14.2$^{\circ}$ (6.22 \AA) shifted to 14.02$^{\circ}$ (6.32 \AA), as shown in Figure \ref{Figures/chap4fig:fig3} a). Most of the peaks retained their positions after charge and discharge showing no significant change in the lattice dimensions. A completely different XRD pattern appeared after charging for Al/\ce{MoSe2} cells, as new peaks appeared at 2$\theta$ values, displayed in Figure \ref{Figures/chap4fig:fig3} b). After discharge, the diffraction patterns of the discharged cathodes resembled the pristine cathode patterns. Every time the cells were charged, \ce{MoSe2} seemed to adopt this new crystal lattice. However, the characteristic peaks of \ce{MoSe2} reappeared after discharge. This follows closely the observations made by Rani \textit{et al.} \cite{rani_fluorinated_2013}, where they proved intercalation of ions into the layers of fluorinated natural graphite during charging. This strongly confirms our hypothesis of a reversible intercalation taking place in \ce{MoSe2}. It was interesting to note that MoSSe did not have a well-defined crystal structure to begin with, Figure \ref{Figures/chap4fig:fig3} c). The patterns after charge and discharge did not look any different from the untested cathode. This confirmed MoSSe layers did not undergo any expansion and the initial specific capacities came from non-faradaic reactions where \ce{AlCl4-} might have been electrostatically absorbed and desorbed onto the surface of the electrode.  

To further understand the interactions between \ce{AlCl4-} and \ce{MoSe2} we used XPS, which is a useful method for distinguishing various oxidation states and helps in identifying different polymorphs (2H and 1T) \cite{fan_hybrid_2017}. The detailed narrow spectrum scans in Figure \ref{Figures/chap4fig:MoAlOverallMoSeMoSSe} show the binding energies of Mo (3d$_{5/2}$ and 3d$_{3/2}$, Figure \ref{Figures/chap4fig:MoAlOverallMoSeMoSSe} a) and b)) and Al 2p peaks for charged \ce{MoSe2} (Figure \ref{Figures/chap4fig:MoAlOverallMoSeMoSSe} c)) and MoSSe electrodes (Figure \ref{Figures/chap4fig:MoAlOverallMoSeMoSSe} d)). In pristine \ce{MoSe2}, two peaks appeared at 229.1 eV and 232.2 eV corresponding to 3d$_{5/2}$ and 3d$_{3/2}$ (Figure\ \ref{Figures/chap4fig:MoSeSeAlClPrtChg} a)). Selenium displayed a doublet at 55.4 eV and 54.6 eV corresponding to Se 3d$_{3/2}$ and 3d$_{5/2}$ respectively (Figure \ref{Figures/chap4fig:MoSeSeAlClPrtChg} c)). Peak splitting in an XPS spectrum can indicate a phase change or a change in oxidation state. After charge, the peak for Mo 3d split into three doublets indicating the presence of multiple oxidation states or phases of Mo (Figure \ref{Figures/chap4fig:MoAlOverallMoSeMoSSe} a)). Se 3d deconvoluted into four peaks after charge, Figure \ref{Figures/chap4fig:MoSeSeAlClPrtChg} e), confirming presence of more than one phase after charge. This was similar to observations made by Fan \textit{et al.} Pristine electrodes of MoSSe contained Mo in more than one oxidation state, and provided evidence for the presence of both 2H and 1T polymorphs, Figure\ \ref{Figures/chap4fig:MoSeSeAlClPrtChg} b). After charging, the width of peaks at 231.7 eV (Mo 3d$_{5/2}$) and 228.6 eV (Mo 3d$_{3/2}$) increased as displayed in Figure\ \ref{Figures/chap4fig:MoAlOverallMoSeMoSSe} b). After comparing Figure\ \ref{Figures/chap4fig:MoSeSeAlClPrtChg} d) and f), we noticed that the Se 3d spectrum deconvoluted into four peaks after charging in MoSSe cells. An increase in the peak width was observed for both Mo and Se binding energies. A new peak at $\sim$236 eV in Mo 3d spectra for \ce{MoS2}, \ce{MoSe2} and MoSSe electrodes generally indicates a Mo$^{6+}$ species present in molybdenum oxide, \ce{MoO3}. 

\begin{figure}
  \centering
  \includegraphics[width=\textwidth]{Figures/chap4fig/MoAlOverallMoSeMoSSe}
  \caption{XPS spectra of Mo 3d orbitals in a charged a) \ce{MoSe2} and b) MoSSe cathode and Al 2p orbitals in a charged c) \ce{MoSe2} and d) MoSSe cathode. e) An overview spectrum of all three tested and charged cathodes.}
  \label{Figures/chap4fig:MoAlOverallMoSeMoSSe}
\end{figure}

The peak shifts detected in Mo 3d spectra for charged \ce{MoS2} and \ce{MoSe2} cathodes were insignificant in MoSSe (cf.\ Figures \ref{Figures/chap4fig:MoS2XPS} a), b), \ref{Figures/chap4fig:MoAlOverallMoSeMoSSe} a), and \ref{Figures/chap4fig:MoSeSeAlClPrtChg} a)). This further confirms the absence of redox reactions and that the capacity was mainly derived from a surface-based charge storage. As expected, charged electrodes showed higher concentration of aluminium and chlorine than discharged electrodes as seen in Figure \ref{Figures/chap4fig:MoSeSeAlClPrtChg} g) and h). The XPS spectra support the observation that \ce{MoSe2} underwent a phase transformation that made it a better performing cathode than \ce{MoS2}. Further analysis is needed to fully understand the mechanism of MoSSe.

\begin{figure}
  \centering
  \includegraphics[width=\textwidth]{Figures/chap4fig/MoS2XPS}
  \caption{XPS spectra of Mo 3d orbitals in a charged and b) discharged \ce{MoS2} cathode and binding energies of S 2p orbital in a charged and b) discharged \ce{MoS2} cathode.}
  \label{Figures/chap4fig:MoS2XPS}
\end{figure}

\begin{figure}
  \centering
  \includegraphics[width=\textwidth]{Figures/chap4fig/MoSeSeAlClPrtChg}
  \caption{XPS spectra of Mo 3d for pristine a) \ce{MoSe2} and b) MoSSe electrodes. The spectra of \ce{MoSe2} are composed of two peaks at 232.2 eV and 229.1 eV corresponding to \ce{Mo^{4+}}. MoSSe spectra consist of three doublet bands, which were assigned to \ce{Mo^{4+}}, one with higher oxidation state and another band corresponding to \ce{Mo^{6+}} at 236 eV. c) Pristine and e) charged Se 3d from \ce{MoSe2}. \ce{MoSe2} observed peaks corresponding to 3d$_{3/2}$ and 3d$_{5/2}$ at 55.6 eV and 54.6 eV respectively. Binding energies of Se 3d from d) pristine and f) charged MoSSe cathodes.}
  \label{Figures/chap4fig:MoSeSeAlClPrtChg}
\end{figure}

Charged \ce{MoSe2} electrodes displayed binding energies of Al 2p at 77 eV (red) and 76 eV (in blue) corresponding to chlorides (\ce{AlxCly}) and \ce{Al2O3} respectively in Figure \ref{Figures/chap4fig:MoAlOverallMoSeMoSSe} c). New peaks were observed at much lower binding energies --- 69 and 70 eV (green) suggesting the presence of a new complex with an increased electron density around aluminium. An overall spectra of charged \ce{MoS2}, \ce{MoSe2} and MoSSe cathodes is shown in Figure \ref{Figures/chap4fig:MoAlOverallMoSeMoSSe} e) indicating the presence of Al and Cl (from chloroaluminates) and oxygen (from \ce{MoO3}).

In addition, we compared the Raman spectra of pristine and charged cathodes to detect shifts in vibrational modes, Figure \ref{Figures/chap4fig:fig5}. E$^1_{2g}$ and A$^1_g$ are the most intense vibrational modes for molybdenum dichalcogenides \cite{yang_pressure-induced_2019, r_2d_2017,sharma_stable_2018}. Peaks corresponding to E$^1_{2g}$ and A$^1_g$ modes for \ce{MoS2} (Figure \ref{Figures/chap4fig:fig5} a)) are prominent at 384.6 cm$^{-1}$ and 410.2 cm$^{-1}$ respectively. A$^1_g$ indicates an out-of-plane symmetric displacement of S atoms, whereas E$^1_{2g}$ suggests an in-layer displacement. Also, separation between the two peaks indicates a multi-layer structure, which was observed for all three materials. No significant peak shift or peak broadening was observed for the charged \ce{MoS2} electrode. For 2H \ce{MoSe2} (Figure \ref{Figures/chap4fig:fig5} b)), A$^1_g$ is the most intense vibration occurring at a frequency lower than that of E$^1_{2g}$. When the number of layers decreases, the A$^1_g$ mode softens (increase in full-width-at-half-maximum (FWHM)). Spectra generated after intercalation were different from the pristine cathodes because phase conversion from 2H to 1T decreases the molecule's symmetry and more Raman bands get active. The presence of J1 and J2 peaks in addition to E$^1_{2g}$ and A$^1_g$ at lower wavelengths suggest the existence of 1T phase especially for \ce{MoSe2} and MoSSe (inset, Figure\ \ref{Figures/chap4fig:fig5} b) and c)). This agrees with our CV scans where a phase transition was observed for \ce{MoSe2} and MoSSe. Raman results suggest that 'chloroalumination' or insertion of chloroaluminates changed the symmetry and vibrational modes of \ce{MoSe2}'s crystal lattice. 

\begin{figure}
  \centering
  \includegraphics[width=\textwidth]{Figures/chap4fig/fig5}
  \caption{Raman spectra of pristine (black) and charged (green) a)\ce{MoS2}, b) \ce{MoSe2} and c) MoSSe electrodes with position of new Raman active J1 and J2 bands marked along with E$^1_{2g}$ and A$^1_g$ bands.}
  \label{Figures/chap4fig:fig5}
\end{figure}

\begin{figure}
  \centering
  \includegraphics[width=\textwidth]{Figures/chap4fig/mose2tem}
  \caption{Raman spectra of pristine (black) and charged (green) a)\ce{MoS2}, b) \ce{MoSe2} and c) MoSSe electrodes with position of new Raman active J1 and J2 bands marked along with E$^1_{2g}$ and A$^1_g$ bands.}
  \label{Figures/chap4fig:mose2tem}
\end{figure}
 
%\section*{Preface}
In this chapter, we discuss the performance of AIBs using carbonised natural products such as human hair and hemp fibers, and a few other carbon-based materials as cathodes and try to establish their mechanism.
\pagebreak
\chapter{Carbon-based cathodes for rechargeable AIBs} % Main chapter title

\label{chap5} % For referencing the chapter elsewhere, use \ref{Chapter1} 

\section{Theory and background}
Different varieties of carbon-based materials have been widely used in energy storage applications. Activated carbons are the most widely used materials today, because of their high specific surface area and low cost \cite{wang_review_2012}. They are derived by carbonisation (heat treatment) of carbon-rich compounds in an inert atmosphere. Precursors can range from natural materials, such as human hair, rice husk, coconut shells, wood carvings, to synthetic materials, such as polymers \cite{hulicovajurcakova_combined_2009,si_tunable_2013,yalcin_studies_2000,barton_tailored_1999}                                 . A porous network inside the bulk of the carbon particles is created after activation. Accordingly, the porous structure of carbon is characterized by a broad distribution of pore size. Micropores (<2 nm in size), mesopores (2–50 nm) and macropores (>50 nm) can be created in carbon grains. Longer activation time or higher temperature leads to larger mean pore size \cite{simon_materials_2008}. In 2000, Sony commercialised LIBs, where metallic lithium was replaced by a carbon host structure that reversibly absorbed and released \ce{Li+} ions at low electrochemical potentials \cite{ozawa_lithium-ion_1994}.
During discharge at the anode, lithium is oxidised. Lithium ions are released from the carbon, along with electrons:
\begin{equation}
\ce{LiC6} \longrightarrow x\ce{Li+} \ce{+xe-} \+ \ce{C6}
\end{equation}
At the cathode, lithium-ions are absorbed by the lithium dioxide, and the electrode is reduced as it also receives the electrons from the circuit:
\begin{equation}
\ce{Li_{1-x}CoO2} \+ x\ce{Li+} x\ce{e-} \longrightarrow \ce{LiCoO2}
\end{equation}
The overall reaction is:
\begin{equation}
\ce{C6} \+ \ce{LiCoO2} \rightleftharpoons \ce{LixC6} \+ \ce{Li_{1-x}CoO2}
\end{equation}
The need for carbon was to allow high capacity for reversible \ce{Li+} absorption at a potential close to lithium metal. This was a significant breakthrough, and was immediately followed by all major battery manufacturers. Researchers coined the term 'rocking-chair' for the  continuous intercalation/ de-intercalation of ions from one electrode to the other during charge/ discharge cycles in the LIBs. After Sony started using lithiated graphite, a large variety of carbons with low crystallinity were tested as electrode materials. They revealed very different electrochemical behaviours due to their different micro and macroscopic structures \cite{yoo_large_2008}. In most of the LIBs we use today, lithiated carbon (\ce{LiC6}) is used as an anode and a lithium metal oxide acts as the cathode (\ce{LiCoO2}, \ce{LiNiO2}, \ce{LiMn2O4}).  
%In early attempts, graphite could not be used as an anode because it reacted strongly with the electrolyte (6,7). Armand \textit{et al.} in the 1980's found out that lithiated carbon was a suitable alternative.  

\begin{table}
\caption{Characteristics of commonly used rechargeable batteries.} \label{tabCref}
\begin{center}
\begin{tabular}{ |p{0.5cm}|p{2.5cm}|p{2cm}|p{2.5cm}|p{2.5cm}|p{1.5cm}|}
\hline
\textbf{Ref} & \textbf{Cathode} & \textbf{Electrolyte} & \textbf{Specific capacity (mAh g$^{-1}$)} & \textbf{Current rate (mA g$^{-1}$)} & \textbf{No. of cycles} \\
\hline
\cite{wang_advanced_2017} & Natural graphite & EMImCl & 60 & 100 & 6000 \\
\cite{song_long-life_2017} & Graphite & NaCl & 43 & 100 & 9000 \\
\cite{sun_new_2015} & Carbon paper & EMImCl & 63 & 50 & 50+ \\
\cite{lin_ultrafast_2015} & 3D graphite foam & EMImCl & 70 & 4000 & 7500 \\
\cite{rani_fluorinated_2013} & Fluorinated natural graphite & DiBIMCl & 225 & 60 & 40 \\
\cite{stadie_zeolite-templated_2017} & Zeolite template carbon & EMImCl & 60 & 100 & 50+ \\
& Graphite & EMImCl & 116 & 5000 & 50+ \\*

\hline
\end{tabular}
\end{center}
\end{table}

Graphite is a naturally occurring crystalline form of carbon and is found in abundance in many places around the world. As a result, the cost of raw graphite remains low. It has been used as an electrode in other battery systems as well. It has a layered structure, and a good thermal and electrical conductivity. It turned out to be a ideal intercalation electrode material in LIBs \cite{ji_recent_2011, yoo_large_2008, lian_large_2010}. For low cost aluminium battery systems, a graphite electrode could be appealing. Studies reveal that reversible ion intercalation into graphite is responsible for the storage capacities in AIBs \cite{rani_fluorinated_2013, lin_ultrafast_2015}. \ce{AlCl4-} anions intercalate into the graphitic layers when the cell is being charged and deintercalate during discharge. Different forms of graphite such as fluorinated graphite, kish graphite flakes, three-dimensional graphitic-foam, graphene aerogels and several other forms have been tested as cathodes for AIBs, which showed specific capacities ranging from 60-250 mAh g$^{-1}$ \cite{rani_fluorinated_2013, wang_kish_2017, wu_3d_2016, huang_graphene_2019}. Post mortem analysis of the cathodes in charged / discharged state revealed the intercalation/ de-intercalation of \ce{AlCl4-} in the graphene layers. For example, the \textit{ex- situ} X-ray diffraction patterns of a charged graphite cathode displayed the disappearance of the sharp graphitic (002) peak at 2$\theta$ = 26.55$^{\circ}$. Two new peaks emerged at 28.25$^{\circ}$ and 23.56$^{\circ}$. The peak separation suggested expansion of the graphitic host layers to 5.7\AA. Since the size of \ce{AlCl4-} anions is 5.28\AA, it confirmed the intercalation of \ce{AlCl4-} anions into the cathode, shown in Figure \ref{Figures/chap5fig:ramanpap} \cite{lin_ultrafast_2015, wang_kish_2017}. \\*

 \begin{figure}[h]
  \centering
  \includegraphics[width=0.75\textwidth]{Figures/chap5fig/ramanpap}
    \caption{\textit{Ex-situ} x-ray diffraction patterns of natural graphite in various charging and discharging states denoted in blue and red in the figure, respectively) through the second cycle. \cite{wang_advanced_2017}}
  \label{Figures/chap5fig:ramanpap}
\end{figure}

%Activated carbon, owing to its porous structure, provides a high surface area for absorption of electrolyte ions in super-capacitors \cite{eliad_ion_2001, zhu_carbon-based_2011-2}. X-ray diffraction (XRD) and Raman spectroscopy studies have widely been used to establish this mechanism \cite{rani_fluorinated_2013, wang_advanced_2017, lin_ultrafast_2015-3} as shown in Figure \ref{Figures/chap5fig:graphmech}. XPS studies have confirmed reversible oxidation/reduction of carbon when \ce{AlCl4-} anions intercalate/deintercalate respectively \cite{stadie_zeolite-templated_2017, liu_binder-free_2019}.

%Graphene, discovered in 2004, has very high crystallinity. Its one of the thinnest and strongest materials known to mankind. It is basically a 2D crystal with monolayers of carbon atoms arranged in a honeycomb structure with a 6-membered ring. Graphene, building block of graphite, has the maximum surface area to volume ratio in layered materials.

 \begin{figure}[h]
  \centering
  \includegraphics[width=\textwidth]{Figures/chap5fig/graphmech}
    \caption{Intercalation of \ce{AlCl4-} ions during cell charge and de-intercalation during discharge in a Al/graphite cell. The interlayer distance between two graphite sheets is 3.3 \AA.}
  \label{Figures/chap5fig:graphmech}
\end{figure}

In this chapter, four different carbon-based materials- activated carbon from human hair, activated carbon from hemp fibers, fullerenes and Super-P carbon black were investigated as cathodes for AIBs. Super-P is an amorphous form of carbon. It is highly conductive and is added in electrode slurries to enhance the conductivity of a cathode material. Activated carbon derived from natural products such as rice husk, coconut shells or wood, have been previously used in batteries and super-capacitors \cite{hussain_development_2019, frackowiak_carbon_2001}. Both these amorphous structures contain pores of various sizes (mesopores and micropores). Fullerenes, on the contrary, have a cage-like structure that gives them a very high surface area. Materials with high surface area allow adsorption of ions on their surfaces and therefore these materials were chosen as potential cathodes for AIBs.   

\section{Results and discussion}

\begin{table}[h!]
\caption{Comparing battery metrics of all carbon-based cathodes tested in this work} \label{table1bm}
\begin{center}
\begin{tabular}{|lcccc|}
\hline
Active material & {\textbf{Size}} & {\textbf{Specific capacity}} & {\textbf{Cell efficiency}} & {\textbf{Cell voltage}}\\
 & {\textbf{(pore size)}} & {\textbf{(mAh g$^{-1}$)}} & {\textbf{(\%)}} & {\textbf{(V)}}\\
\hline
Human hair & 5${\mu}$ m & 102 & 97 & 1.9 \\
Fullerene mix & 8.8 \AA & 78 & 85 & 1.7 \\
Hemp fibers & 2.3 $\mu$ m & 49 & 75 & 1.8 \\
SPCB & 300 \AA & 46 & 40 & 1.5 \\
\hline  % Please only put a hline at the end of the table
\end{tabular}
\end{center}
\end{table}

\begin{figure}[h]
  \centering
  \includegraphics[width=\textwidth]{Figures/chap5fig/cdcall}
    \caption{Specific capacities of AC (human hair, hemp fibers), CFEx and SPCB in their a) first and b) 50$^{th}$ cycle at a current rate of 50 mA g$^{-1}$. c) Coulombic efficiencies (CEs) of cells at a current rate of 50 mAg$^{-1}$. d) Galvanostatic charge/discharge profile of all cells at various current rates ranging from 25 mAg$^{-1}$ to 100 mAg$^{-1}$ in a two-electrode setup against Al$^{3+}$/Al.}
  \label{Figures/chap5fig:cdcall}
\end{figure}

AC derived from human hair and hemp fibers, and SPCB, both have a non-crystalline structure. However their Raman spectra revealed the presence of a few graphitic planes. These planes would allow \ce{AlCl4-} ions to intercalate during charging. CFEx (a mixture of \ce{C60} and \ce{C70} fullerenes) does not have a layered structure. Fullerenes have a cage-like morphology and the chloroaluminates are not small enough to move in and out of them during charge/discharge. It was assumed that \ce{AlCl4-} anions migrated through the gaps present in between the fullerenes and stored charge on its surface with the electron transfer taking place on its surface. Using the battery analyser, specific capacities and CEs of the cathodes were recorded at various current rates (cf.\ Figure \ref{Figures/chap5fig:cdcall}a and b). Morphology of the cathodes before and after the cycles were studied using Raman spectroscopy, X-ray diffraction (XRD) patterns and scanning electron microscopy (SEM).

Figures \ref{Figures/chap5fig:cdcall}a and b compare the specific capacities of all cells for their first and 50$^{th}$ cycles. Human hair cathodes exhibited a high capacity of $\sim$100 mAh g$^{-1}$ with CE of $\sim$95$\%$ shown in Figure \ref{Figures/chap5fig:cdcall}c. Hemp batteries displayed a capacity of 56 mAh g$^{-1}$ in their first cycle, which decreased to 45 mAh g$^{-1}$ after 50 cycles. CFEx displayed a capacity of around 80 mAh g$^{-1}$ with CE of $\sim$90\% and maintained that for 50 cycles. With an initial value of 84 mAh g$^{-1}$, specific capacity of SPCB decreased to 47 mAh g$^{-1}$ and a low CE of $\sim$40\% was observed. Discharging capacity of SPCB and hemp fibers decreased considerably after repeated charge/discharge cycles. A low CE that was observed in both hemp and SPCB, can be attributed to side reactions in a battery. These may include electrode or electrolyte interactions with impurities, or degradation of the cathode structure (pulverisation) \cite{gyenes_understanding_2015}. However, capacity fade for CFEx and human hair cells was minimal. This suggests that CFEx and human hair have a more stable structure and have the potential to store charge reversibly \cite{pramanick_human_2016}.\\


\fbox{\begin{minipage}{35.5em}
\section*{Activation of carbon}
The production of activated carbon consists of carbonisation of a precursor at a temperature below 900$^{\circ}$ C in an inert atmosphere and a chemical or physical activation of the carbonised precursor. Activating agents play an important role in determining the porosity of an AC \cite{arenas_effect_2004}. Using alkali hydroxides at high temperature creates micropores which increases the surface area of the material \cite{dong_commercial_2019, liu_hair-based_2017}. In this work, sodium hydroxide (NaOH) was used as the activating agent. The reaction that takes place inside the carbon matrix after adding NaOH is as follows:

\begin{center}
    4NaOH + C $\longrightarrow$ 4Na + 4\ce{CO2} + 2\ce{H2O} \cite{satish_macroporous_2015}
\end{center}

In this reaction, NaOH was reduced to free metal, Na. These atoms in turn expanded the carbon matrix after intercalating into the carbon structure. High temperature forced the  atoms out of the carbon matrix, thus creating micropores. Oxidation of carbon from oxygen atoms of the hydroxide group formed carbon dioxide (\ce{CO2}), providing routes for channeling the sodium atoms into the internal structure, resulting in a well-connected porous structure \cite{satish_macroporous_2015}. The calcinating temperature used here was 750$^{\circ}$ C. Figure \ref{Figures/chap5fig:achsyn} illustrates a flowchart describing an AC synthesis. \\*
\end{minipage}}

\begin{figure}[h]
\centering
\includegraphics[width=35.5em]{Figures/chap5fig/achsyn}
\caption{Synthesis of activated carbon (AC) from human hair using NaOH as the activating agent.}
\label{Figures/chap5fig:achsyn}
\end{figure}


\begin{figure}[h]
\centering
\includegraphics[width=0.75\textwidth]{Figures/chap5fig/cfexsol}
\caption{Comparison of solubility of a) pure \ce{C60} fullerene b) fullerene extract (CFEx) and AC derived from hemp fibers in \ce{AlCl3}/EMImCl ionic liquid electrolyte.}
\label{Figures/chap5fig:cfexsol}
\end{figure}

Super-P is an amorphous form of carbon with a high surface area of 62 m$^2$ g$^{-1}$ and a highly disordered structure \cite{see_reversible_2017}. The pore sizes range from $\sim$30-50 nm \cite{younesi_analysis_2015}. As a cathode material, SPCB underwent the highest capacity loss after 30 cycles (45\%). It seems continuous cycling destroyed the structural arrangement of the carbon atoms, which resulted in further alleviated capacity and low CEs.\\*
Fullerenes have a fused-ring structure with a nucleus-to-nucleus diameter of 7.1\AA\ and a van der Waals (vdW) diameter of 11\AA\ in a single crystal. However, they are zero-dimensional materials, which means they cannot provide an efficient path for electron transport or a long-range conductivity \cite{winkler_two-component_2007}. They are known to be weak battery materials owing to their solubility in electrolytes, especially in LIBs \cite{seger_prospects_1991}. To test their solubility in the \ce{AlCl3}-EMImCl electrolyte, 100 mg of CFEx was mixed in the electrolyte and stirred for 24 hours. The solution was left to stand for another 24 hours inside a \ce{N2}-filled glove box. CFEx dissolved in the electrolyte (Figure \ref{Figures/chap5fig:cfexsol}) since no phase separation was observed. It has been reported that poly-sulphides (formed during charge/discharge cycles) are soluble in the electrolyte of a Li-S battery. They form an insulating layer of \ce{Li2S} on the anode, which results in capacity fading \cite{sun_effect_2017}. Since no such effect was observed in aluminium-fullerene cells, the solubility of fullerenes in \ce{AlCl3}/EMImCl does not seem to impact its charge-storing capacity. On the contrary, these batteries demonstrated excellent capacity retention at various current rates(Figure \ref{Figures/chap5fig:cdcall}d).\\*

\newpage
 \begin{figure}[h!]
  \centering
  \includegraphics[width=\textwidth]{Figures/chap5fig/allmech}
    \caption{Suggested mechanism for an a) \textbf{Al-CFEx} cell, b) \textbf{Al-hemp} cell, hemp fibers have pore sizes as large as 2.0-2.5 $\mu$m allowing the \ce{AlCl4-} to get absorbed on their surface. However, agglomeration of these fibers after a few cycles reduces the number of active sites available for effective charge storage, and c) \textbf{Al-Super-P} cell, chloroaluminates intercalate into the very few graphitic planes in Super-P, while few anions adsorb onto its surface. However, further cycling leads to cathode pulverisation, which results in capacity fading. }
  \label{Figures/chap5figs:allmech}
\end{figure}

 \begin{figure}[h!]
  \centering
  \includegraphics[width=\textwidth]{Figures/chap5fig/raman}
    \caption{Raman spectra of pristine (in black) and charged (in red) a) CFEx, b) AC from hemp fibers, c) from human hair (ACH) and d) Super-P cathodes displaying presence of both D and G bands.}
  \label{Figures/chap5fig:raman}
\end{figure}

\section*{Raman analysis}
It has been previously established that chloroaluminates intercalate into the graphitic planes during charging \cite{lin_ultrafast_2015}. Since both AC and SPCB cathodes tested in this work have graphitic planes present in them, this meant intercalation would occur during cell charging. To study these changes, charged cathodes of all materials analysed via Raman spectroscopy and compared with the pristine ones. The data obtained is displayed in Figure \ref{Figures/chap5fig:raman}. Graphite has a D-band present at 1300 cm$^{-1}$, which originates from a hybridized vibrational mode and is associated with graphene edges. Pristine hair, hemp and SPCB cathodes also had a significant D-band present in them, indicating absence of an ordered structure at $\sim$1300 cm$^{-1}$ for human hair, 1329.7 cm$^{-1}$ for hemp fibers, and 1352.0 cm$^{-1}$ for SPCB). Both the peaks at 1600 cm$^{-1}$ and 1350 cm$^{-1}$ are broad due to the presence of \ce{sp2} clusters like $\alpha$-carbons, which have a bond angle disorder \cite{shimodaira_raman_2002}.\\*
Raman spectra of CFEx in Figure \ref{Figures/chap5fig:raman}c displayed the characteristic bands of both \ce{C60} and \ce{C70} molecules. A 'pentagonal pinch mode', usually observed in a \ce{C60} Raman spectrum, was present at 1460 cm$^{-1}$. It was observed that \ce{C70} had multiple bands. This was due to its reduced molecular symmetry, which increased the number of vibrational modes, consequently increasing the number of active Raman bands \cite{kimbrell_analysis_2014}. It was interesting to note that the spectra of charged CFEx looked strikingly similar to the pristine ones. Since Raman spectroscopy is sensitive to minute differences in the molecular morphology, results suggested that the fullerenes did not undergo any significant structural change during the cycles. As a result, the cells displayed a highly stable CE. Furthermore, fullerenes stored the same amount of charge after every cycle (similar discharge capacity after 50 cycles) without undergoing any significant structural changes. This suggested to wards a mechanism different than intercalation!\\* 
Charged AC (from hair) cathodes showed an increased full-width at half maxima (FWHM), which suggested an increase in the lattice defects and structure deformities. These defects would arise if the chloroaluminates intercalated into the few graphitic planes present in the carbon matrix. However, a highly porous structure would also allow surface adsorption of charge carrying species much like in super-capacitors \cite{beguin_carbons_2014}. Since activated carbon lacks the long-range ordered structure present in natural graphite, intercalation into graphitic planes, along with surface adsorption of chloroaluminates, can attribute to its high capacity\cite{brezesinski_ordered_2010}. This might have resulted in aluminium-hair batteries performing better than other materials by storing more charge. Schematic of an aluminium-hair battery is illustrated in Figure \ref{Figures/chap5fig:achmech}.\\*

 \begin{figure}[h!]
  \centering
  \includegraphics[width=\textwidth]{Figures/chap5fig/achmech}
    \caption{Suggested mechanism for an aluminium-human hair cell. Chloroaluminates (\ce{AlCl4-}) intercalate into the few graphitic planes and micro/mesopores present in them, in addition to surface adsorption of ions displaying both Faradaic and non-Faradaic processes for charge storage.}
  \label{Figures/chap5fig:achmech}
\end{figure}

Both hemp fibers and SPCB had a highly disordered structure to begin with. Repeated intercalation or absorption of ions on their surface in the first few cycles would have further damaged their structure, which is one of the reasons why hemp and SPCB batteries failed to retain their capacity. However, CFEx does not have graphitic planes to intercalate chloroaluminate ions. Since the fullerenes have a very high surface area, surface adsorption of ions is highly likely \cite{adams_van_1994}. Furthermore, it might be possible for the anions to seep through the gaps present in between two fullerenes (as shown in Figure \ref{Figures/chap5figs:allmech}a) and for the surface-based redox processes to take place in a systematic way.\\*

\newpage
\begin{figure}[h!]
  \centering
  \includegraphics[width=\textwidth]{Figures/chap5fig/SEM}
    \caption{Scanning electron microscopy (SEM) images comparing pristine a) human hair, b) hemp, c) CFEx and d) SPCB and charged e) human hair, f) hemp, g) CFEx and h) SPCB cathodes. Hemp fibers and SPCB undergo permanent changes after charge/discharge cycles and fail to retain capacity.}
  \label{Figures/chap5fig:SEM}
\end{figure}

\section*{SEM analysis}
Figure \ref{Figures/chap5fig:SEM} shows the SEM images of pristine (Figure \ref{Figures/chap5fig:SEM}a, b, c and d) and charged (\ref{Figures/chap5fig:SEM}e,f,g and h) cathodes of all carbon materials. Human hair and hemp fibers have a highly porous structure as seen in Figure \ref{Figures/chap5fig:SEM}b and d. However, hemp lost its surface porosity after 30 cycles, Figure \ref{Figures/chap5fig:SEM}f). In addition, Figure \ref{Figures/chap5fig:SEM}d and h implies that SPCB underwent significant agglomeration during cycles. It was interesting to note that the fullerene retained its morphology before and after cycles.\\*

\begin{figure}[h!]
  \centering
  \includegraphics[width=\textwidth]{Figures/chap5fig/XRD}
    \caption{X-ray diffraction patterns of a) CFEx and b)human hair cathodes to study changes in their lattice after galvanostatic cycles in a two-electrode setup against \ce{Al$^{3+}$/Al} with characteristic peaks marked for \ce{C60} (in grey boxes) and \ce{C70} (in blue inverted triangles).}
  \label{Figures/chap5fig:XRD}
\end{figure}

Since activated carbon from human hair and fullerenes were the best performing cathodes in this project, x-ray diffraction patterns were studied to establish their meachnism. Pristine (in black), charged (in green) and discharged (in red) cathodes of CFEx and human hair are showed in Figure \ref{Figures/chap5fig:XRD}a and b respectively. Figure \ref{Figures/chap5fig:XRD}a displayed the characteristic peaks of both \ce{C60} and \ce{C70} at 2$\theta$ values of 10.9$^{\circ}$, 17.8$^{\circ}$, 20.9$^{\circ}$ and 28.2$^{\circ}$ for \ce{C60} and 18.9$^{\circ}$, 19.3$^{\circ}$ and 21.8$^{\circ}$ for \ce{C70} molecules. New diffraction peaks at lower 2$\theta$ values appeared for charged electrodes. However, after discharge, the XRD patterns looked similar to the diffraction peaks of the pristine cathode. This data strongly suggests a reversible process taking place during cycles. To confirm this, the unit cell lattice parameters for both pristine and charged cathodes for a \ce{C60} molecule were calculated. The unit cell had a tetragonal crystal system with space group of P42/ mmc and a space group number 131 (ICDD: 04-013-1339). Lattice parameters 'a' and 'b' for the charged cathode increased from 9.06 \AA\ to 9.57 \AA\ and 'c' increased from 15.03 \AA\ to 15.65 \AA, as shown in Figure \ref{Figures/chap5fig:cfexcrys}a and b. Lattice parameters of a discharged fullerene were closer to pristine values. These changes suggested a reversible insertion of chloroaluminates into the free spaces between fullerene molecules taking place. A possible site for \ce{AlCl4-} intercalation is depicted in Figure \ref{Figures/chap5fig:cfexcrys}c. XRD patterns of human hair cathodes were inconclusive (Figure \ref{Figures/chap5fig:XRD}b). The pristine cathode displayed broad peaks that confirmed a structure, which was amorphous and highly porous. However, the material became more symmetrical and crystalline after cycles. Charged and discharged cathodes exhibited similar looking patterns implying no change in the newly formed crystal lattice during cycles. It is to be noted that presence of crystallinity in an active material does not limit the surface-based charge storing capacity \cite{kim_synthesis_2006, jow_factors_2018}.Further analysis is required to investigate this unique behaviour and establish the mechanism of an Al/hair battery.\\*

\begin{figure}[h!]
  \centering
  \includegraphics[width=\textwidth]{Figures/chap5fig/cfexcrys}
    \caption{Changes in the lattice parameters of a \ce{C60} unit cell. a) Pristine \ce{C60}unit cell, b) charged \ce{C60}unit cell with increased parameters suggesting a uniform shift in the lattice after charge/discharge. c) Expected intercalation sites of \ce{AlCl4-} ions in the unit cell.}
  \label{Figures/chap5fig:cfexcrys}
\end{figure}

%Charged Super-P and hemp fiber electrodes underwent degradation and appeared clumped together resulting in capacity decay. This was visible from their electrochemical results where a rapid decrease in capacity and cell efficiency was noted. 
%XPS analysis results... C 1s 
\begin{figure}[h!]
  \centering
  \includegraphics[width=\textwidth]{Figures/chap5fig/XPSC}
    \caption{Carbon 1s XPS spectra of pristine a) hair, b) Super-P, c) hemp fibers and d) CFEx cathodes. While AC from human hair, hemp fibers and Super-P contain carbonyl functional groups, CFEx cathodes have symmetrical looking $\pi$* and $\sigma$* satellite peaks.}
  \label{Figures/chap5fig:XPSC}
\end{figure}
\section*{XPS analysis}
XPS spectra of carbon 1s orbital of all pristine cathodes is shown in Figure \ref{Figures/chap5fig:XPSC}. Hair, hemp fibers and SPCB had similar looking peaks for the 1s orbital. All three cathodes displayed peaks for sp$^2$ C-C/ C-H, C-O/ C-OH, O-C=O/ C=O and C-F bonds. Since hair is mainly composed of a protein called keratin (Figure \ref{Figures/chap5fig:keratin}), it can be deduced that multiple chemical environments of carbon, shown in Figure \ref{Figures/chap5fig:XPSC}a), are derived from Keratin. Sulphide bonds are an essential part of this protein and a C-S binding energy was observed at 286.94 eV (blue peak).\\*

\begin{figure}[h!]
\centering
\includegraphics[width=0.5\textwidth]{Figures/chap5fig/keratin}
\caption{Keratin: a protein abundantly found in human hair contains C-O, C=O, C-NH$_2$ bonds.}
\label{Figures/chap5fig:keratin}
\end{figure}

\begin{sidewaystable}
\centering
\caption{Characteristics of commonly used rechargeable batteries.} \label{table2xps}
\begin{tabular}{ |p{2.5cm}|p{2cm}|p{2cm}|p{2cm}|p{1.5cm}|p{2.5cm}|p{2.5cm}|p{2.5cm}|}
\hline
\textbf{Active material} & \textbf{C-H/C-C} & \textbf{C-O/C-OH} & \textbf{C=O/O-C=O} & \textbf{C-F} & \textbf{Pyrrolic N / Pyridinic N} & \textbf{Aliphatic C-O} & \textbf{Aromatic C=O}\\
\hline
Human hair & 284.5 eV & 285.8 eV & 288.4 eV & 290.2 eV & 400.2 eV / 398.3 eV & 533.0 eV & 531.2 eV\\
CFEx & 284.2 eV & 286.0 eV & 288.2 eV & 290.0 eV & 399.3 eV & 531.3 eV & 530.2 eV\\
Hemp fibers & 284.5 eV & 286.0 eV & 288.7 eV & 290.5 eV & 400.3 eV & 532.9 eV & 531.4 eV\\
SPCB & 284.3 eV & 286.7 eV & 288.0 eV & 290.7 eV & 400.2 eV & 532.8 eV & ---\\
\hline
\end{tabular}
\end{sidewaystable}

\newpage

Functional groups that contain oxygen, such as carbonyl and ester groups, improve the wettability of a material. This increases the availability of active surface area as more electrolyte ions can now interact with the active material\cite{younesi_analysis_2015}. SPCB is produced from partial oxidation of petrochemical precursors \cite{gnanamuthu_electrochemical_2011}. A perfect graphite surface containing only carbon atoms, without heteroatoms like oxygen and sulfur, would give a very well-ordered structure. Presence of impurities such as carbonyl groups creates defects resulting in a less graphitic and more amorphous structure\cite{hao_carbonaceous_2013} and peaks at 288.0 eV for -CO bonds in AC and SPCB cathodes confirm that. The presence of these defects were also observed in the form of D-bands in their Raman spectra (Figure \ref{Figures/chap5fig:raman}. However, spectra for C 1s orbital of CFEx was uniquely different than the others due to presence of several highly symmetrical peaks. The presence of $\pi$ electrons on its surface resulted in  multiple $\pi$ satellite peaks, which are typical in a \ce{C60} molecule \cite{skryleva_xps_2016}. These peaks appear in both high (in green) and low energy ranges (in red) \cite{erbahar_spectromicroscopy_2016, poirier_carbon_1993}.   

\begin{figure}[h!]
  \centering
  \includegraphics[width=0.8\textwidth]{Figures/chap5fig/XPSON}
    \caption{XPS spectra of O 1s orbital for a) AC from human hair (ACH), b) hemp fibers c) CFEx and d) SPCB cathodes. Hair and hemp fibers contained significant amounts of aliphatic (red) and aromatic (pink) C=O groups  compared to CFEx and SPCB. Binding energies for N 1s orbital of e) hair, f) hemp fibers g) CFEx and h) SPCB cathodes. Human hair displayed distinct binding energies for pyridinic and pyrrolic N-species; hemp fibers, CFEx and SPCB had smaller amounts of surface proteins.}
  \label{Figures/chap5fig:XPSON}
\end{figure}

Figure \ref{Figures/chap5fig:XPSON}a-d shows various binding energies for O 1s orbital. In addition to enhancing the wettability of a material \cite{li_effect_2011, oh_oxygen_2014}. Oxygen-containing surface functional groups can provide pseudo-capacitance by reacting with electrolyte ions and make redox reactions feasible \cite{bleda-martinez_role_2005}. This might be another reason why aluminium-hair batteries performed better than the others. The active material was more exposed to the chloroaluminates than the others, due to its enhanced wettability. Figure \ref{Figures/chap5fig:XPSoverall} displays the overall spectra of all tested cathodes. Table \ref{table2xps} compiles all the functional groups with their respective binding energies for all tested cathode materials in this project. \\*

\begin{figure}[h!]
\centering
\includegraphics[width=\textwidth]{Figures/chap5fig/XPSoverall}
\caption{Overall spectra of human hair (black), hemp fibers (blue), CFEx (green) and Super-P(red).}
\label{Figures/chap5fig:XPSoverall}
\end{figure}

 \begin{figure}[h!]
  \centering
  \includegraphics[width=\textwidth]{Figures/chap5fig/CV}
    \caption{Cyclic voltammograms of a) CFEx, b) hair, c) Super-P and d) hemp fibers cathodes at a scan rate of 10 mV s$^{-1}$ against \ce{Al3+}/Al as a counter/reference electrode in a two-electrode setup. ACH cathode observed a larger CV area than other cathodes, which comes from an additional pseudocapacitance, adding capacity to the system.}
  \label{Figures/chap5fig:CV}
\end{figure}
\section*{Cyclic voltammetry}
%Cyclic Voltammetry is a study of electrochemistry, which is the formation of compounds under certain potential (or voltage) that drives ions in the solution resulting from an electric field. You sweep a voltage (for example -0.6V to +0.6 V)at a scan rate (for example, 0.5V/sec) for a selected range, if the reduction potential exist within that range (for material that desired to be deposited) you see a peak. Basically you are depositing a material on the forward sweep and stripping (anodic peaks) of the same material on the reverse sweep in order to find out the potential at which you can deposit the desired material (Or you also find out what others species could form during this process). 
Lastly, to confirm whether AC derived from human hair is a psedocapacitive material, we compared cyclic voltammograms of all cathodes at a scan rate of 10 mV s$^{-1}$. Figure \ref{Figures/chap5fig:CV}a-e showed that human hair batteries demonstrated a more rectangular, capacitor-like CV curve. However, redox processes were noticeable at a scan rate of 10 mV s$^{-1}$ and tiny redox peaks were visible (Figure \ref{Figures/chap5fig:CV}b). At a higher scan rate (50 mV s$^{-1}$), the redox peaks disappeared and the material displayed an ideal capacitor-like CV curve, shown in Figure \ref{Figures/chap5fig:hair50mVs} \cite{guan_capacitive_2016, dupont_separating_2015}. This happens because at a higher scan rate, the diffusion layer grows very close to the electrode and as a result of which higher current is recorded. Although redox peaks were observed for CFEx and others, the measured current was very low. At such low currents, it becomes impossible to make conclusions about the reduction/ oxidation reactions taking place and the stability of the species resulting from the electron transfer.

\begin{figure}[h!]
\centering
\includegraphics[width=\textwidth]{Figures/chap5fig/hair50mVs}
\caption{Cyclic voltammogram of ACH at a scan rate of 50 mV/s in a two electrode setup against \ce{Al3+}/Al showing a capacitor-like behaviour with no visible oxidation-reduction peaks unlike Figure \ref{Figures/chap5fig:CV}b where we distinctly observed redox peaks.}
\label{Figures/chap5fig:hair50mVs}
\end{figure}

\begin{figure}[h!]
  \centering
  \includegraphics[width=\textwidth]{Figures/chap5fig/cfexachlong}
    \caption{Discharge capacities of a) ACH and b) CFEx cathodes at current rates of 50mAg$^{-1}$, 500mAg$^{-1}$, 1.0 Ag$^{-1}$ and 1.5 Ag$^{-1}$ along with their CEs. }
  \label{Figures/chap5fig:cfexachlong}
\end{figure}

\newpage
% Are the following the conclusions?
\section*{Conclusion}
At the end of this project, it was found that in CFEx \ce{AlCl4-} anions seeped in and out of the gaps in between the fullerenes changing its structure and slightly expanding the crystal lattice during charging(Figure \ref{Figures/chap5figs:allmech}a). Moreover, fullerenes maintained their structural integrity and CE throughout the cycles. Hemp fibers and Super-P on the other hand, have a highly amorphous structure, which degraded after every cycle, resulting in a low capacity value. Furthermore, activated carbon derived from human hair proved to be the best carbon-based cathode among all the tested materials in this work, with a specific capacity of 100 mAh g$^{-1}$ for 50 cycles. It displayed a potential of 1.9 V with a CE of $\sim$90$\%$. Intercalation and de-intercalation of \ce{AlCl4-} ions might have taken place in the very few graphitic layers, but most of the specific capacity of the cell came form the surface-based adsorption of ions.Figure \ref{Figures/chap5fig:cfexachlong} compares the 50th cycle measurement for Al/hair and Al/natural graphite cell. It not only displays a higher specific capacity than conventional graphite, but also a high battery voltage of 1.92 V with an energy density of 202 Wh kg$^{-1}$. The high battery performance can be attributed to the porosity of the material combined with high surface area, hetero-atom doping effects resulting in surface-based non-Faradaic electron transfer reactions. Hair based aluminium-ion batteries would not only be cheaper than state of the art, but would also make a bio-degradable battery! 
%\section*{Preface}
This chapter is about discovering a material which achieved one of the highest specific capacities for non-aqueous aluminium-ion batteries! Boron nitride was tested as a cathode material and it showed a very high discharge capacity (>250 mAh g$^{-1}$). However, it turned out that boric anhydride (\ce{B2O3}), which was an impurity in hexagonal boron nitride (hBN), was the active material. Pure \ce{B2O3} was tested as a cathode material, the battery produced a similar discharge capacity of >250 mAh g$^{-1}$.   

\begin{figure}[tbh!]
\centering
\includegraphics[width=\textwidth]{Figures/BOhBN/ah}
\end{figure}

\newpage
\chapter{Boron nitride/boron oxide as a cathode for rechargeable AIBs} 
\label{BOhBN} 

\section{Theory and background}

\begin{figure}[tbh!]
\centering
\includegraphics[width=\textwidth]{Figures/BOhBN/grpBNcomp}
\caption{Honeycomb lattice of a) natural graphite and b) hexagonal boron nitride. Both structures display an interlayer distance of 3.3\AA.}
\label{Figures/BOhBN:grpBNcomp}
\end{figure}

Graphite has been extensively used as a cathode in AIBs due to its high conductivity and its layered structure. Graphite and hexagonal boron nitride (hBN) are materials that possess a hexagonal lattice structure \cite{hod_graphite_2012}. hBN is also known as inorganic graphite. Where graphite has non-polar homonuclear C-C intralayer bonds, hBN on the other hand displays highly polar B-N bonds. In the bulk form, the two materials have different stacking modes. Furthermore, the static polarizabilities of the constituent atoms are significantly different, suggesting large differences in the dispersive component of the interlayer bonding\cite{song_large_2010, zeng_white_2010}. Despite these major differences, both materials present practically identical interlayer distances as shown in  Figure \ref{Figures/BOhBN:grpBNcomp}. Structurally, a single layer of hBN is very similar to a graphene sheet and has a hexagonal backbone where each couple of bonded carbon atoms is replaced by a boron nitride pair, making the two materials isoelectronic. Nevertheless, due to the electronegativity differences between the boron and the nitrogen atoms, the $\pi$ electrons tend to localize around the nitrogen atomic centers, thus making it an insulating material. The nature of bonding between nitrogen and boron differs from the carbon-carbon bonds found in graphite. hBN possesses coordinate bonds resulting from donation of \ce{e-} pair from nitrogen into empty p-orbital of a neighbouring B atom. Each N atom develops a partial positive charge and each B develops a partial negative charge. The partial ionic character of BN bonding makes it a semi-conductor as opposed to a conductor like graphite. hBN has been used in solid-state LIBs in various forms. When it is mixed with the electrolyte, it imparts exceptional thermal stability that allows high-rate operation of solid-state rechargeable LIBs at temperatures up to 175 $^{\circ}$C\cite{hyun_high}. When coated onto the surface of a poly(ethylene oxide) (PEO)-based electrolyte, hBN formed a robust interfacial layer to improve the chemical and mechanical stability of the PEO-based electrolyte, leading to the enhanced performance of solid-state Li metal batteries\cite{shen_chem}. Boron nitride nanotubes (BNNTs) were synthesized by Rahman \textit{et al.} and used for the modification of a polyolefin separator without blocking the porous channels of the conventional separator for \ce{Li+} ion diffusion. This improved the thermal stability up to 150 $^{\circ}$C \cite{rahamn_}.\\*
For the above mentioned advantages, hBN was considered as a cathode for AIBs. Despite the fact that hBN is an insulator, it was assumed that additives like conductive carbon (Super-P) would compensate for the loss of conductivity. In addition, hBN was a new material in the field of AIBs.\\* 

\section{Results and discussion}

\begin{figure}[tbh!]
\centering
\includegraphics[width=\textwidth]{Figures/BOhBN/hBNiniCDC}
\caption{a) Galvanostatic cycles of an Al/hBN, using hBN from the stores (with \ce{B2O3} as an impurity), at a current density of 50 mA g$^{-1}$ compared with natural graphite (inset). b) Capacity fading of Al/hBN cell recorded for 30 cycles at a current rate of 50 mA g$^{-1}$.}
\label{Figures/BOhBN:hBNiniCDC}
\end{figure}

\begin{figure}[tbh!]
\centering
\includegraphics[width=\textwidth]{Figures/BOhBN/BNNSCDCCE}
\caption{a) Performance of an aluminium-ion battery using pure hBN as cathode at a current rate of 50 mA g$^{-1}$. b) Discharge capacity drops down to 5 mAh g$^{-1}$ after 50 cycles. hBN displays a very low coulombic efficiency $\approx$ 55\%.}
\label{Figures/BOhBN:hBNCDCCE}
\end{figure}

For all the above mentioned reasons, we tested hBN as a cathode for AIBs. To save cost, an old bottle of hBN was retrieved from VUW's chemical stores. A cell was assembled and preliminary electrochemical tests were performed. Figure \ref{Figures/BOhBN:hBNiniCDC} displays the galvanostatic charge/discharge profile of an Al/hBN cell at a current density of 50 mA g$^{-1}$. hBN exhibited very high specific capacities reaching values as high as 270 mAh g$^{-1}$. The cell displayed a discharge potential of 0.6 V. Despite being not as conductive as graphite, the discharge capacity value was more than three times the capacity of graphite (inset, Figure \ref{Figures/BOhBN:hBNiniCDC}a). Repeated measurements of Al/hBN cells using hBN from the same old bottle of hBN gave similar results, Figure \ref{Figures/appendix:hBNrepeat}. However, it was noted that the specific capacity dropped after a few cycles. In Figure \ref{Figures/BOhBN:hBNiniCDC}b, it was observed that the capacity retention of hBN was very poor (decreased by 60\%) after 30 cycles. In expectation of better results, a new bottle of hBN from Sigma Aldrich (98\%, $\sim$1 $\mu$m in size) was purchased. New batch of cathodes were made and tested. Surprisingly, the discharge capacity obtained from the new cells was nowhere near the values achieved by the older hBN as displayed in Figure \ref{Figures/BOhBN:hBNCDCCE}. A number of batches were made from the new bottle, but none of them achieved capacities higher than 50 mAh g$^{-1}$ (Figure \ref{Figures/appendix:hBNmultiattempts}). It seemed that two completely different materials were being tested! Given the old nature of the hBN sample, it was important to investigate whether the material was actually hBN or had it changed to some other compound over time.\\*
Figure \ref{Figures/BOhBN:oldxps} displays the binding energy of 1s orbital of boron. Th experimental peak was split into two peaks after curve fitting. In addition, with the B-N bond, a new B-O bond with a binding energy at 193.5 eV was observed. This suggested a strong presence of a \ce{B2O3} in the old hBN sample.
\begin{figure}[tbh!]
\centering
\includegraphics[width=\textwidth]{Figures/BOhBN/oldxps}
\caption{X-ray photoelectron spectra an old hBN cathode.}
\label{Figures/BOhBN:oldxps}
\end{figure}

To conclude that hBN did not play any active role in the old sample cell, boron nitride nano sheets (BNNS) were synthesised and tested as cathodes for AIBs. 
\large{Boron nitride nano sheets (BNNS)}
As mentioned in previous chapters, nano-sized materials increase the contact area between an electrode and electrolyte. They provide short path lengths for both ion diffusion and electron transport, which improves the charge/ discharge rate. The high surface area of the material allows large volume expansion/ contraction associated with ion transport and prevents cathode pulverisation leading to longer cycle-life \cite{zhang_ultrathin_2015,cong_intrinsic_2015}. \\*
\Large{Synthesis of BNNS} \\*
Nano sheets of hexagonal boron nitride were synthesised via mechanical exfoliation. 250 mg of hBN was dispersed in 75 ml isopropanol (IPA) in a 100 ml round bottom flask (RBF). The mixture was heated at 500$^{\circ}$C for 24 hours and was magnetically stirred. To accelerate the dissolution of hBN into IPA, the RBF was then put in an ultrasonic bath for 20 hours. The solution was then left to stand for 2 days and the supernatant was removed in a centrifuge tube. It was centrifuged at a speed of 14000 rpm. The obtained precipitate was washed with acetone to remove residual IPA. The product was dried overnight at 60$^{\circ}$C. 

\begin{figure}[tbh!]
\centering
\includegraphics[width=\textwidth]{Figures/BOhBN/BNNSSEM}
\caption{SEM images of hexagonal boron nitride nano sheets.}
\label{Figures/BOhBN:BNNSSEM}
\end{figure}

\begin{figure}[tbh!]
\centering
\includegraphics[width=\textwidth]{Figures/BOhBN/BNNSCDCCE}
\caption{Galvanostatic charge/discharge profile of Al/BNNS cell at a current rate of 50 mA g$^{-1}$. The cell achieved 22 mA h g$^{-1}$ in its first cycle, which dropped down to 2 mAh g$^{-1}$ after 50 cycles. Coulombic efficiency was also very low $\sim$50\%. }
\label{Figures/BOhBN:BNNSCDCCE}
\end{figure}

The galvanostatic charge/discharge profile of Al/BNNS is displayed in Figure \ref{Figures/BOhBN:BNNSCDCCE}a and b. Capacity fade and low coulombic efficiencies similar to pure hBN was observed in BNNS as well. This experiment proved that hBN present in the old hBN/\ce{B2O3} mixture, did not have high enough capacity and \ce{B2O3} played a significant role in achieving capacity as high as 270 mAh g$^{-1}$. \\*

\begin{figure}[tbh!]
\centering
\includegraphics[width=\textwidth]{Figures/BOhBN/oldhBNXPS}
\caption{XPS spectra of a a) pristine, b) charged and c) discharges old hBN cathodes after 30 cycles.}
\label{Figures/BOhBN:oldhBNXPS}
\end{figure}

To examine the changes taking place in the old hBN cathode during charge/discharge cycles, \textit{ex-situ} X-ray photoelectron spectroscopy (XPS) was carried out. The high resolution spectra displaying the binding energies of B 1s and Cl 2p orbitals is shown in Figure \ref{Figures/BOhBN:oldhBNXPS}a-c). The spectra reveal the presence of B and Cl elements for a pristine (Figure \ref{Figures/BOhBN:oldhBNXPS}a) charged (Figure \ref{Figures/BOhBN:oldhBNXPS}b) and discharged (Figure \ref{Figures/BOhBN:oldhBNXPS}c) cathode. Since the pristine electrode was not in contact with the electrolyte, Cl 2p binding energy was absent and the spectra displayed binding energies for boron only. The presence of Cl in the charged and discharged cathodes was mainly derived from the expected intercalation of chloroaluminates into the hBN layers. It can been seen from Figure \ref{Figures/BOhBN:oldhBNXPS}a-c) that the binding energy of B 1s includes a pair of peaks at 193.5 and 192.0 eV before test, which are attributed to B-O (from \ce{B2O3}) and B-N bonds (from hBN) respectively. After charging to 2.35 V, the peak area of the B-O bond significantly decreases. B-N bond from hBN becomes much more dominant. Interestingly, after discharging to 0.3 V, the B-O bond completely disappears as shown in Figure \ref{Figures/BOhBN:oldhBNXPS}c. Binding energy at 191.2 eV was attributed to a B-N bond. Curve fitting for Cl 2p orbital in the charged cathode was problematic. Two broad peaks were fitted into the one broad experimental peak. However, the peak was deconvoluted into 4 peaks at 202.9, 201.3, 199.5 and 197.9 eV after complete discharge. The binding energies (charged cathode) at 193.2 and 191.8 eV are again attributed to B-O from \ce{B2O3} and B-N from hBN respectively. After complete discharge, the B-O bond's binding energy is completely eliminated and only B-N bond remains.

\begin{figure}[tbh!]
\centering
\includegraphics[width=\textwidth]{Figures/BOhBN/hBNAlXPS}
\caption{Honeycomb lattice of a) natural graphite and b) hexagonal boron nitride. Both structures display an interlayer distance of 3.3\AA.}
\label{Figures/BOhBN:hBNAlXPS}
\end{figure}

XPS spectra of Al 2p orbital showed a variation in its binding energies during charge and discharge. During cell charging (Figure \ref{Figures/BOhBN:hBNAlXPS}b), the Al 2p peak deconvolutes into two binding energies at 76.0 and 77.3 eV. The peak at 76.0 eV was attributed to an Al-Cl bond from the chloroaluminates (\ce{AlxCly}). After complete discharge (Figure \ref{Figures/BOhBN:hBNAlXPS}c), the peak shifts to lower binding energies. Experimental curve fitting suggested two binding energies at 75.0 and 75.8 eV. The binding energy at 75.0 eV was attributed to an Al-O bond. It was noted that the binding energy of Al 2p in the oxidised state is 75 eV, which is very close to a typical Al-O bond in \ce{Al2O3} at 74.5 eV \cite{}. This indicated formation of \ce{Al2O3} during discharge. The binding energy at 75.8 eV corresponds to chloroaluminates. It is known that if the electronegativity of the doping element is higher than Al, the electron density around Al decreases and its binding energy increases. Chlorine has a higher electronegativity than oxygen, therefore the peak at higher binding energy in Figure \ref{Figures/BOhBN:hBNAlXPS}c is attributed to \ce{AlxCly} and the lower energy corresponds to \ce{Al2O3}. Since the pristine cathode dis not come in contact with the electrolyte, no peak was observed in Figure \ref{Figures/BOhBN:hBNAlXPS}a for Al. \\*

\begin{figure}[tbh!]
\centering
\includegraphics[width=\textwidth]{Figures/BOhBN/hBNOXPS}
\caption{Honeycomb lattice of a) natural graphite and b) hexagonal boron nitride. Both structures display an interlayer distance of 3.3\AA.}
\label{Figures/BOhBN:hBNOXPS}
\end{figure}

Figure \ref{Figures/BOhBN:hBNOXPS}a-c shows the high resolution O 1s spectra of the pristine (Figure \ref{Figures/BOhBN:hBNOXPS}a), charged (Figure \ref{Figures/BOhBN:hBNOXPS}b) and discharged (Figure \ref{Figures/BOhBN:hBNOXPS}c) cathodes made from the old hBN sample. The O 1s core level spectrum for pristine hBN (Figure \ref{Figures/BOhBN:hBNOXPS}a) was split into two peaks that corresponded to an O-H bond at 535.3 eV. This binding energy was attributed to adsorbed moisture (\ce{H2O}) from the environment. The peak at 534.1 eV was attributed to an O-B bond, which confirmed the presence of \ce{B2O3} in the old hBN sample. In the charged cathode, the major contribution of oxygen comes from \ce{B2O3} at 533.6 eV and the remaining from an O-Al bond with a peak at 531.7 eV, suggesting presence of \ce{Al2O3}.  Figure \ref{Figures/BOhBN:hBNOXPS}c shows the spectrum after complete discharge. The peak obtained was deconvoluted into three binding energies at 535.3, 532.9 and 531.6 eV, corresponding to an O-H bond from adsorbed moisture, O-Al and OH-Al bonds from \ce{Al2O3} and possible formation of \ce{Al(OH)3} respectively \cite{}. Following points were concluded from the XPS analysis:
\begin{itemize}
    \item \ce{B2O3} was present in significant amounts in the old hBN sample. However, it disappears completely after discharge. This might suggest the possibility of a conversion reaction where \ce{B2O3} is being converted into something else during discharge.  
    \item \ce{Al2O3} was formed after complete discharge. AL-Cl bonds were present in both charged and discharged electrodes. Even if deintercalation of ions took place during discharge, not all ions came out and a few \ce{AlCl4^-} ions remained in between the layers. However, an Al-Cl bond might also be a result of electrolyte residue on the cathode. 
\end{itemize}

It was assumed that \ce{B2O3} oxidised the \ce{AlCl4^-} ions to \ce{Al2O3} and itself reduced to elemental boron. However, during charge, elemental B was oxidised to \ce{B2O3} and \ce{Al2O3} was reduced back to \ce{AlCl4^-} ions. The conversion reaction in the form of an equation is mentioned below: 

\begin{equation}
    0.5\ce{B2O3 + AlCl3 + 3e-} \longrightarrow \ce{B + 0.5Al2O3 + 3Cl-}
\end{equation}

\begin{figure}[tbh!]
\centering
\includegraphics[width=\textwidth]{Figures/BOhBN/hBNXPS}
\caption{A wide scan spectrum of pristine (in black), discharged (in red) and charged (in green) hBN (old) cathode. Figure shows the survey spectra with peaks corresponding to aluminum and chlorine observed in the charged and discharged cathodes. Intensity of Al 2p and Cl 2p is higher in discharged cathode.}
\label{Figures/BOhBN:hBNXPS}
\end{figure}

It was important to find out if hBN played any role in this conversion reaction. hBN has a long-range order in its crystal lattice, while \ce{B2O3} has an amorphous structure. An XRD analysis of the old hBN sample would reveal if the layered structure of hBN allows any intercalation of chloroaluminates and undergoes structural changes during cycles. \\*


\begin{figure}[tbh!]
\centering
\includegraphics[width=\textwidth]{Figures/BOhBN/hBNXRD2}
\caption{.}
\label{Figures/BOhBN:hBNXRD2}
\end{figure}

In Figure \ref{Figures/BOhBN:hBNXRD2}, X-ray diffraction patterns of pristine (in black), charged (in green) and discharged (in red) cathodes of old hBN is shown. The patterns are in good agreement with the standard ICDD pattern (04-003-6253) and show the existence of hBN with P6/mmc space group. The miller indices (hkl) of all the characteristic peaks are marked as per the standard pattern. Peak at 2$\theta$ value of 26.5$^{\circ}$ confirms the d-spacing of 3.3\AA\, which matches well with the 002 plane of hBN. An additional shoulder that begins at 22$^{\circ}$ suggested presence of amorphous \ce{B2O3}. Interestingly, a new peak was observed at a 2$\theta$ value of 52.36$^{\circ}$. This peak matched with xxx plane of \ce{B2O3} (ICDD:). There was a slight shift observed for some peaks during the charge and discharge process. The peaks at 100 and 101 shift to lower 2$\theta$ values. This indicated an increase in the d-spacing value. The spacing increased from to 2.17 \AA to 2.22\AA for 100 and from 2.06\AA to 2.16\AA for 101 plane after cycles. In an ideal case, when ions intercalate during charging, the d-spacing increases and then when the ions deintercalate during discharging, the d-spacing returns back to its original value \cite{wang_advanced_2017}. In this case however, the d-spacing does not shift back to its original value after discharge. This suggests that the structural changes that take place during cycles in the hBN crystal lattice are permanent in nature. Since hBN is more crystalline in nature and therefore dominates the XRD data, it was difficult to study the patterns obtained from \ce{B2O3}.  \\*

\textit{Ex-situ} SEM experiments were also conducted to determine the structural changes taking place in the old hBN cathode. Figure \ref{Figures/BOhBN:hBNSEM} displays the SEM images of hBN cathode before and after cycles. Although a few sites showed agglomeration after cycles in Figure \ref{Figures/BOhBN:hBNSEM}b and d, it was difficult to make conclusions from these images.

\begin{figure}[tbh!]
\centering
\includegraphics[width=\textwidth]{Figures/BOhBN/hBNSEM}
\caption{SEM images of a), c) pristine hBN. The hexagonal shape of boron nitride is distinctly visible. Figure\ref{Figures/BOhBN:hBNSEM} b and d) suggest that after a few cycles, the particles agglomerate. The particles retain their distinct hexagonal shape.}
\label{Figures/BOhBN:hBNSEM}
\end{figure}



\section*{Pure boric anhydride \ce{B2O3} as an active material}
To find determine how \ce{B2O3} performed as a cathode material in absence of hBN, cells were assembled using \ce{B2O3} as the active material and the results are shown in Figure \ref{Figures/BOhBN:BOCDC}. 

\begin{figure}[tbh!]
\centering
\includegraphics[width=\textwidth]{Figures/BOhBN/BOCDC}
\caption{Charge/discharge profile of pure \ce{B2O3} as the active material in an AIB at the current rate of 50 mA g$^{-1}$. After 15 cycles, the capacity drops by $\sim$50\%. Coulombic efficiency of Al/\ce{B2O3} is low and  stabilises at $\sim$78\%.}
\label{Figures/BOhBN:BOCDC}
\end{figure}

\ce{B2O3} achieved a discharge capacity of 390 mAh g$^{-1}$ in its first cycle, which decreased to 200 mAh g$^{-1}$ in its second cycle and further dropped down to 90 mAh g$^{-1}$ after 15 cycles. Significant amount of capacity fading was observed, although coulombic efficiency of $\sim$ 78\% was maintained. It was possible that hBN provided structural support to \ce{B2O3}, therefore a combination of both was required to store high amounts of charge Figure \ref{Figures/appendix:hBNrepeat}. 

\begin{figure}[tbh!]
\centering
\includegraphics[width=\textwidth]{Figures/BOhBN/BonhBN}
\caption{Schematic showing the layered structure of hBN supporting \ce{B2O3}. During discharge \ce{AlCl3} reacts with \ce{B2O3} resulting in formation of elemental boron and \ce{Al2O3} and free \ce{Cl-} ions. Further analysis is needed to fully understand the role of hBN.}
\label{Figures/BOhBN:BohBN}
\end{figure}

The ratio of hBN  and \ce{B2O3} in the old sample was unknown. For this reason, different weight ratios of hBN and \ce{B2O3} were investigated as cathodes. The ratios and their performance is tabulated below in Table \ref{tabdiffpc}. All cells achieved discharge voltage plateaus at $\sim$ 0.6 V. 50\% hBN/50\%\ce{B2O3} was the best performing cathode amongst all tested ratios. The results for 1:1 hBN/\ce{B2O3} are displayed in Figure \ref{Figures/BOhBN:hBNBO5050}. Since a conversion reaction (Equation 6.1) was taking place during electron transfer, evaluating a combination of other nitrides and oxides was an important part of this chapter. As the optimum ratio obtained for hBN /\ce{B2O3} cathode was 1:1, all new nitrides and oxides were mixed in that same ratio. 

\begin{table}[tbh!]
\centering
\caption{Comparing performance of hBN/\ce{B2O3} cathodes at different weight percentages.} \label{tabdiffpc}
\begin{tabular}{|ccc|}
\hline
\textbf{Weight \% of} & \textbf{Weight \% of} & \textbf{Discharge capacity in 20th cycle} \\
\textbf{hBN} & \textbf{\ce{B2O3}} & \textbf{mAh g$^{-1}$} \\
\hline
\hline
50 & 50 & 120\\
25 & 75 & 22\\
20 & 80 & 48\\
15 & 85 & 48\\
10 & 90 & 68\\
5 & 95 & 171\\
0 & 100 & 104\\
\hline 
\end{tabular}
\end{table}

\begin{figure}[tbh!]
\centering
\includegraphics[width=\textwidth]{Figures/BOhBN/hBNBOdifpc}
\caption{Charge/discharge curves of aluminium-ion cells with \ce{B2O3}/hBN as cathode in their 20$^{th}$ cycle. The weight percentage varied from 75\%\ce{B2O3}-25\%hBN to 100\% pure\ce{B2O3}.}
\label{Figures/BOhBN:hBNdifpc}
\end{figure}



\begin{figure}[tbh!]
\centering
\includegraphics[width=\textwidth]{Figures/BOhBN/hBNBO5050}
\caption{Charge/discharge curves of hBN/ \ce{B2O3} cells a) for the first 20 cycles at a current rate of 50 mA g$^{-1}$, b) coulombic efficiencies and c) long term cell performance at various current densities ranging from 50 mA g$^{-1}$ to 1500 mA g$^{-1}$ .}
\label{Figures/BOhBN:hBNBO5050}
\end{figure}

Other nitrides such as graphitic carbon nitride g-\ce{C3N4}, aluminium nitride (AlN) and silicon nitride (\ce{Si3N4}) were mixed with \ce{B2O3} in a ratio of 1:1 and tested as cathodes. The results are displayed in Figure \ref{Figures/BOhBN:Bonit}. 
\begin{figure}[tbh!]
\centering
\includegraphics[width=\textwidth]{Figures/BOhBN/Bonit}
\caption{Galvanostatic charge/discharge cycles of cells at different current rates using \ce{B2O3} mixed with other nitrides such as a) \ce{C3N4}, c) AlN and e) \ce{Si3N4} as cathodes. Coulombic efficiencies of b) \ce{B2O3}/\ce{C3N4}, d) \ce{B2O3}/AlN and f) \ce{B2O3}/\ce{Si3N4} cells .}
\label{Figures/BOhBN:Bonit}
\end{figure}

Other oxides such as manganese dioxide (\ce{MnO2}) and titanium dioxide (\ce{TiO2}) were mixed with hBN in a ratio of 1:1 and tested as cathodes for AIBs. The results are displayed in Figure \ref{Figures/BOhBN:BNdifO}.

\begin{figure}[tbh!]
\centering
\includegraphics[width=\textwidth]{Figures/BOhBN/BNdifO}
\caption{Galvanostatic charge/discharge profile and cell efficiencies of AIBs composed of a-b) \ce{hBN}/\ce{MnO2} and c-d) \ce{TiO2}/hBN cathodes.}
\label{Figures/BOhBN:BNdifO}
\end{figure}

A few other combinations were tested as cathodes and the results are displayed in Figure \ref{Figures/BOhBN:othON}.

\begin{figure}[tbh!]
\centering
\includegraphics[width=\textwidth]{Figures/BOhBN/othON}
\caption{Galvanostatic charge/discharge profile and cell efficiencies of AIBs composed of a-b) \ce{MnO2}/\ce{C3N4} c-d) \ce{TiO2}/\ce{C3N4} and e-f) \ce{MnO2}/\ce{Si3N4} cells.}
\label{Figures/BOhBN:othON}
\end{figure}

The research findings from this chapter raises a lot of questions that need to be answered.
\begin{itemize}
    \item {\Large{Role of hBN.}} Is it only providing structural support to \ce{B2O3} or is it actually participating actively in the electron transfer process? If this is just a conversion reaction and hBN is not playing any significant role, why do other nitrides when combined with \ce{B2O3}, do not produce high discharge capacity $\sim$ 250 mAh g$^{-1}$?
    \item {\large{First principle studies of hBN/\ce{B2O3}.}} A conversion-type reaction has been hypothesised for this new material. However, it is important to study the changes taking places inside the cell \textit{in-situ} to fully determine the cell's mechanism. 
    \item {\Large{Using other nitrides and oxides.}} Once the role of nitrides and oxides in this mixture is established, can cheaper alternatives be used instead of hBN? On that note, why do the results in Table \ref{tabdiffpc} differ so much?
\end{itemize}


%% Chapter 6
\section*{Preface}
In this chapter, performance of AIBs using various kinds of materials was explored. Nanostructured molybdenum dichalcogenides (\ce{MoS2} and \ce{MoSe2}), graphitic carbon nitride (g-\ce{C3N4}), and electrospun tin oxide \ce{SnO2} fibers were tested as cathodes and their results have been reported. Prussian blue, which is a three-dimensional (3D) metal-organic framework (MOF), was also investigated as a cathode material. A comparison has been made between performance of bulk-\ce{MoSe2} and \ce{MoSe2} nanoflowers as cathodes in AIBs. 
\pagebreak
\chapter{Aluminium batteries using various 2D/3D materials as cathodes} 
\label{chap6} 
Recent development in the field of two-dimensional (2D) materials has shown a lot of potential in the field of energy storage. Materials like graphene and its analogues, have remarkable electrochemical properties. They can be used in most of the energy storage devices such as batteries, supercapacitors, redox flow batteries, photovoltaics etc. In addition to their tunable chemical and physical properties, 2D materials possess different crystallographic structures and elemental compositions. For this reason, they find immense use as electrode materials in electrochemical energy storage devices\cite{wang_graphene_2009,bonaccorso_graphene_2015}. Furthermore, graphene and other 2D nanomaterials have a larger theoretical gravimetric capacity \cite{zhou_progress_2014}. 

\begin{figure}[h!]
  \centering
  \includegraphics[width=\textwidth]{Figures/chap6fig/nanoTMDintro.pdf}
    \caption{An illustration of various applications of graphene and its analogues such as nanostructured \ce{MoS2}.}
  \label{Figures/chap6fig:nanoTMDintro}
\end{figure}

\section{Nanostructured molybdenum sulfide and selenide}
Some transition metal dichalcogenides (TMDs) such as \ce{TiS2} and exfoliated \ce{MoS2} flakes exhibit fast ionic conductivity as an electrochemically active material \cite{du_superior_2010,whittingham_electrical_1976}. A $\sim$750mAh g$^{-1}$ specific capacity was reported for restacked \ce{MoS2} single layers as a lithium-ion battery electrode. The restacking enlarged the c-axis parameter, which denotes its interlayer spacing, and consequently increased the accessible surface area \cite{ammundsen_novel_2001}.

Nanosized materials increase the contact area between an electrode and the electrolyte. They provide shorter path lengths for both ion diffusion and electron transport in comparison with bulk particles. As a result, the charge/ discharge rate is improved. Since a shorter path length for electronic transport is created, materials having low electronic conductivity can also be utilised \cite{pitchai_nanostructured_2011}. The high surface area of nanomaterials allows large volume expansion/ contraction associated with ion intercalation/ de-intercalation and prevents cathode pulverisation, which leads to a longer cycle-life \cite{zhang_ultrathin_2015, cong_intrinsic_2015}. TMDs have the potential to undergo redox reactions because of their multiple oxidation states. Nano TMDs have been widely explored as electrode materials since they have the potential to improve the electrochemical performance. 
These materials have shown some remarkable performances in LIBs, sodium-ion batteries (SIBs), lithium-sulphur (Li-S) batteries, magnesium-ion batteries (MIBs), etc \cite{xie_mos2/graphene_2015,cao_preparation_2013,dong_insights_2019,li_rechargeable_2018}. Nanostructured \ce{MoS2} has been established as a promising material to be used for energy storage. Xie and group synthesised \ce{MoS2}/ reduced graphene oxide (rGO) nanocomposites as an anode in SIBs \cite{xie_mos2/graphene_2015}. The \ce{MoS2} nanosheets were anchored onto the rGO substrate Using computational studies. The group established that \ce{Na+} ions preferred to be adsorbed on \ce{MoS2} rather that intercalate into the \ce{MoS2}/rGO network. It was shown that the conductivity of \ce{MoS2} increased due to presence of \ce{MoS2}/rGO \enquote{heterointerface} A heterointerface has been proven to introduce novel properties and new functionalities to the material. Due to formation of this interface, the nanocomposite stored more \ce{Na+} ions resulting in high-performing SIBs. The cell displayed excellent reversible capacity of $\sim$800 mAh g$^{-1}$ in its second cycle. Another interesting observation was that the capacity retention was determined by the amount of \ce{MoS2} in the composite. The composite with highest amount of \ce{MoS2} (more heterointerfacial area) had higher \ce{Na+} ion storage capacity. However, the drawback of these cells was the volume change during charge/discharge cycles. The capacity retention for the nanocomposite with highest \ce{MoS2} ratio was the poorest after 100 cycles. This can be attributed to the insufficient amounts of rGO, which allowed significant volume change during the cycles and store more \ce{Na+} ions. In 2013, Cao \textit{et al.} used \ce{MoS2} coated 3D graphene network (\ce{MoS2}/3DGN) as a binder-free anode material in LIBs \cite{cao_preparation_2013}. The 3D graphene network improved the electrical contact between the current collector and deposited \ce{MoS2}. The cell displayed a discharge capacity of 466 mAh g$^{-1}$ after 10 cycles at a  high current density of 4 A g$^{-1}$. The lithiation process was accompanied by a phase transformation of \ce{LixMoS2} from 2H to 1T during first discharge cycle. To determine the significance of \ce{MoS2} nanoparticles, another composite using bulk-\ce{MoS2} was prepared. It was observed that due to poor electrical contact between \ce{MoS2} and 3D-GN, the cell underwent rapid capacity decay. This confirmed that the graphene network provided an efficient pathway for \ce{Li+} exchange.It also showed that the nanostructured \ce{MoS2} had more active edge sites than bulk \ce{MOS2}, which resulted in high-performing LIBs. Based on similar principle, Dong \textit{et al.} used \ce{MoS2} nanosheets as an anode material for potassium-ion batteries. They suggested that highly crystalline-\ce{MoS2} displayed higher discharge capacity that poorly-crystallised \ce{MoS2} \cite{dong_insights_2019}. \ce{MoS2} microspheres were used as cathodes in non-aqueous AIBs by Li \textit{et al} \cite{li_rechargeable_2018}. The cell achieved a capacity of 66.7 mAh g$^{-1}$ after 100 cycles at a current rate of 40 mA g$^{-1}$. They found that intercalation of \ce{Al^{3+}} ions into \ce{MoS2} was accompanied by a phase transformation at the electrode interface. It was suggested that formation of a solid electrolyte interface (SEI) layer changed the microstructure of \ce{MoS2}. Furthermore, the cell failed to deliver a stable performance due to electrochemical polarization of \ce{MoS2}. Irreversible capacity losses and low CEs were also reported for this battery. With the rapid progress in research on 2D nanomaterials, large-scale preparation of nanostructured materials at a low cost can be expected for practical applications in the near future. In this chapter \ce{MoS2} and \ce{MoSe2} nanoflowers were used as cathodes in non-aqueous AIBs. The cells displayed a capacity of X at a current rate of Y after Z cycles. The performance was significantly better that previously reported bi Li and his group \cite{li_rechargeable_2018}. \\*

\begin{sidewaystable}
\centering
\caption{Summary of performances of 2D materials in various energy storage devices. Mixing transition metal oxides with nano forms of carbon improves the electrochemical performance of the material.} \label{table1}
\begin{tabular}{ |p{1.5cm}|p{3.5cm}|p{4.5cm}|p{4.5cm}|p{4.5cm}|}
 \hline 
\textbf{Ref.} & \textbf{Electrode} & \textbf{Electrolyte} & \textbf{Storage capacity} & \textbf{Cycle performance} \\ 
\hline
\cite{acerce_metallic_2015-1} & {Supercapacitors: Metallic 1T \ce{MoS2}} & 0.5M \ce{H2SO4}, \ce{Li2SO4}, \ce{K2SO4} or KCl or EMIm \ce{BF4} & Volumetric capacitance: 400-700 F cm$^{-3}$ & >90\% retained after 5000 cycles\\
\cite{zhao_flexible_2015} & Supercapacitors: MXene/CNTs & 1M \ce{MgSO4} & Volumetric capacitance: 350F cm$^{-1}$ & No degradation after 10000 cycles at 10 A g$^{-1}$\\
\cite{hu_hierarchical_2015} & LIB: \ce{TiO2} & 1M \ce{LiPF6} in a 1: (v:v) mixture of ethylene carbonate and dimethyl carbonate & Specific capacity: 182 mAh g$^{-1}$ at 5C & 87.9\% retained after 400 cycles at 5C \\
\cite{cao_preparation_2013} & LIB: \ce{MoS2}/graphene & 1M \ce{LiPF6} in a 1:1 (V:v) mix of ethylene carbonate and dimethyl carbonate & Specific capacity: 466 mAh g$^{-1}$ at 4 A g$^{-1}$ & 566 mAh g$^{-1}$ retained after 50 cycles at 0.5 A g$^{-1}$ \\
\cite{ding_facile_2012} & LIB: \ce{TiO2}/CNT \ce{SnO2}/CNT & 1M \ce{LiPF6} in a 1:1 (w:w) mix of ethylene carbonate and dimethyl carbonate & Specific capacity: 320 mAh g$^{-1}$ for \ce{TiO2}, 580 mAh g$^{-1}$ for \ce{SnO2} at 0.4 A g$^{-1}$ & 93.8\% retained after 120 cycles for \ce{TiO2}, 72.4\% retained after 40 cycles at 0.4 A g$^{-1}$ for \ce{SnO2}\\
\cite{xie_mos2/graphene_2015} & SIB: \ce{MoS2}/rGO & 1M \ce{NaClO4} in a 1:1 (V:v) mix of ethylene carbonate and dimethyl carbonate & Specific capacity: 350 mAh g$^{-1}$ at 0.64 A g$^{-1}$ & 227 mAh g$^{-1}$ retained after 300 cycles at 0.32 A g$^{-1}$ \\
\hline
\end{tabular}
\end{sidewaystable}

\subsection{Experimental methods}
The materials were obtained from Tohoku university and used as received. 
%Before adding sulfur, pristine \ce{MoO3} (Wako chemicals) was ball-milled for 4 hours. 1 mmol of ascorbic acid (Wako chemicals) as a reducing agent was dissolved in 5 ml of water and the mixture was magnetically stirred for at least 20 minutes under air. Subsequently, 1 mmol of S powder (Sigma-Aldrich) and 0.3 mmol of ball-milled \ce{MoO3} were placed in the reactor. Lastly, 5 ml of ascorbic acid aqueous solution was injected into the reactor vessels containing the powder mixture. The sealed reactor was kept at 400$^{\circ}$C in a tube furnace for 30 minutes. After heating, the samples were collected in the same procedure as above.

\subsection{Results and discussion}
\ce{MoS2} and \ce{MoSe2} nanoflowers achieved a discharge capacity of $\sim$120 mAh g$^{-1}$ and $\sim$60 mAh g$^{-1}$ at a current rate of 50 mA g$^{-1}$ after 5 and 10 cycles respectively. Figure \ref{Figures/chap6fig:mose2yncdcce} and Figure \ref{Figures/chap6fig:mos2yncdcce} shows the charge and discharge curves and CEs of the cells with cutoff voltages at 0.2 and 2.35 V. It was observed that the discharge curve for both \ce{MoSe2} and \ce{Mos2} looked similar to their bulk counterparts. \ce{MoSe2} displayed similar looking discharge voltage plateaus at 1.9 V. In addition, there was a visible voltage bend during discharge at 0.75 V. A plateau was formed during charge during charge at 2.0 V (similar to bulk \ce{MoSe2}). Furthermore, the CV curves displayed redox peaks at potentials that matched with the charge/discharge voltage plateaus confirming the presence of redox couples and possibility of intercalation taking place, Figure \ref{Figures/chap6fig:mox2yncv}a. Also, the discharge capacity decreased with every cycle. With an initial capacity of 90 mAh g$^{-1}$, the capacity decreased to 60 mAh g$^{-1}$ after only 10 cycles, which was different from what was observed with the bulk material. The reversible nature of charge storage, which was observed in bulk-\ce{MoSe2}, was missing. The CE of both materials was low at $\sim$40-50\%. This meant that the total charge extracted from the cell was more than the charge put into it over a full cycle. It also indicated presence of side reactions , which were unrelated to the normal redox processes occurring inside the cell. The behaviour of nano-\ce{MoS2} was similar to bulk-\ce{MoS2}. Voltage bends at 2.1 and 0.65 V were observed during discharge (similar to bulk-\ce{MoS2}. However, the cyclic voltammogram did not display any oxidation or reduction peaks, which were quite distinctly observed in bulk-\ce{Mos2} as seen in Figure \ref{Figures/chap6fig:mox2yncv}b. It was possible that the nano dimensions might not have allowed reversible intercalation/ de-intercalation. The schematic of a plausible mechanism is illustrated in Figure \ref{Figures/chap6fig:nanbulkmox2}. 

\begin{figure}[h!]
  \centering
  \includegraphics[width=\textwidth]{Figures/chap6fig/mose2yncdcce}
    \caption{Galvanostatic charge and discharge curves of a) Al/\ce{MoS2} and b) a graph plotting CE for every cycle.}
  \label{Figures/chap6fig:mose2yncdcce}
\end{figure}

\begin{figure}[th!]
\centering
\includegraphics[width=\textwidth]{Figures/chap6fig/mox2yncv}
\caption{Cyclic voltammograms of a)\ce{MoSe2} and b)\ce{MoS2} nanoflowers at the scan rate of 10 mV s$^{1}$.}
\label{Figures/chap6fig:mox2yncv}
\end{figure}

\begin{figure}[h!]
  \centering
  \includegraphics[width=\textwidth]{Figures/chap6fig/mos2yncdcce}
    \caption{Galvanostatic charge and discharge curves of a) Al/\ce{MoS2} and b) \ce{MoSe2} cell at various current densities. Long-term stability test of c) Al/\ce{MoS2} and d) \ce{MoSe2} cells. All capacity was recorded between charging and discharging voltages of 0.2 and 2.35 V.}
  \label{Figures/chap6fig:mos2yncdcce}
\end{figure}

\begin{figure}[h!]
  \centering
  \includegraphics[width=\textwidth]{Figures/chap6fig/nanbulkmox2.pdf}
    \caption{Schematic illustration of charge storage by \ce{MoX2} nanoflowers (left) and bulk material (right). The }
  \label{Figures/chap6fig:nanbulkmox2}
\end{figure}

The nanosized structure of \ce{MoX2} seems to obstruct a continuous intercalation process. Because of their high surface area, \ce{MoS2} nanoflowers follow a capacitor-like charge storage where \ce{AlCl4-} anions electrostatically get absorbed and desorbed at their surfaces after a few \ce{AlCl4-} ions intercalate in the first few cycles. It seems possible that a random arrangement of \ce{MoSe2} nanoflowers did not allow all the layers of the dichalcogenide to be accessible by the charge carrying species. For this reason \ce{MoSe2} exhibited a lower capacity than bulk and also displayed lower CEs. Further analysis is required to establish the above-mentioned hypothesis. 

\subsection{Future outlook}
The nanomaterials used in this chapter might behave differently if:
\begin{itemize}
    \item new nanocomposites are made using carbon-based materials (rGO, CNTs), which would not only improve the contact between the current collector and the active material, but also enhance the cell's capacity by phase transformation, which was missing in the nanoflowers
    \item increasing the number of layers present; the nanoflowers tested above have very few layers, which might have resulted in faster agglomeration of the active material resulting low cell efficiencies and low discharge capacities. The number of layers can be increased by modifying the synthesis procedure. 
\end{itemize}

\section{Tin oxide}

\subsection{Introduction}

\begin{figure}[th!]
  \centering
  \includegraphics[width=\textwidth]{Figures/chap6fig/SnO2crys}
    \caption{Crystal structure of \ce{SnO2}. a) Tetragonal unit cell with space group P4/nmm and space group number 129. b) Top view of the crystal lattice.}
  \label{Figures/chap6fig:SnO2crys}
  \end{figure}
  
Due to its high theoretical capacity ($\approx$ 782mAh g$^{-1}$) and safe handling, \ce{SnO2} has been a popular choice as anodes in LIBs  \cite{idota_tin-based_1997}. Unfortunately, the major disadvantage of these materials is the large volume change during lithium insertion/extraction. A volume change of $\sim$360\% in pure tin metal causes an internal strain. Whittingham \textit{et al.} showed that pure tin foil (bulk) can be cycled at 600 mAh g$^{-1}$ for 10 to 15 cycles \cite{yang_anodes_2003}. However, the expansion and contraction of the anode during the cycles causes anode pulverisation and increases of the cell impedance. Consequently, due to the loss of electronic contact between the active material and the current collector, the capacity of the cells decreases after 15 cycles. In some cases, formation of \ce{Li2O} in addition to volume expansion, further deteriorates the battery performance \cite{zhao_tin-based_2016}. Equation 1 describes the formation of \ce{Li2O} and Equation 2 describes the large volume variation that takes place due to formation of \ce{LixSn} after Li atom reacts with Sn produced in Equation 1\cite{park_effect_2008}.

\begin{center}
\ce{SnO2} + 4\ce{Li+} + 4\ce{e-} $\longrightarrow$ 2\ce{Li2O} + \ce{Sn} (1) 
\end{center}
\begin{center}
x\ce{Li} + x\ce{e-} + y\ce{Sn} $\longrightarrow$ \ce{Li_{x}Sn}  (2)
\end{center}

Since \ce{Li2O} is electrochemically inactive and non-conductive, it is also responsible for the large initial irreversible capacity. It has been reported that \ce{Li2O} can be decomposed via structural modifications of \ce{SnO2} on a nanoscale. Tin-based anodes have demonstrated improved electrochemical performance and cycle life and a controlled expansion process during lithiation.  

\begin{figure}[th!]
\centering
\includegraphics[width=\textwidth]{Figures/chap6fig/SnO2SEM}
\caption{SEM images of a,c)pristine and b,d) cycled \ce{SnO2} cathode.}
\label{Figures/chap6fig:SnO2SEM}
\end{figure}

It has been proposed that adding carbonaceous materials to \ce{SnO2} increases its surface area, which makes more active sites available for lithiation \cite{navarrosuarez_2d_2018}. It also controls the volume expansion/shrinkage. Furthermore, it improves the conductivity of the material \cite{nowak_composites_2018}. Several nanostructured tin-based materials such as nanorods \cite{liu_direct_2009}, nanobelts \cite{duan_single_2005}, nanowires \cite{huang_situ_2010}, nanotubes \cite{wang_large-scale_2011} have been synthesised and tested as electrodes. In this chapter, \ce{SnO2} fibers were obtained via  \enquote{electrospinning}. 

\begin{figure}[th!]
\centering
\includegraphics[width=\textwidth]{Figures/chap6fig/electrospinning}
\caption{A schematic illustration of an electrospinning setup.}
\label{Figures/chap6fig:electrospinning}
\end{figure}

\subsection{Experimental methods}
Electrospun \ce{SnO2} fibers were obtained from University of Montpellier, France and was used as received. \\
\textbf{Electrospinning} is an electrostatic fiber fabrication technique. It uses electrical forces to produce polymer fibers with diameters ranging from 2nm to a few $\mu$m. The techniques utilizes solution of both natural and synthetic polymers \cite{bhardwaj_electrospinning_2010}. The fibers obtained via electrospinning have smaller pores and a higher surface area than regular fibers produced by standard mechanical fiber-spinning technology \cite{huang_review_2003}. An electrospinning setup is illustrated in Figure \ref{Figures/chap6fig:electrospinning}. The topography and orientation of the fibers can be controlled by modifying parameters such as voltage, pump speed or nozzle thickness in the electrospinning setup.


\subsection{Results and discussion}
To evaluate the electrochemical properties, the Al/\ce{SnO2} cell was charged and discharged galvanostatically between 0.2-2.35 V at current densities ranging from 50-1500 mA g$^{-1}$. Figure \ref{Figures/chap6fig:SnO2newCDC} displays the voltage vs. specific capacity plot of the AIB. The curves demonstrate a well defined discharge plateau at $\sim$ 0.55 V. In the first cycle, the battery achieved a capacity of 105 mAh g$^{-1}$, which decreased to 50 mAh g$^{-1}$ after 120 cycles. CE of the cell decreased with every cycle and reduced to <60\% after 500 cycles. The discharge capacity decreased after every cycle and reached a value of $\sim$20 mAh g$^{-1}$ after 500 cycles, it might be possible that electrospun fibers suffered repeated expansion and contraction during the cycles, which caused cathode pulverisation. As a result \ce{SnO2} failed to deliver a stable performance. Distinct voltage plateaus indicate reversible redox processes taking place during charge and discharge. To confirm the redox activity, cyclic voltammetry was performed and the resultant scans are displayed in Figure \ref{Figures/chap6fig:Sno2CV}. The CV exhibits a reduction peak at 0.45 V and an oxidation peak at 0.55 V, which match perfectly with the discharging voltage plateau observed at 0.47 V and charging voltage plateau at 0.55 V respectively.    

 \begin{figure}[th!]
  \centering
  \includegraphics[width=\textwidth]{Figures/chap6fig/SnO2newCDC}
    \caption{a) Galvanostatic charge and discharge curve of an Al/\ce{SnO2} cell at the current rate of 40 mA g$^{-1}$. b) Long-term stability test of the cell.}
  \label{Figures/chap6fig:SnO2newCDC}
\end{figure}

\begin{figure}[th!]
\centering
\includegraphics[width=0.75\textwidth]{Figures/chap6fig/sno2pap.pdf}
\caption{The cyclic performance of \ce{SnO2} nanomaterials in lithium-ion batteries up to the fiftieth cycle at a current density of 100 mA g$^{-1}$. Capacity of the materials decreases with every cycle due to expansion of \ce{SnO2}, which leads to cathode pulverisation and capacity fading \cite{park_effect_2008}.}
\label{Figures/chap6fig:sno2pap}
\end{figure}

However, the X-ray diffraction patterns in Figure \ref{Figures/chap6fig:SnO2XRD} look alike after charge and discharge except a shoulder that develops in the charged and discharged cathodes at 2$\theta$ value of 22$^{\circ}$. Sharp Bragg peaks correspond to the crystalline phase and the broad bump under the peaks at 22$^{\circ}$ and  34$^{\circ}$ observed in the charged cathode corresponds to the amorphous state of the same material. Since the X-rays were scattered in many directions leading to a large bump distributed in a wide range. Furthermore, the peak at  51$^{\circ}$ splits after charge. The peak splitting might suggest formation of another crystal structure or presence of secondary phase that was formed during charge. SEM images in Figure \ref{Figures/chap6fig:SnO2SEM} display the cathode morphology before and after cycles where a few fibers show signs of damage after charge/discharge in Figure \ref{Figures/chap6fig:SnO2SEM}c. 

 \begin{figure}[th!]
  \centering
  \includegraphics[width=0.8\textwidth]{Figures/chap6fig/SnO2XRD}
    \caption{\textit{Ex-situ} X-ray diffraction patterns of \ce{SnO2} cathode in a pristine (black), charged (green) and discharged (red) state.}
  \label{Figures/chap6fig:SnO2XRD}
\end{figure}

 \begin{figure}[th!]
  \centering
  \includegraphics[width=0.8\textwidth]{Figures/chap6fig/Sno2CV}
    \caption{\textit{Ex-situ} X-ray diffraction patterns of \ce{SnO2} cathode in a pristine (black), charged (green) and discharged (red) state.}
  \label{Figures/chap6fig:Sno2CV}
\end{figure}

Further analysis such as \textit{in-situ} XRD or an \textit{ex-situ} XPS analysis of the charged and discharged cathodes is needed to investigate whether \ce{SnO2} follows a similar type of charge storage in AIBs as it does in LIBs. It would be interesting to find out whether new complexes are being formed after \ce{AlCl4-} anions interact with \ce{SnO2} during charge/discharge cycles. 


\section{Molybdenum trioxide}

\subsection{Introduction}

 \begin{figure}[th!]
  \centering
  \includegraphics[width=\textwidth]{Figures/chap6fig/MoO3crys}
    \caption{Crystal structure of \ce{MoO3}. a) Tetragonal unit cell with space group P4/nmm and space group number 129. b) Top view of the crystal lattice.}
  \label{Figures/chap6fig:MoO3crys}
\end{figure}

Molybdenum trioxide, \ce{MoO3} is an intermediate formed during production of molybdenum metal. It has a layered orthorhombic arrangement with a space group of Pnma, and contains four formula units of \ce{MoO3} per unit cell. The single sheet adopts a bilayer structure with both sides of the surface terminated with oxygen atoms. The crystal structure of \ce{MoO3} is illustrated in Figure \ref{Figures/chap6fig:MoO3crys}. Due to its layered structure, \ce{MoO3} has been popularly used as an electrode material in LIBs \cite{wu_mixed_2017,li_vapor-transportation_2006,tsumura_lithium_1997}. The intercalation of \ce{Li+} ions and the resulting redox reactions resulted in capacities ranging from 200-400 mAh g$^{-1}$ \cite{tsumura_lithium_1997,chen_fast_2010,zhou_-moo3_2010}. As one of the earliest studied host materials for \ce{Li+} insertion, $\alpha$ \ce{MoO3} can accommodate $\sim$ 1.5 lithium per Mo atom. Lithiated \ce{MoO3} (\ce{LixMoO3}) displayed good electronic conductivity and high \ce{Li+} mobility. It has been reported that the \ce{Li+} ions insert not only into the interlayer spacing between the \ce{MoO6} octahedron layers but also into the \ce{MoO6} intralayers \cite{li_vapor-transportation_2006,chen_fast_2010}. However, high concentrations of unsolvated \ce{Li+} in the host lattice sometimes causes irreversible structural changes resulting in poor cell performance \cite{tao_moo3_2011,li_theoretical_2014}.\\
The reaction that takes place inside a LIB during discharge is given below. 

\begin{center}
    \ce{xLi+} + \ce{MoO3} + \ce{xe-} $\longrightarrow$ \ce{LixMoO3} \cite{li_vapor-transportation_2006}
\end{center}

The given reaction is reversed during charge.

\ce{MoO3} has also been used as a cathode material in aqueous AIBs \cite{joseph_hexagonal_2019, shakir_structural_2010, lahan_al3+_2019, lahan_active_2018}. Lahan and Das \textit{et al.} reported that intercalation of \ce{Al^{3+}} cations was possible in orthorhombic \ce{MoO3}. However the performance was dependent on the electrolyte composition. They demonstrated that \ce{MoO3} stored more \ce{Al3+} ions when \ce{AlCl3} was used instead of \ce{Al2(SO4)_3} and Al\ce{(NO3)_3}. It also minimized the cell polarization and improved the long-term stability of the cell. \ce{MoO3} cell achieved a specific capacity of 680 mAh g$^{-1}$ (highest reported value for any aqueous AIB) after its first discharge cycle. \\ 
Since \ce{MoO3} was never tested before as a cathode in non-aqueous AIBs, it was used as an active material for this project. However, in 2018 (after the preliminary tests for this PhD thesis were completed), Nacimiento \textit{et al.} used layered-type $\alpha$-\ce{MoO3} as a cathode in non-aqueous AIBs using \ce{AlCl3}:EMIC in the ratio of 1.1:1.0 (slightly acidic melt) as the electrolyte. The maximum capacity achieved was 100 mAh g$^{-1}$, which decreased to 85 mAh g$^{-1}$ after 7 cycles. In addition, a rapid capacity decay was observed in Figure \ref{Figures/chap6fig:moo3pap}. After 11 cycles, the capacity decreased to 40 mAh $^{-1}$ at a low current rate of 3 mA g$^{-1}$. 

\begin{figure}[th!]
\centering
\includegraphics[width=0.5\textwidth]{Figures/chap6fig/moo3pap}
\caption{Galvanostatic experiments for MoO3 in aluminum cell. A) Voltage-capacity curves and, B) corresponding capacity as a function of cycle number at current density 3 mA g$^{-1}$, and C) capacity as a function of cycle number at 10 mA g$^{-1}$ for 0.1-2.1 V of voltage limits \cite{nacimiento_exploring_2018}}
\label{Figures/chap6fig:moo3pap}
\end{figure}

\subsection{Experimental methods}
Molybdenum trioxide (\ce{MoO3} ACS reagent, $\geq$99.5\% was purchased from Sigma-Aldrich and used as received.

\subsection{Results and discussion}
At a high current rate of 1500 mA g$^{-1}$ (Figure \ref{Figures/chap6fig:MoO3cdcce}a),\ce{MoO3} achieved the highest capacity of $\sim$80 mAh g$^{-1}$ with CE>100\%. After 120 cycles, the cell managed to retain 75\% of its original capacity. Voltage bends were observed during charge at 2.0 and 1.7 V for the first 100 cycles and a plateau was observed during discharge at 1.4 V. Discharge capacities at various current densities ranging from 50 mA g$^{-1}$ to 1500 mA g$^{-1}$ have been shown in Figure \ref{Figures/chap6fig:MoO3cdcce}. This cell performed better than Nacimiento's cell displaying higher discharge capacities and displayed better capacity retention during discharge.  

\begin{figure}[th!]
\centering
\includegraphics[width=\textwidth]{Figures/chap6fig/MoO3cdcce}
\caption{Charge/discharge cycles of an Al/\ce{MoO3} cell at various current rates.}
\label{Figures/chap6fig:MoO3cdcce}
\end{figure}

\subsection{Summary and conclusions}
The electrochemical data suggests fast intercalation-based redox reactions similar to LIBs. With an energy density of $\sim$85 Wh kg$^{-1}$, this material shows a lot of potential for high-performing AIBs. 


\section{Graphitic Carbon Nitride}

\subsection{Introduction}
Graphitic carbon nitride or g-\ce{C3N4} is a carbon-based material that was first synthesised in 2012. The material is highly stable under physiological conditions and displays semiconductive properties. The crystal structure can be regarded as N-substituted graphite framework consisting of $\pi$-conjugated graphitic planes formed via sp$^2$ hybridisation of C and N atoms. The inter-layer distance between two stacks is 3.26\AA, and is 2\% more densely packed than crystalline graphite \cite{zheng_graphitic_2012}. Due to N-atom substitution, the binding between two layers is strengthened, which decreases its inter-layer distance. g-\ce{C3N4} is a low-cost metal free material with a high surface area \cite{zheng_graphitic_2012}. Since the structure of g-\ce{C3N4} is analogous to graphite, it has been used for electrochemical energy storage applications. \\*
g-\ce{C3N4} has been used as a battery electrode material in LIBs, Li-\ce{O2}, Li-S, Zn-air, vanadium redox flow batteries and SIBs. The presence of \enquote{pyridinic} N in g-\ce{C3N4} favors high \ce{Li+} intake and prohibits irreversible reactions. Therefore, nitrogen content and its type, highly influence the performance of g-\ce{C3N4} in LIBs \cite{shah_highly_2017}. \\*
Vanadium redox flow batteries on the other hand, have also used g-\ce{C3N4} as a catalyst. A typical vanadium redox flow battery consists of two electrolyte tanks with \ce{VO2+}/\ce{VO^{2+}} and \ce{V3+}/\ce{V2+} redox couples, two pumps and a battery cell. The electrochemical reactions take place at the electrode. For this reason, highly efficient catalysts are needed. Huang \ce{et al.} modified carbon felt (usually used as a catalyst in flow batteries) with g-\ce{C3N4} for catalyzing the redox reactions. This increased the energy efficiency (EE) of the cells to 87\%. In addition, Nafion membranes were replaced by g-\ce{C3N4} hybrids, such as sulfonated poly(ether ether ketone) SPEEK/g-\ce{C3N4} and oxidised g-\ce{C3N4} (OCN) ion membranes \cite{niu_novel_2017, wang_novel_2017}. An ion membrane separates the ion-pairs in the flow battery and accelerates the proton flow during charge/ discharge cycles. The g-\ce{C3N4} modified membranes not only improved the EE of the flow batteries but also increased its efficiency (CE: 97\% and EE: 83.6\%) when compared to Nafion (CE: 90\% and EE: 73.8\%). A good structure stability against strong oxidizing and acidic condition was reported. The acid-base pairs formed between -\ce{NH2} groups of g-\ce{C3N4} and sulfonic acid groups of SPEEK enhanced the vanadium ion permeability and its selectivity, and also improved the proton transport channel \cite{wang_novel_2017}. SPEEK/OCN membranes due to their high surface area and intrinsic stability of OCN contributed to decrease the vanadium ion permeability. The functional groups of OCN helped in improving the proton conductivity and improved the battery's performance. Due to all of the above-mentioned properties such as high surface area and structural stability, g-\ce{C3N4} was tested as a cathode material for non-aqueous AIBs. It was speculated that the chloroaluminate ions would undergo a similar intercalation-type process that would be supported by the graphitic framework present in g-\ce{C3N4}.

\begin{figure}[th!]
\centering
\includegraphics[width=\textwidth]{Figures/chap6fig/C3N4crys}
\caption{.}
\label{Figures/chap6fig:C3N4crys}
\end{figure}
\subsection{Experimental methods}
g-\ce{C3N4} that was synthesised using urea and was obtained from the University of Auckland. The material was used as received for making the slurry. 

\subsection{Results and discussion}
Figure \ref{Figures/chap6fig:CNUcdccv}a and b shows the charge/discharge curves and CV scan of g-\ce{C3N4} cell. The discharge capacity decreased from 160 mAh g$^{-1}$ to $\sim$100 mAh g$^{-1}$ after 120 cycles at the current rate of 50 mA g$^{-1}$. Figure \ref{Figures/chap6fig:CNUcdccv}a shows the discharge capacities at current densities ranging from 50 mA g$^{-1}$ to 1500 mA g$^{-1}$. The cell managed to achieve a capacity of 95 mAh g$^{-1}$ at 500 mA g$^{-1}$. A distinct charging plateau at 2.1 V and a slight bend during discharge at 0.6 V suggests the presence of reversible redox reactions. However, the capacity decreased to almost zero at 1500 mA g$^{-1}$, which implies that the current was too high and the cell was not allowed to complete its reaction. 

\begin{figure}[th!]
\centering
\includegraphics[width=\textwidth]{Figures/chap6fig/CNUcdccv}
\caption{a) Charge/discharge cycles of an Al/\ce{MoO3} cell at various current rates. The cell managed to retain 67\% of its original capacity after 120 cycles. b) Cyclic voltammogram of an Al/g-\ce{C3N4} cell; a reduction peak at 1.6 V was observed, which does not correspond to any of the voltage bends or plateaus observed in the charge/ discharge curves. }
\label{Figures/chap6fig:CNUcdccv}
\end{figure}

The storage performance of g-\ce{C3N4} in other battery systems was largely affected due to its large contact resistance and low band-gap \cite{shah_highly_2017}. Considerable efforts were made to improve its performance. Despite the fact that g-\ce{C3N4} has a more open structure than graphite, which should improve its \ce{Li+} intake and exchange capability, Luo \cite{luo_graphitic_2019} \textit{et al.} showed that the LIB using g-\ce{C3N4} anode displayed a capacity of 134.9 mAh g$^{-1}$ and an irreversible capacity loss of >98\% after 7 cycles. Veith and Hankel \textit{et al.} reported that reactions of lithium with g-\ce{C3N4} was responsible for the irreversible capacity loss \cite{veith_electrochemical_2013, hankel_lithium_2015}. Furthermore, replacement of C atoms with N atoms in the benzene ring was also proven to be responsible for its poor conductivity. These reasons might be valid for an AIB system as well since the material failed to retain its capacity. Nevertheless, there is still scope for improvement as new functional groups can be added to g-\ce{C3N4} that might improve its charge-storing capacity. 

\section{Prussian blue}

\subsection{Introduction}

 \begin{figure}[th!]
  \centering
  \includegraphics[width=\textwidth]{Figures/chap6fig/pbcrys}
    \caption{Crystal structure of Prussian blue. The voids present in the framework (20.4\AA) are spacious enough to allow the chloroaluminate ions (5.36\AA) to move in and out of it during charge and discharge. }
  \label{Figures/chap6fig:pbcrys}
\end{figure}

Prussian blue belongs to the family of metal-organic framework, also known as MOF. A MOF is a hybrid crystalline porous material. It consists of a regular arrangement of positively charged metal ions and surrounded by organic molecules. A repeating cage-like structure is formed when metal cations form nodes that binds with an organic molecule's linker. MOFs have a large surface area (as high as 700 m$^{2}$g $^{-1}$) due to their hollow structure. Figure \ref{Figures/chap6fig:pbcrys} displays the crystal structure of Prussian Blue. The structural uniformity and and the flexibility in their network topology, makes MOF an ideal battery material. Its open framework allow insertion of ions in the sub-cages. The structure has a number of redox sites present as each molecular formula contains two redox centres \ce{M+2}/\ce{M3+}, where M is any transition metal (Fe, Co, Ni, Mn, Cu, Zn). It reaches 2\ce{e-} redox capacity after reversibly intercalating 2 monovalent alkali ions per molecular unit. Presence of large lattice interstices and ionic channels renders a high specific capacity. The ability to tune MOFs along with its structural design, makes it different from the conventional porous materials. Due to their high surface area and and tailored pore-size, MOFs have been utilised as electrode materials in electric double layer capacitors or EDLCs, LIBs and SIBs. A Co-Zn MOF was designed by Diaz \textit{et al.} and it showed a typical EDLC behaviour in a non-aqueous electrolyte \cite{diaz_co8-mof-5_2012}. However, the specific capacitance for Co-Zn MOF was low. It was observed that cobalt-based MOF derived from Co\ce{(NO3)_2}, displayed pseudocapacitance behaviour (presence of redox couples) instead of an EDLC behaviour with an improved specific capacitance. Lee \ce{et al.} observed a relationship between the pore size of MOFs and their performance in supercapacitors. After several attempts, it was determined that inclusion of conducting species within the MOF framework could solve the problem of low capacitance by providing an efficient electron pathway. Furthermore, MOFs with tailored channels or pores would allow faster diffusion of active ions during charge/ discharge cycles. The electrolyte also plays an important role in achieving high-performing supercapacitors. \\*
Recently, MOFs have been investigated as anode materials in LIBs \cite{li_shape-controlled_2006,han_synthesis_2012,zhao_metalorganic_2015}. To be used as anodes, the porous structure of MOFs allows reversible intercalation of \ce{Li+} ions during charge/discharge cycles. MOF-177 was the first MOF that was used as an anode material in LIBs \cite{}. The material underwent a conversion reaction inside the cell that destroyed its structure and consequently resulted in a poor cycling performance.  Zhang \textit{et al.} investigated Prussian blue nanoparticles as anodes \cite{nie_prussian_2014}. The open-framework structure allowed rapid intercalation/ de-intercalation of \ce{Li+} ions. The cell achieved a reversible capacity of 300 mAh g$^{-1}$ and superior rate capability. Yagi and his group studied the lithiation mechanism in Prussian blue and its analogues (PBAs) \cite{yagi_eqcm_2014}. They reported that the redox reaction of PB might also proceed with the electrochemical adsorption/ desorption of \ce{PF6-} ions, which are the counter ions in the electrolyte. MOFs have also been used as templates for preparing nanostructured metal oxides and and carbon materials. A few other MOFs used in LIBs as anode materials have been listed in Table \ref{tableMOF}. \ce{Li+} storage in a typical LIB takes place via the following mechanisms: \\

\textbf{Conversion reaction}- metal ions present in the MOF get replaced by \ce{Li+} ions \\
\textbf{Intercalation reaction}- \ce{Li+} ions get stored in the MOF's cage-like structure and keeps the structure intact \cite{wang_metalorganic_2016}. \\

MOFs have shown to have many applications in electrochemical energy storage. They can be used both in supercapacitors and in batteries. Based on the multiple valence of metal ions, open framework and ligands with functionalities, novel electrode materials can be designed from MOFs and used in AIBs. For this reason, Prussian blue, a popular MOF, was tested as a cathode material in this project. 

\vspace{0.5cm}
\begin{table}
\centering
\caption{Summary of performances of MOFs used as anode and cathode materials in various energy storage devices.} \label{tableMOF}
\begin{tabular}{ |p{1cm}|p{3.5cm}|p{2.2cm}|p{1.2cm}|p{1.5cm}|}
 \hline 
\textbf{Ref.} & \textbf{MOFs} & \textbf{Capacity (mAhg$^{-1}$)/ Capacitance (Fg$^{-1}$)} & \textbf{Voltage (V)} & \textbf{Surface area (m$^{2}$g$^{-1}$)} \\ 
\hline
\cite{li_shape-controlled_2006} & {MOF 177} & 425 & 0.1-1.6 & -\\
\cite{han_synthesis_2012} & Li/Ni-NTC & 1084 & 0.01-3 & -\\
\cite{zhao_metalorganic_2015} & Asp-Cu nanofibers & 1255 & 0.01-3 & -\\
\cite{wu_mof-templated_2013} & CuO & 1208 & 0.05-3 & -\\
\cite{huang_metal-organic_2014} & \ce{Fe2O3}/\ce{NiCo2O4} & 1311 & 0.01-3 & -\\
\cite{nagarathinam_redox-active_2012} & \ce{K2.5VO2}\ce{HPO4}\ce{C2O4} & 62 & 2.5-4.6 & -\\
\cite{zhang_monitoring_2014} & Cu(2,7-AQDC) & 147 & 1.7-4 & -\\
\cite{liu_metalorganic_2010} & NPC650 & 222 & - & 1521\\
\cite{hu_porous_2010} & MC-A & 208 & - & 1674\\
\cite{tang_thermal_2015} & NC/GC & 270 & - & 1276\\
\cite{chen_high-performance_2013} & N-PC & 219 & - & 484\\
\cite{banerjee_mof-derived_2014} & MOF-DC & 149 & - & 2714\\
\hline
\end{tabular}
\end{table}

\subsection{Results and discussion}
The first charge–discharge curve exhibited distinct voltage bends and plateaus at 1.8, 1.2 V and 0.6 V with a capacity reaching $\sim$140 mAh g$^{-1}$ after first 20 cycles. To investigate the rate capabilities of the battery, the cell was charged and discharged at various current densities (Figure \ref{Figures/chap6fig:pbcdccecv}a) ranging from 50-1500 mA g$^{-1}$. The specific capacities and CEs over 180 cycles at different current rates are shown in Figure \ref{Figures/chap6fig:pbcdccecv}b. Discharge capacity after the first cycle is 140 mAh g$^{-1}$, and it exhibits multiple discharge voltage plateaus. With an increase in current density, the capacity of the battery gradually decreased to 5 mAh g$^{-1}$. 

 \begin{figure}[tbh!]
  \centering
  \includegraphics[width=\textwidth]{Figures/chap6fig/pbcdccecv}
    \caption{Galvanostatic cycle test of an Al/\ce{C19Fe7N18}, Prussian blue, cell in a two-electrode setup at various current rates. b) Long-term cell stability test at various current rates ranging from 50 to 1500 mA g$^{-1}$. c) CV curve of an Al/PB cell at a scan rate of 10 mV s$^{-1}$. The plateaus observed during charge at 0.6 and 1.8 V are almost identical to the oxidation peaks at 0.65 and 1.8 V. Voltage bend at 1.8 V during discharge corresponds to the sharp reduction peak at 1.8 V. Another oxidation peak at 2.15 V with a shoulder at 2.0 V }
  \label{Figures/chap6fig:pbcdccecv}
\end{figure}

\subsection{Summary}
In other battery systems, it had been reported that due to structure degradation, MOFs underwent large irreversible capacity losses and completed lesser number of cycles. This might be one of the reasons why the Al/MOF cell displayed a similar trend. However, Wang \ce{et al.} proposed a number of possible solutions to improve the performance of MOFs as a battery material. Supposing that the cell undergoes conversion-type reaction, the type of ligand in the MOF plays an important role because it determines if the MOF structure can be easily regenerated after every cycle. In case the cell undergoes an intercalation-type mechanism, MOF with a robust structure is highly desirable. A metal ion with multiple valence states and low molecular weight ligands rich in functional groups promotes insertion of \ce{Li+} ions \cite{wang_metalorganic_2016}. Therefore, altering the current MOF i.e. \ce{C19Fe7N18}, using the above-mentioned techniques would certainly improve its performance in AIBs. 

\section{Future developments}
\begin{itemize}
\item  In this chapter nanostructured cathode materials, owing to their special properties such as high surface area and ability to provide shorter ion diffusion path lengths, were tested as cathodes for non-aqueous AIBs. It was found that the performance of molybdenum dichalcogenide nanoflowers was not significantly different from the bulk. Similar battery voltages were observed for both bulk and nano-\ce{MoSe2}. Where bulk-\ce{MoSe2} displayed a capacity of $\sim$80mAh g$^{-1}$, \ce{MoSe2} nanoflowers displayed a capacity of <60 mAh g$^{-1}$ after 10 cycles with rapid capacity decay. Modifications can be made by increasing the number of layers present in nano-\ce{MoX2}, this might improve their performance and retain higher amounts of charge after every cycle.  
\item Another solution would be redesigning \ce{SnO2} fibers using materials that might make it more robust and eventually solve its problem of poor capacity retention. It is suggested that mixing the active material with carbon-based materials or making \ce{SnO2} hybrids (as suggested above for other battery systems) might improve the battery voltage. Like other battery systems, altering the structure of Prussian blue and g-\ce{C3N4} by adding suitable ligands will enhance the battery performance by achieving higher voltages and capacities. 
\end{itemize}

Supplemental research is required for an in-depth analysis of all the above-mentioned materials so that a comparative study can be made and a post-mortem analysis would also help in determining their individual mechanism. The current chapter makes way for a number of future projects in the field of AIBs, which will help in discovery of many more cathode materials for AIBs.  
%% Chapter 7

\chapter{Aluminium-ion batteries using different solvents while slurry preparation} % Main chapter title

\label{Chapter7} % For referencing the chapter elsewhere, use \ref{Chapter1} 

%----------------------------------------------------------------------------------------

% Define some commands to keep the formatting separated from the content 
\newcommand{\keyword}[1]{\textbf{#1}}
\newcommand{\tabhead}[1]{\textbf{#1}}
\newcommand{\code}[1]{\texttt{#1}}
\newcommand{\file}[1]{\texttt{\bfseries#1}}
\newcommand{\option}[1]{\texttt{\itshape#1}}

%----------------------------------------------------------------------------------------
\section{Theory and background}
%Commercial battery electrodes are manufactured by casting a slurry onto a metallic current collector. The slurry contains active material, conductive carbon, and binder in a solvent. The binder, most commonly polyvinylidene fluoride (PVDF), are pre-dissolved in the solvent, most commonly N-Methyl-2-pyrrolidone (NMP). During mixing, the polymer binder flows around and coat the active material and carbon particles1,2,3,4,5,6,7,8,9. After uniformly mixing, the resulting slurry is cast onto the current collector and must be dried. Evaporating the solvent to create a dry porous electrode is needed to fabricate the battery. Drying can take a wide range of time with some electrodes taking 12–24 hours at 120 °C to completely dry5,10. In commercial applications, an NMP recovery system must be in place during the drying process to recover evaporated NMP due to the high cost and potential pollution of NMP11,12. While the recovery system makes the entire process more economical it does require a large capital investment. Less expensive and environmentally friendly solvents, such as aqueous based slurries, could eliminate the large capital cost of the recovery system but the electrode would still require a time and energy demanding drying step9,10,13,14,15,16. Uncoventional manufacturing methods have also been used to create battery electrodes. Solvent based electrostatic spray deposition has been used to coat current collectors with electrode material17,18,19. This is achieved by adding high voltage to the deposition nozzle and grounding the current collector, which causes the deposition material to become atomized at the nozzle and drawn to the current collector. Electrodes constructed with this method exhibit similar characteristics as slurry-cast electrodes and have similar disadvantages in that they also require a time and energy intesive drying process (2 hours at 400 °C)19.
\section{Experimental methods}
\section{Results and discussions}
\section{Conclusion and future work}
%% Chapter 9
\section*{Preface}
This chapter summarises the research findings of chapters 4-8 and provides an outlook for future researches. Many new scientific findings have been made, which need to be studied and analysed in greater detail, so that aluminium-ion batteries can find commercial use.
\newpage
\chapter{Conclusion and future outlook} % Main chapter title
\label{chap9} % For referencing the chapter elsewhere, use \ref{Chapter1} 
\section*{Conclusion}
\section{Future outlook}

%	THESIS CONTENT - APPENDICES

\appendix % Cue to tell LaTeX that the following "chapters" are Appendices

% Include the appendices of the thesis as separate files from the Appendices folder
% Uncomment the lines as you write the Appendices

% Appendix Template
\chapter{Appendix} % Main appendix title
\label{appA} % Change X to a consecutive letter; for referencing this appendix elsewhere, use \ref{AppendixX}
\begin{figure}[tbh!]
\centering
\includegraphics[width=\textwidth]{Figures/appendix/blankmol}
\caption{a) Galvanostatic charge/discharge curves of an Al/Mo cell using a two-electrode setup at a current rate of 40 mA g$^{-1}$. The cell failed to achieve any significant discharge capacities during both charge and discharge. b) Blank Mo foil displayed CE at 40\%. This confirmed that when acting as the current collector, molybdenum did not contribute any capacity of its own.}
\label{Figures/appendix:blankmol}
\end{figure}
\begin{figure}[tbh!]
\centering
\includegraphics[width=\textwidth]{Figures/appendix/hBNrepeat}
\caption{Galvanostatic charge/discharge curves of Al/old h-BN cells using a two-electrode setup at a current rate of 40 mA g$^{-1}$. The cell achieved discharge capacities of 125 mAh g$^{-1}$ with plateau observed during discharge at 0.4 V. and a charging voltage at 1.0 V.}
\label{Figures/appendix:hBNrepeat}
\end{figure}
The figure below presents the data that was collected using \rq pouch cells\lq\ in Fraunhofer Institute, Dresden, Germany. A pouch cell contains conductive foil-tabs, made of nickel and aluminium, which were welded to both the electrodes and sealed completely inside a glove box. The pouch cell offers a simple, flexible and lightweight solution to battery design. The pouch cell makes most efficient use of space and achieves 90–95\% packaging efficiency, the highest among battery packs such as coin cells, Swagelok-type cells, cylindrical cells, etc.
\begin{figure}[tbh!]
\centering
\includegraphics[width=\textwidth]{Figures/appendix/pouchCE}
\caption{a) Coulombic efficiency recorded for 1600 cycles of Al/old-hBN pouch cells assembled in IKTS, Germany.}
\label{Figures/appendix:pouchCE}
\end{figure}

\begin{figure}[tbh!]
\centering
\includegraphics[width=\textwidth]{Figures/appendix/hBNmultiattempts}
\caption{CDCs of Al/hBN cells using pure hBN, 98\% pure, $\approx$1 $\mu$m in size using similar assembly conditions. All cells were run at a current rate of 40 mA g$^{-1}$. Despite repeated attempts, none of the cells recorded a capacity above 50 mAh g$^{-1}$. This was an issue because we were trying to replicate our previous results where an Al/hBN cells recorded capacities above 100 mAh g$^{-1}$, Figure \ref{Figures/appendix:hBNrepeat}.}
\label{Figures/appendix:hBNmultiattempts}
\end{figure}

\begin{figure}[tbh!]
\centering
\includegraphics[width=\textwidth]{Figures/appendix/pouchcellCDCCE}
\caption{a-c) Cell performance of various Al/hBN pouch cells assembled in ITKS, Germany. None of the cells managed to achieve capacities above 50 mAh g$^{-1}$. d) Coulombic efficiency of an Al/hBN cell run for thousand cycles at a current rate of 40 mA g$^{-1}$. }
\label{Figures/appendix:pouchcellCDCCE}
\end{figure}
\begin{figure}[tbh!]
\centering
\includegraphics[width=\textwidth]{Figures/appendix/WS2CDCCE}
\caption{ a) Galvanostatic charge/discharge cycles, and b) coulombic efficiency of an Al/\ce{WS2} cell for 10 cycles at a current rate of 50 mA g$^{-1}$. A distinct plateau was seen during discharge at 0.68 V and a charging plateau was observed at 1.0 V vs \ce{Al$^{3+}$}/Al. \ce{WS2} has a layered structure similar to \ce{MoS2} with an interlayer distance of 6.18 \AA. Looking at the CDCs, a few \ce{AlCl4-} ions manage to intercalate but it seems the side reactions taking place at the electrode/electrolyte interface prevent the cell from achieving high specific capacities.}
\label{Figures/appendix:WS2CDCCE}
\end{figure}
\begin{figure}[tbh!]
\centering
\includegraphics[width=\textwidth]{Figures/appendix/mox2sem}
\caption{SEM images of pristine a) \ce{MoS2} and b) \ce{MoSe2}; and cycled c) \ce{MoS2} and d) \ce{MoSe2}. SEM images of e) pristine and f) cycled MoSSe. \ce{MoS2} and \ce{MoSe2} clearly have a layered structure, while MoSSe lacks a long-range order.}
\label{Figures/appendix:semmox2cnt}
\end{figure}
\begin{figure}[tbh!]
\centering
\includegraphics[width=\textwidth]{Figures/appendix/mox2tem}
\caption{TEM micrographs of a-b) pristine \ce{MoS2}, and c-d) \ce{MoSe2}. a and b display the presence of layered structure of \ce{MoS2}, c and d display the presence of multiple lattice fringes.}
\label{Figures/appendix:mox2tem}
\end{figure}
\begin{figure}[tbh!]
\centering
\includegraphics[width=\textwidth]{Figures/appendix/mos2edxpt}
\caption{Energy-dispersive X-ray spectroscopy (EDXS) profiles of a) pristine \ce{MoS2} and b) cycled \ce{MoS2} cathode. The cycled cathode shows presence of aluminium and chlorine in its spectra.}
\label{Figures/appendix:mos2edxpt}
\end{figure}
\begin{figure}[tbh!]
\centering
\includegraphics[width=\textwidth]{Figures/appendix/mose2mosseprt}
\caption{EDXS profiles of a) pristine \ce{MoSe2} and b) MoSSe. }
\label{Figures/appendix:mose2mosse}
\end{figure}
\begin{figure}
  \centering
  \includegraphics[width=\textwidth]{Figures/appendix/MoS2XPS}
  \caption{XPS spectra of Mo 3d orbitals in a a) pristine and b) charged \ce{MoS2} cathode and binding energies of S 2p orbital in a a) pristine and b) charged \ce{MoS2} cathode.}
  \label{Figures/appendix:MoS2XPS}
\end{figure}
\begin{figure}[tbh!]
\centering
\includegraphics[width=\textwidth]{Figures/appendix/hbntem.pdf}
\caption{TEM micrographs of h-BN displaying the hexagonal layered structure.}
\label{Figures/appendix:hbntem}
\end{figure}
\begin{figure}[tbh!]
\centering
\includegraphics[width=\textwidth]{Figures/appendix/hbnedxsem.pdf}
\caption{a) EDXS profile and b) SEM image of a pristine h-BN cathode. c) EDXS profile and d) SEM image of a cycled h-BN cathode. The cycled cathode shows agglomeration and its EDXS spectra contains peaks of both aluminium and chlorine.}
\label{Figures/appendix:hbnedxsem}
\end{figure}
\begin{figure}[tbh!]
\centering
\includegraphics[width=\textwidth]{Figures/appendix/cntem.pdf}
\caption{TEM micrographs of g-\ce{C3N4} showing presence of different lattice fringes, which confirm presence of crystallinity.}
\label{Figures/appendix:cntem}
\end{figure}

%	BIBLIOGRAPHY
\printbibliography
\end{document}
