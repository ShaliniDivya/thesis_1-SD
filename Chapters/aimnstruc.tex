\section*{\centering Aims and objectives}
This PhD dissertation aims to discover new cathode materials for rechargeable non-aqueous aluminium-ion batteries with improved specific capacity and battery voltage than state-of-the-art. The goal raises the following research objectives:
\begin{itemize}
    \item to prepare, test and investigate layered-type structures such as transition-metal dichalcogenides, main group oxides, carbides and nitrides, as cathodes and explore their mechanism
    \item to prepare carbonised natural products and other forms of carbon with high surface area (other than graphite) and test them as cathodes 
    \item to establish the mechanism behind all successful cathodes (two-dimensional, carbon-based, etc.) using analytical tools such as X-ray diffraction, Raman spectroscopy, and X-ray photoelectron spectroscopy
    \item to explore the impact of cost-effective solvents and current collectors used in an aluminium-ion battery. 
    
\end{itemize}
\newpage
\newpage
\section*{\centering Thesis structure}
\begin{itemize}
    \item \textbf{Chapter 1}: This chapter gives a brief introduction on batteries and the terms associated with understanding a battery technology. Lithium-ion batteries and its shortcomings are discussed, and a comparison between batteries that currently exist in the market is made. Aluminium-ion batteries, both aqueous and non-aqueous, are introduced and ways of finding new cathode materials that can be used in aluminium-ion batteries is explored.
    \item \textbf{Chapter 2}: This chapter explains the experimental methods carried out to assemble a battery on a lab-scale. Procedures for preparing cathode slurries and electrolytes for an aluminum-ion cell have been briefly described.  
    \item \textbf{Chapter 3}: This chapter discusses the characterisation techniques that were implemented post-mortem, to fully analyse how a battery works. Electrochemical processes such as cyclic voltammetry and galvanostatic charge/ discharge curves have been discussed in detail.   
    \item \textbf{Chapter 4, 5, 6 and 7}: These chapters discuss the new materials that were tested as cathodes in aluminium-ion batteries. With a brief review of the literature, new batteries were made using molybdenum dichalcogenides (Chapter 4), carbon-based materials (Chapter 5) and boron nitride/oxide (Chapter 6) as cathodes. Results of several other two-dimensional materials have been reported in Chapter 7. 
    \item \textbf{Chapter 8}: To find cheaper alternatives to the state-of-the-art, new solvents and current collectors were tested while preparing cathodes and their performance was recorded.   
    \item \textbf{Chapter 9}: This chapter summarises the research findings of chapters 4-8 and provides an outlook for future research possibilities. Many new scientific findings have been made, which need to be studied and analysed in greater detail, so that aluminium-ion batteries can find commercial use.
    \end{itemize}
