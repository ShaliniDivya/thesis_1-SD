% Chapter 1
\chapter{Rechargeable aluminium-ion batteries using boron nitride/ boron oxide as cathode material} % Main chapter title
This has been so far, the most interesting phase in my entire PhD. In this chapter we talk about how \ce{B2O3}, which we considered an impurity in boron nitride, turned out to be the active material in the cathode. This battery produced one of the highest capacities ever reported for non-aqueous aluminium-ion batteries.   \label{BOhBN} % For referencing the chapter elsewhere, use \ref{Chapter1} 
%---------------------------------------------------------------------------------------
% Define some commands to keep the formatting separated from the content 
\newcommand{\keyword}[1]{\textbf{#1}}
\newcommand{\tabhead}[1]{\textbf{#1}}
\newcommand{\code}[1]{\texttt{#1}}
\newcommand{\file}[1]{\texttt{\bfseries#1}}
\newcommand{\option}[1]{\texttt{\itshape#1}}
%----------------------------------------------------------------------------------------
\section{Theory and background}
Graphite has been extensively used as a cathode in AIBs due to its high conductivity and a layered structure. Graphite and hexagonal boron nitride (hBN) are two prominent members of layered materials possessing a hexagonal lattice structure \cite{hod_graphite_2012}. Graphite has non-polar homonuclear C-C intralayer bonds, hBN presents highly polar B-N bonds resulting in different optimal stacking modes of the two materials in the bulk form. Furthermore, the static polarizabilities of the constituent atoms considerably differ from each other, suggesting large differences in the dispersive component of the interlayer bonding. Despite these major differences, both materials present practically identical interlayer distances, Figure \ref{Figures/BOhBN:grpBNcomp}. hBN is popularly known as "white graphene" cite{song-large, zeng-white}. Structurally, a single layer of hBN is very similar to a graphene sheet having a hexagonal backbone where each couple of bonded carbon atoms is replaced by a boron nitride pair, making the two materials isoelectronic. Nevertheless, due to the electronegativity differences between the boron and the nitrogen atoms, the $\pi$ electrons tend to localize around the nitrogen atomic centers, thus forming an insulating material. The nature of bonding between nitrogen and boron differs from the carbon-carbon bonds found in graphite. hBN possesses coordinate bonds resulting from donation of \ce{e-} pair from nitrogen into empty p-orbital of a neighbouring B atom. Each N atom develops a partial positive charge and each B develops a partial negative charge. The partial ionic character of BN bonding makes it a semi-conductor as opposed to a conductor like graphite. 
\begin{figure}[tbh!]
\centering
\includegraphics[width=\textwidth]{Figures/BOhBN/grpBNcomp}
\caption{Honeycomb lattice of a) natural graphite and b) hexagonal boron nitride. They have similar interlayer distance of 3.3\AA.}
\label{Figures/BOhBN:grpBNcomp}
\end{figure}
\section{Results and discussion}
An aluminium-ion cell was assembled using hBN as the cathode and preliminary electrochemical tests were performed, Figure \ref{Figures/BOhBN:hBNiniCDC}. hBN showed very high specific capacities for the first 30 cycles. It recorded a capacity of 270 mAh g$^{-1}$ which decreased to 100 mAh g$^{-1}$ at a discharge potential of 0.6 V. In spite of having a very similar to graphite and not being as conductive, this value was almost 3 times than that of graphite (inset, Figure \ref{Figures/BOhBN:hBNiniCDC}). Repeated experiments from a single batch of hBN cathodes gave us similar results, Figure \ref{Figures/appendix:hBNrepeat}. However, cathodes made from pure hBN (Sigma Aldrich, CAS: 10043-11-5) did not yield similar results after several reruns! This led us to examine the previous sample using XPS, which showed presence of \ce{B2O3}. 

\begin{figure}[tbh!]
\centering
\includegraphics[width=\textwidth]{Figures/BOhBN/hBNiniCDC}
\caption{a) Galvanostatic cycles of hBN at a current density of 50 mA g$^{-1}$ compared with natural graphite (inset). b) Capacity fading of Al/hBN cell within 30 cycles.}
\label{Figures/BOhBN:hBNiniCDC}
\end{figure}
\begin{figure}[tbh!]
\centering
\includegraphics[width=\textwidth]{Figures/BOhBN/hBNSEM}
\caption{a) Coulombic efficiency recorded for 1600 cycles of an Al/hBN pouch cell assembled in IKTS, Germany.}
\label{Figures/BOhBN:hBNSEM}
\end{figure}
\begin{figure}[tbh!]
\centering
\includegraphics[width=\textwidth]{Figures/BOhBN/BNNSCDCCE}
\caption{a) Coulombic efficiency recorded for 1600 cycles of an Al/hBN pouch cell assembled in IKTS, Germany.}
\label{Figures/BOhBN:hBNCDCCE}
\end{figure}
\begin{figure}[tbh!]
\centering
\includegraphics[width=\textwidth]{Figures/BOhBN/BNNSSEM}
\caption{a) Coulombic efficiency recorded for 1600 cycles of an Al/hBN pouch cell assembled in IKTS, Germany.}
\label{Figures/BOhBN:BNNSSEM}
\end{figure}
\begin{figure}[tbh!]
\centering
\includegraphics[width=\textwidth]{Figures/BOhBN/hBNXPS}
\caption{a) Coulombic efficiency recorded for 1600 cycles of an Al/hBN pouch cell assembled in IKTS, Germany.}
\label{Figures/BOhBN:hBNXPS}
\end{figure}
\section{Experimental methods}
\section{Conclusion and future outlook}
