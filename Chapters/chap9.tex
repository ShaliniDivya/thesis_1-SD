% Chapter 9
\section*{Preface}
This chapter summarises the research findings of chapters 4-8 and provides an outlook for future researches. Many new scientific findings have been made, which need to be studied and analysed in greater detail, so that aluminium-ion batteries can find commercial use.
\newpage
\chapter{Summary} 
\label{chap9} 
\section{Conclusion}

In this thesis, new cathode materials were discovered, which are believed to have made important contributions to the field of non-aqueous AIBs. The mechanism for charge storage in transition metal dichalcogenides (chapter \ref{chap4}) and a few carbon-based materials (chapter \ref{chap5}) have been studied in detail. Cheaper alternatives for the existing current collector (molybdenum foil) and solvent (NMP) were explored in Chapter \ref{chap7}. The first of its kind aluminium-ion battery using a combination of h-BN and boric anhydride was tested , which displayed a very high capacity  of >200 mAh g$^{-1}$ (chapter \ref{BOhBN}). A provisional patent has been filed for this material is currently making the initial step towards commercialisation. The studies on cathode materials from the chapters \ref{chap4}, \ref{chap5} and \ref{Figures/BOhBN:othON} have been summarised and concluded below:\\
\begin{itemize}

\item \textbf{Chapter \ref{chap4}} : Many successful battery electrodes are based on 2D-layered materials. Aluminium-ion batteries using molybdenum dichalcogenides: \ce{MoS2}, \ce{MoSe2} and MoSSe as active cathode materials were explored. The batteries showed clear discharge voltage plateaus in the ranges 1.6 - 1.4 V for \ce{MoS2} and \ce{MoSe2}, and 0.6 - 0.5 V for MoSSe. It was found that \ce{MoSe2} performed better than \ce{MoS2}. MoSSe exhibited a higher initial specific capacity over \ce{MoS2} and \ce{MoSe2}, but the energy density was lower than \ce{MoSe2} as the cells became inactive after a few cycles.  \\
CV and XPS results indicated an irreversible phase transition from a semi-conducting 2H phase to a more metallic 1T phase during the first cycle. This transition worked in favour of \ce{MoSe2} and improved its charge-storing capacity. An additional electro-capacitive behaviour was observed in \ce{MoSe2} that added to the its overall capacity.

\item \textbf{Chapter \ref{chap5}} : Four different forms of carbon: activated carbon (AC) from human hair, AC from hemp fibers, a mixture consisting of \ce{C60} and \ce{C70} fullerenes and Super-P carbon black (SPCB) were tested as cathodes for non-aqueous AIBs. Activated carbon derived from human hair proved to be the best carbon-based cathode among all the tested materials. For fullerene extract, it was found that \ce{AlCl4-} anions seep in and out of the gaps present in between the fullerenes, which changed its structure and expanded the crystal lattice slightly during charging (proved via XRD). Hemp fibers and Super-P on the other hand, have a highly disordered structure, and the cathode degraded quickly, further lowering the capacity value after every cycle. In fact, hair displayed a higher specific capacity than conventional Al/graphite battery, and a high battery voltage of 1.92 V with an energy density of 202 Wh kg$^{-1}$. The high battery performance can be attributed to the porosity of the material combined with its high surface area and presence of hetero atoms that enhance its charge-storing capacity. Hair based aluminium-ion batteries would not only be cheaper than state-of-the-art non-aqueous AIBs, but would also bio-degradable.

\item \textbf{Chapter \ref{BOhBN}} : An interesting discovery was made according to which a combination of boron nitride and boron oxide made a suitable cathode material for AIBs. Hexagonal boron nitride and boric anhydride delivered a capacity of > 200 mAh g$^{-1}$. A comprehensive analysis is required to fully establish its working mechanism. Few alterations can be made (e.g. modifying the current collector or using a highly porous conductive carbon additive instead of Super-P) to the existing material that should enhance its low voltage ($\sim$0.6 V), which might further improve its cycle life. 
\end{itemize}

\section{Future outlook}
Lithium-ion batteries (LIBs) are clear winners in terms of performance when talking about energy storage in batteries. In 2019, the Royal Swedish Academy of Sciences awarded the Nobel Prize in Chemistry to John Bannister Goodenough (The University of Texas at Austin, USA), Michael Stanley Whittingham (Binghamton University, State University of New York, USA) and Akira Yoshino (Asahi Kasei Corporation and Meijo University, Japan) for the development of LIBs. They have \enquote{laid the foundation of a wireless, fossil fuel-free society}, and are of the greatest benefit to humankind. Nonetheless, their limited natural resources are a limitation to future trends. 
Despite the remarkable advances, the current state-of-the-art for aluminium-ion batteries still needs improvement for long-term practical uses. One of the long-standing issues in achieving this is the low voltage and short cycle lives. Certainly, similar issues were faced during this thesis with hBN/\ce{B2O3} and a few transition metal oxides (\ce{SnO2} and Prussian blue). However, cathode materials such as \ce{MoSe2}\cite{divya_molybdenum_2019}, activated carbon from human hair, fullerenes, g-\ce{C3N4}, \ce{MoO3} overcame these limitations and showed promising results. 

Following studies of the investigated materials should include exploring porous, high surface area materials as cathodes coated on highly porous current collectors. This way, it may be possible to further stabilize the crystal structures and obtain a longer cycle life (as noted in CFEx and hair batteries). Most of the materials used in this project have a coating thickness of 60 $\mu$m. Similar to what was observed in chapter \ref{BOhBN}, it would be interesting to deposit thin films on the current collector. An interesting method of analysis would be \textit{in-situ} XRD or Raman analysis to study the structural changes taking place inside the material during charge/discharge cycles. For example, in chapter \ref{chap5}, amorphous AC derived from hair turned crystalline after a few cycles (Figure \ref{Figures/chap5fig:XRD} b). Studying this behaviour \textit{in-situ} would contribute to the study of different behaviour of activated carbon in different energy storage devices. Computational studies are critical in this area of research since they are crucial in understanding the battery mechanism \cite{bhauriyal_computational_2017,gao_understanding_2017,bhauriyal_staging_2017,agiorgousis_role_2017}. 

In order to combat climate change, it is important to switch to renewable sources of energy. Batteries are the devices that enable renewable energy storage solutions. They have the potential to charge up the fight against climate change. The battery market is anticipated to be worth \$100 billion by 2025. By 2040, batteries storing solar power for businesses and households will account for 57 \% of the world’s energy storage capacity. However, the social and environmental problems of lithium-ion batteries are not hidden. There is trace amount of graphite present in cell phones, but in China (the leading producer of graphite), its in their air. Despite steeply falling costs, LIBs are still far too expensive. Batteries must cost less than one-fifth the likely minimum cost of LIBs ($\sim$\$100 a kWh). There has been significant improvement in AIB research in recent years. It has become imperative to find a better battery, which is why AIBs are the perfect choice in replacing lithium-ion.