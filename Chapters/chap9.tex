% Chapter 9
\section*{Preface}
This chapter summarises the research findings of chapters 4-8 and provides an outlook for future researches. Many new scientific findings have been made, which need to be studied and analysed in greater detail, so that aluminium-ion batteries can find commercial use.
\newpage
\chapter{Conclusion and future outlook} 
\label{chap9} 
\section*{Conclusion}

In this thesis, new cathode materials were discovered (reported in chapters \ref{chap4}-\ref{BOhBN}), which, we believe, have made important contributions to the field of non-aqueous aluminium batteries. We have studied in detail the mechanism for charge storage in transition metal dichalcogenides (chapter \ref{chap4})  and a few carbon-based materials (chapter \ref{chap5}). To further reduce the cost of these batteries, we looked for cheaper alternatives and tested a number of solvents and current foils that can be used during cell assembly (chapter \ref{chap7}). DMSO and titanium foil turned out to be suitable options. Finally, we demonstrated the first of its kind aluminium-ion battery using a combination of nitrides and oxides, especially hBN and \ce{B2O3} that displayed a very high capacity  of >200 mAh g$^{-1}$ (chapter \ref{BOhBN}). The material has been patented and is making the initial step towards commercialisation.The studies on cathode materials from the chapters \ref{chap4}, \ref{chap5} and \ref{Figures/BOhBN:othON} have been summarised and concluded below:\\
\begin{itemize}

\item \textbf{Chapter \ref{chap4}} : Many successful battery electrodes are based on 2D-layered materials. Aluminium-ion batteries using molybdenum dichalcogenides: \ce{MoS2}, \ce{MoSe2} and MoSSe as active cathode materials were explored. The batteries showed clear discharge voltage plateaus in the ranges 1.6 - 1.4 V for \ce{MoS2} and \ce{MoSe2}, and 0.6 - 0.5 V for MoSSe. \ce{MoS2} and \ce{MoSe2} have similar crystal structures, interestingly we found that \ce{MoSe2} performed better than \ce{MoS2}. MoSSe exhibited a higher specific capacity over \ce{MoS2} and \ce{MoSe2}, but the energy density was lower than \ce{MoSe2}.  \\
CV and XPS results indicated an irreversible phase transition from a semi-conducting 2H phase to a more metallic 1T phase. This transition worked in favour of \ce{MoSe2} and its capacity increased. XRD, XPS and Raman results supported the hypothesis that \ce{AlCl4-} intercalated reversibly into \ce{MoSe2}. An additional electro-capacitive behaviour was observed in \ce{MoSe2} that added to the its overall capacity.

\item \textbf{Chapter \ref{chap5}} : We compared four different forms of carbon: activated carbon (AC) from human hair, AC from hemp fibers, a mixture consisting of \ce{C60} and \ce{C70} fullerenes and Super-P carbon black (SPCB) as cathodes for non-aqueous aluminium-ion batteries. These materials differed in their general structure, porosity and morphology. The fullerenes displayed a crystalline structure, whereas hemp fibers, SPCB and hair were amorphous in nature. Activated carbon derived from human hair proved to be the best carbon-based cathode among all the tested materials. Intercalation and deintercalation of \ce{AlCl4-} takes place in the very few graphitic layers present in it. We found that in CFEx, \ce{AlCl4-} anions seep in and out of the gaps in between the fullerenes changing its structure and slightly expanding the crystal lattice during charging. Moreover, fullerenes maintain their structural integrity and coulombic efficiency throughout the cycles. Hemp fibers and Super-P on the other hand, have a highly amorphous structure, which degraded after every cycle, resulting in a low capacity value. In fact, hair displays a higher specific capacity than conventional Al/graphite battery, and a high battery voltage of 1.92 V with an energy density of 202 Wh kg$^{-1}$. The high battery performance can be attributed to the porosity of the material combined with high surface area and hetero-atom doping effects, combining surface-based non-Faradaic and Faradaic contributions. Hair based aluminium-ion batteries would not only be cheaper than state of the art, but would also be one of its kind bio-degradable materials in the battery industry.

\item \textbf{Chapter \ref{BOhBN}} : An interesting discovery was made according to which a combination of nitrides and oxides made a suitable cathode material for AIBs. Hexagonal boron nitride and boric anhydride worked as the best combination and delivered a capacity of > 200 mAh g$^{-1}$. A comprehensive analysis is required to fully establish its working mechanism. Furthermore, the low voltage of the Al/hBN/\ce{B2O3} cell ($\sim$0.6 V) can be increased by making a few changes, like modifying the current collector or using a highly porous conductive carbon additive, which might further improve its cycle life. 
\end{itemize}

\section{Future outlook}
Lithium-ion batteries (LIBs) are clear winners in terms of performance. In 2019, the Royal Swedish Academy of Sciences decided to award the Nobel Prize in Chemistry to John Bannister Goodenough (The University of Texas at Austin, USA), Michael Stanley Whittingham (Binghamton University, State University of New York, USA) and Akira Yoshino (Asahi Kasei Corporation and Meijo University, Japan) for the development of lithium-ion batteries. LIBs have revolutionised our lives since they first entered the market in 1991. They have \enquote{laid the foundation of a wireless, fossil fuel-free society}, and are of the greatest benefit to humankind. Nonetheless, their limited natural resources are a limitation to future trends. 
Despite the remarkable advances, the current state-of-the-art for aluminium-ion batteries still needs improvement for long-term practical uses. One of the long-standing issues in achieving this is the low voltage and short cycle lives. Certainly, we faced similar issues with hBN/\ce{B2O3} and a few transition metal oxides (\ce{SnO2}, shown in Figure \ref{Figures/chap6fig:SnO2perfCDC} and Prussian blue, shown in Figure \ref{Figures/chap6fig:pbCDC2}). However, cathode materials such as \ce{MoSe2}\cite{divya_molybdenum_2019-1}, activated carbon from human hair, fullerenes, g-\ce{C3N4}, \ce{MoO3}, etc. with their promising results, bring an opportunity for us to transit from the present lithium battery technology to aluminium battery technology as illustrated in Figure X.

Following studies of the investigated materials could include exploring porous, high surface area materials as cathodes coated on highly porous current collectors. This way, it may be possible to stabilize the crystal structures and obtain a longer cycle life (as noted in CFEx and hair batteries). Most of the materials used in this project have a coating thickness of 60 $\mu$m. Similar to what we observed in chapter \ref{BOhBN}, it would be interesting to deposit thin films of other cathode materials on the current collector. It would eliminate the possibility of multiple interfaces which are present in a composite cathode and enable higher capacity cycling. Another interesting method of analysis would be \textit{in-situ} XRD or Raman analysis to study the structural changes taking place inside the material during charge/discharge cycles. Also, we need to look for an electrolyte, which enhances the electrochemical intercalation and deintercalation process with a wide enough electrochemical stability window in the presence of Al anode \cite{jayaprakash_rechargeable_2011} and increase the number of ions that participate in redox reactions.\\
In addition, I am especially interested in studying the behavior of a few materials such as fullerenes, activated carbon from human hair and electrospun \ce{SnO2} fibers before and after galvanostaic charge/ discharge cycles. For example, in chapter \ref{chap5}, we saw that highly amorphous AC derived from hair turned crystalline after a few cycles (Figure \ref{Figures/chap5fig:xrd}b). Studying this behaviour \textit{in-situ} would contribute to the study of different behaviour of activated carbon in different energy storage devices! Computational studies are critical in this area of research since they are crucial in understanding the battery mechanism \cite{bhauriyal_computational_2017,gao_understanding_2017-1,bhauriyal_staging_2017-1,agiorgousis_role_2017}. 

In order to combat climate change, it is important to reduce greenhouse gases. For this reason, electric vehicle market is on a very steep rise. Therefore, it is crucial to look into high performing batteries. There has been significant improvement in AIB research in
recent years, which is why these low cost and recyclable battery systems are expected to rise in the near future.