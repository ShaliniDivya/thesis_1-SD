% Chapter 6

\chapter{Aluminium-ion batteries using oxides and other materials as cathodes} % Main chapter title

\label{Chapter6} % For referencing the chapter elsewhere, use \ref{Chapter1} 

%----------------------------------------------------------------------------------------

% Define some commands to keep the formatting separated from the content 
\newcommand{\keyword}[1]{\textbf{#1}}
\newcommand{\tabhead}[1]{\textbf{#1}}
\newcommand{\code}[1]{\texttt{#1}}
\newcommand{\file}[1]{\texttt{\bfseries#1}}
\newcommand{\option}[1]{\texttt{\itshape#1}}

%----------------------------------------------------------------------------------------
\section{Theory and background}
\subsection*{Tin oxide, \ce{SnO2}}
Tin (IV) oxide is an inorganic compound with the formula \ce{SnO2}. It finds abundant use as a colorant, polishing powder, for glass coatings and as a sensor for combustible gases. It is a colourless, diamagnetic, amphoteric (reacts with both acid and base) solid. Due to its high theoretical capacity ($\approx$ 782mAh g$^{-1}$), safe handling and environmental-friendliness, \ce{SnO2} is a good cathode candidate for AIBs. Miyasak \textit{et al.} used tin-based amorphous composite oxide (TCO) contains Sn(II)-O as the active center for lithium-insertion and other glass-forming elements, which made an oxide network \cite{idota_tin-based_1997}. In its charged state, TCO accepted 8 moles of lithium ions per unit mole. However, it underwent large volumetric changes $\approx$ 300{\%}, which caused slow diffusion kinetics. It further resulted in agglomeration and pulverisation of cathodes after continuous charge-discharge cycles. This resulted in capacity fading. To improve the battery performance, carbon-based materials were added to \ce{SnO2}. It not only provided a high surface area that would act as a buffer for volume expansion/shrinkage, but also would improve the conductivity of the active material. Based on this concept, Nowak \textit{et al.} \cite{nowak_composites_2018}
Carbon is known not to react with tin and does not form tin carbide \cite{}. It is very crucial in terms of utilizing carbon as a buffer in preventing electric contact loss of the tin negative electrode with the current collector\cite{}. The historical background for modification of tin oxides with carbonaceous material origins from Lee \textit{et al.} who obtained synthetic graphite modified by a highly dispersed tin oxide which improved the cell's performance \cite{navarro-suarez_2d_nodate}. 
It was observed that carbon coating on \ce{SnO2} surface prevents their agglomeration and volume expansion. A smaller particle size (like nanorods\cite{}, nanosheets\cite{}, nanospheres\cite{}, nanowires\cite{}, nanotubes\cite{}, nanoflowers\cite{}) or/and a porous structure of the active material would further alleviate the contact area between the electrode and the electrolyte, accelerating the transport of ions. 
We tested SnO2 and Sb-SnO2 as cathodes for AIBs. Antimony is commonly used as a n-type dopant which increases the conductivity of SnO2 by increasing its band gap. 
\subsection*{Molybdenum trioxide, \ce{MoO3}}

%Boron nitride is a thermally and chemically resistant compound. It exists in various crystalline forms that are isoelectronic to a similarly structured carbon lattice and therefore is also called 'inorganic graphite'. The hexagonal form corresponding to graphite is the most stable and soft among BN polymorphs, and is therefore used as a lubricant and an additive to cosmetic products. The most stable crystalline form is the hexagonal one, also called hBN, $\alpha$BN, and graphitic boron nitride. Hexagonal boron nitride has a layered structure similar to graphite. Within each layer, boron and nitrogen atoms are bound by strong covalent bonds, whereas the layers are held together by weak van der Waals forces. While graphite has non-polar homonuclear C−C intralayer bonds, h-BN presents highly polar B−N bonds resulting in different optimal stacking modes of the two materials in the bulk form. Furthermore, the static polarizabilities of the constituent atoms considerably differ from each other, suggesting large differences in the dispersive component of the interlayer bonding. Despite these major differences, both materials present practically identical interlayer distances\cite{}. 
\subsection*{graphitic Carbon Nitride, g-\ce{C3N4}}
\subsection*{Prussian blue, \ce{C19Fe7N18}}
\section{Experimental methods}
Same as discussed in Chapter\ref{chap3}.
\section{Results and discussions}
\section{Conclusion and future work}