% Chapter 7

\chapter{Aluminium-ion batteries using different solvents while slurry preparation} % Main chapter title

\label{Chapter7} % For referencing the chapter elsewhere, use \ref{Chapter1} 

%----------------------------------------------------------------------------------------

% Define some commands to keep the formatting separated from the content 
\newcommand{\keyword}[1]{\textbf{#1}}
\newcommand{\tabhead}[1]{\textbf{#1}}
\newcommand{\code}[1]{\texttt{#1}}
\newcommand{\file}[1]{\texttt{\bfseries#1}}
\newcommand{\option}[1]{\texttt{\itshape#1}}

%----------------------------------------------------------------------------------------
\section{Theory and background}
%Commercial battery electrodes are manufactured by casting a slurry onto a metallic current collector. The slurry contains active material, conductive carbon, and binder in a solvent. The binder, most commonly polyvinylidene fluoride (PVDF), are pre-dissolved in the solvent, most commonly N-Methyl-2-pyrrolidone (NMP). During mixing, the polymer binder flows around and coat the active material and carbon particles1,2,3,4,5,6,7,8,9. After uniformly mixing, the resulting slurry is cast onto the current collector and must be dried. Evaporating the solvent to create a dry porous electrode is needed to fabricate the battery. Drying can take a wide range of time with some electrodes taking 12–24 hours at 120 °C to completely dry5,10. In commercial applications, an NMP recovery system must be in place during the drying process to recover evaporated NMP due to the high cost and potential pollution of NMP11,12. While the recovery system makes the entire process more economical it does require a large capital investment. Less expensive and environmentally friendly solvents, such as aqueous based slurries, could eliminate the large capital cost of the recovery system but the electrode would still require a time and energy demanding drying step9,10,13,14,15,16. Uncoventional manufacturing methods have also been used to create battery electrodes. Solvent based electrostatic spray deposition has been used to coat current collectors with electrode material17,18,19. This is achieved by adding high voltage to the deposition nozzle and grounding the current collector, which causes the deposition material to become atomized at the nozzle and drawn to the current collector. Electrodes constructed with this method exhibit similar characteristics as slurry-cast electrodes and have similar disadvantages in that they also require a time and energy intesive drying process (2 hours at 400 °C)19.
\section{Experimental methods}
\section{Results and discussions}
\section{Conclusion and future work}