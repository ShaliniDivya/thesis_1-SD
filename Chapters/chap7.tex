% Chapter 7

\chapter{Aluminium-ion batteries using different solvents while slurry preparation} % Main chapter title

\label{Chapter7} % For referencing the chapter elsewhere, use \ref{Chapter1} 

%----------------------------------------------------------------------------------------

% Define some commands to keep the formatting separated from the content 
\newcommand{\keyword}[1]{\textbf{#1}}
\newcommand{\tabhead}[1]{\textbf{#1}}
\newcommand{\code}[1]{\texttt{#1}}
\newcommand{\file}[1]{\texttt{\bfseries#1}}
\newcommand{\option}[1]{\texttt{\itshape#1}}

%----------------------------------------------------------------------------------------
\section{Theory and background}
Battery electrodes are manufactured by casting a slurry onto a metallic current collector. The slurry contains active material, conductive carbon, and binder in a solvent. The binder, most commonly polyvinylidene fluoride (PVDF), are pre-dissolved in the solvent, most commonly N-Methyl-2-pyrrolidone (NMP). During mixing, the polymer binder flows around and coat the active material and carbon particles\cite{ludwig_solvent-free_2016}. After uniformly mixing, the resulting slurry is cast onto the current collector and must be dried. Evaporating the solvent to create a dry porous electrode is needed to fabricate the battery. Drying can take a wide range of time with some electrodes taking 12–24hours at 120C to completely dry. In commercial applications, an NMP recovery system must be in place during the drying process to recover evaporated NMP due to the high cost and potential pollution of NMP. While the recovery system makes the entire process more economical it does require a large capital investment. Less expensive and environmentally friendly solvents, such as aqueous based slurries, could eliminate the large capital cost of the recovery system but the electrode would still require a time and energy demanding drying step. Uncoventional manufacturing methods have also been used to create battery electrodes. Solvent based electrostatic spray deposition has been used to coat current collectors with electrode material. This is achieved by adding high voltage to the deposition nozzle and grounding the current collector, which causes the deposition material to become atomized at the nozzle and drawn to the current collector. Electrodes constructed with this method exhibit similar characteristics as slurry-cast electrodes and have similar disadvantages in that they also require a time and energy intesive drying process.
\section{Experimental methods}
\section{Results and discussions}
The interface between the electrolyte and the cathode heavily influences the bat-tery performance as well as the bulk composition of the cathode. Both is depen-dent on the solvent used in the manufacturing process determining the mixture quality and the surface morphology. The standard solvent used was NMP offering a homogeneous mixture and a smooth surface after the annealing. Considering the price of 150 e/L a cheaperalternative is needed for in-dustrial scalability. Table 4.2 shows the list of solvents investigated. The relative mixture of active material, conductive carbon, binder and solvent was kept con-stant throughout the experiment series.As discussed in section 3 the solution preparation starts with the mixture of solvent and binder. Already in this early stage of the process it became obvious that water as a solvent was not suitable. The PVDF binder was found to be immiscible in water therefore resulting in large clusters present in the mixture as shown in figure 7 a). No further investigation was conducted towards water as a solvent.
It has to be noted that for the remaining tested solvents visible differences could be spotted. On the one hand, DMSO, DMF, DMA and NMP resulted in transparent solutions after mixing (figure 7 b), bottom), while Ethanol, Isopropanol, Toluene,Acetone, Methanol and Butanol resulted in a milky solution (figure 7 b), top).
A limiting factor observed during the casting process was the boiling point of the solvents. Low boiling points of Acetone, Ethanol, Methanol Isopropanol and Acetonitrile caused problems while casting. Casting of the solution lead to the desired liquid thin film, however the creation of this rather high surface area to volume ratio lead to high evaporation rates causing rapid drying of the film. Tensions built up by this fast solvent evaporation resulted in cracks occurring in the film making it unsuitable for further processing as shown in figure 17.
It can be seen that the crack formation due to fast solvent evaporation is stepwise evolving going from a smooth wet surface 17a) to a cracked dry film 17. This process of drying took, dependent on the solvent, between one and two minutes. The result however stayed unchanged for all solvents with low boiling points.
The remaining slurries based on DMSO, DMF, DMA, NMP, Butanol and Toluene were casted and dried according to the process described in section 3. It was observed that solutions casted with Toluene and Butanol resulted in a
cracked film quality similar to the one shown in figure 17 d) while the solutions based on DMSO, DMF, DMA and NMP lead to smooth crack-free films as shown in 9.
The cathodes using DMSO, DMF, DMA and NMP were tested on their per-formance by performing charge/discharge cycles observing their behaviour with increasing cycle numbers. It can be seen that the trends between the solvents is similar throughout the cycles while the total capacity is decreasing for all as it is expected with increasing cycle number. Figures 10 a)-c) show that the overall capacity of the DMA based cathode is highest in the earlier cycles. However, it can also be observed that its decrease in capacity is the highest as well and especially higher compared to the capacity loss of NMP. A possible explanation could be deducted from the Coulombic efficiency shown in figure 10 d). With efficiencies constantly higher than 100\% it is very likely that a side reaction is taking place during the charge/discharge cycles that is providing extra energy from the system. Usually this is a sign of degradation of battery components. Since the only varied factor is the solvent, it is likely that some residuals of the DMA is causing this reaction. Due to this degradation process DMA is not considered as a feasible alternative.
Contrary to DMA, DMSO based hBN cathodes are showing good Coulombic efficiencies around 100\%. The overall efficiencies throughout the cycles are how-ever lower compared to the other tested cathodes making it not ideal for high capacity applications.
When comparing DMF and NMP based cathodes it can be observed that in the early stage both cathodes perform equally well. With increasing cycle numbers however the capacity of the DMF based cathode is decreasing significantly faster compared to the NMP based cathode. Also comparing the Coulombic efficiencies yields that NMP has a constantly slightly higher efficiency around 100\% com-pared to DMF with around 90\%.
Considering the said performance comparisons, it can be said that DMF might be an alternative for NMP with limitations to rather small cycle numbers which can be improved. When going for cycle numbers higher than 25 which is typically the case in order to determine the cycle life of a battery, NMP is the best and most reliable option to go with. 
The topography of a collection of tested slurries was observed in order to determine whether performance changes could be explained by the quality of mixing and thus surface roughness of the cathode layers. As can be seen by com-paring the SEM pictures in figure 11 a), c), e), corresponding to NMP, DMSO and DMF based cathode layers magnified 500 times respectively, observable difference can be pointed out. While NMP and DMF are similar in appearance, containing agglomerates at the surface, DMSO is seen to be almost completely free of such agglomerated hBN structures. This observation can be confirmed when looking at an even smaller scale with magnifications of 10K. As figures 11 b), d) and e) corresponding to NMP, DMSO and DMF based cathode layers respectively, a similar conclusion can be drawn. Bigger agglomerates are present for NMP and DMF while such structures are missing in the DMSO based cathode.
This concludes that DMSO is the best solvent regarding the mixing properties. It doesn’t however explain the fact that DMSO based cathode layers appear to be the worst performing in battery systems.

\begin{table}
\caption{List of materials used as current collectors in increasing order of their marker price.} \label{t1}
\begin{center}
 \begin{tabular}{|ccc|} 
 \hline
 \textbf{Material} & \textbf{Thickness (mm)} & \textbf{Price (\$)} \\
 \hline
Steel & 0.1 & 0.6 \\ 
Carbon paper  & <0.1 &  \\
Aluminium & 0.1 & 1.76 \\
Brass & 1 & 2.93 \\
Copper & 0.1 & 5.73 \\ 
Nickel foam & 1 & 10.37 \\
Molybdenum & 0.1 & 22.91 \\
 \hline
\end{tabular}
\end{center}
\end{table}

\section{Conclusion and future work}